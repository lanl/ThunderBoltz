\begin{savenotes}\sphinxattablestart
\sphinxthistablewithglobalstyle
\sphinxthistablewithnovlinesstyle
\centering
\begin{tabulary}{\linewidth}[t]{\X{1}{2}\X{1}{2}}
\sphinxtoprule
\sphinxtableatstartofbodyhook
\sphinxAtStartPar
\sphinxcode{\sphinxupquote{E}}
&
\sphinxAtStartPar
(float) The electric field component (V/m) in the \(z\) direction, which can change in AC scenarios.
\\
\sphinxhline
\sphinxAtStartPar
\sphinxcode{\sphinxupquote{MEe}}
&
\sphinxAtStartPar
(float) The mean energy (eV) of the species at index \(0\) (usually electrons), computed as \(\langle\epsilon\rangle = \frac{m_0}{2N_0}\sum_{i=1}^{N_0}v_{0i}^2\) where \(m_0\) and \(N_0\) are the mass and particle count of the \(0^{\rm th}\) species, and \(v_{0i}\) is the velocity vector of the \(i^{\rm th}\) particle of species \(0\).
\\
\sphinxhline
\sphinxAtStartPar
\sphinxcode{\sphinxupquote{step}}
&
\sphinxAtStartPar
(int) The number of time steps elapsed in the simulation, with \sphinxcode{\sphinxupquote{t}} \(=0\) corresponding to \sphinxcode{\sphinxupquote{step}} \(=0\), and with \sphinxcode{\sphinxupquote{t}} = \sphinxcode{\sphinxupquote{DT}} corresponding to \sphinxcode{\sphinxupquote{step}} \(=1\).
\\
\sphinxhline
\sphinxAtStartPar
\sphinxcode{\sphinxupquote{t}}
&
\sphinxAtStartPar
(float) The time (s) elapsed in the simulation.
\\
\sphinxbottomrule
\end{tabulary}
\sphinxtableafterendhook\par
\sphinxattableend\end{savenotes}

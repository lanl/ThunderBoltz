%% Generated by Sphinx.
\def\sphinxdocclass{report}
\documentclass[letterpaper,10pt,english,openany,oneside]{sphinxmanual}
\ifdefined\pdfpxdimen
   \let\sphinxpxdimen\pdfpxdimen\else\newdimen\sphinxpxdimen
\fi \sphinxpxdimen=.75bp\relax
\ifdefined\pdfimageresolution
    \pdfimageresolution= \numexpr \dimexpr1in\relax/\sphinxpxdimen\relax
\fi
%% let collapsible pdf bookmarks panel have high depth per default
\PassOptionsToPackage{bookmarksdepth=5}{hyperref}

\PassOptionsToPackage{booktabs}{sphinx}
\PassOptionsToPackage{colorrows}{sphinx}

\PassOptionsToPackage{warn}{textcomp}
\usepackage[utf8]{inputenc}
\usepackage{authblk}
\ifdefined\DeclareUnicodeCharacter
% support both utf8 and utf8x syntaxes
  \ifdefined\DeclareUnicodeCharacterAsOptional
    \def\sphinxDUC#1{\DeclareUnicodeCharacter{"#1}}
  \else
    \let\sphinxDUC\DeclareUnicodeCharacter
  \fi
  \sphinxDUC{00A0}{\nobreakspace}
  \sphinxDUC{2500}{\sphinxunichar{2500}}
  \sphinxDUC{2502}{\sphinxunichar{2502}}
  \sphinxDUC{2514}{\sphinxunichar{2514}}
  \sphinxDUC{251C}{\sphinxunichar{251C}}
  \sphinxDUC{2572}{\textbackslash}
\fi
\usepackage{cmap}
\usepackage[T1]{fontenc}
\usepackage{amsmath,amssymb,amstext}
\usepackage{babel}



\usepackage{tgtermes}
\usepackage{tgheros}
\renewcommand{\ttdefault}{txtt}



\usepackage[Bjarne]{fncychap}
\usepackage{sphinx}

\fvset{fontsize=auto}
\usepackage{geometry}


% Include hyperref last.
\usepackage{hyperref}
% Fix anchor placement for figures with captions.
\usepackage{hypcap}% it must be loaded after hyperref.
% Set up styles of URL: it should be placed after hyperref.
\urlstyle{same}


\usepackage{sphinxmessages}
\setcounter{tocdepth}{1}



\title{ThunderBoltz}
\date{Oct 15, 2023}
\release{0.1}
\author{Ryan Park, Brett Scheiner, Mark Zammit}
\affil{Los Alamos National Laboratory, Los Alamos, NM, 87545}
\newcommand{\sphinxlogo}{\vbox{}}
\renewcommand{\releasename}{Release}
\makeindex
\begin{document}

\ifdefined\shorthandoff
  \ifnum\catcode`\=\string=\active\shorthandoff{=}\fi
  \ifnum\catcode`\"=\active\shorthandoff{"}\fi
\fi

\pagestyle{empty}
\sphinxmaketitle
\pagestyle{plain}
\sphinxtableofcontents
\pagestyle{normal}
\phantomsection\label{\detokenize{index::doc}}


\sphinxAtStartPar
This documentation includes some simple tutorials for using
the ThunderBoltz plasma simulation package and a complete
public API reference.


\chapter{Installation}
\label{\detokenize{index:installation}}
\sphinxAtStartPar
For now, the code must be downloaded from a repository.
Use the following command to clone the code into a local repository.

\begin{sphinxVerbatim}[commandchars=\\\{\}]
git\PYG{+w}{ }clone\PYG{+w}{ }git@gitlab.com/Mczammit/thunderboltz.git
\end{sphinxVerbatim}

\sphinxAtStartPar
You may need to set up SSH keys in order to access gitlab. See the
\sphinxhref{https://docs.gitlab.com/ee/user/ssh.html}{Gitlab SSH Guide} to
set up access to GitLab repositories.

\sphinxAtStartPar
The basic ThunderBoltz functionality is available either
as an executable in \sphinxcode{\sphinxupquote{bin/thunderboltz.bin}} or can be compiled from the
source in \sphinxcode{\sphinxupquote{src/thunderboltz}}. To install the Python interface, run the
\sphinxcode{\sphinxupquote{install.sh}} script from the root directory to install the python package.

\begin{sphinxVerbatim}[commandchars=\\\{\}]
./install.sh
\end{sphinxVerbatim}

\begin{sphinxadmonition}{warning}{Warning:}
\sphinxAtStartPar
This will upgrade \sphinxhref{https://pypi.org/project/pip/}{pip} and install specific versions of python packages,
so create an environment if you are concerned with python package overwrite.
\end{sphinxadmonition}


\chapter{Tutorials}
\label{\detokenize{index:tutorials}}
\sphinxAtStartPar
See the {\hyperref[\detokenize{quickstart::doc}]{\sphinxcrossref{\DUrole{doc}{Quick Start Guide}}}} to go through a brief tutorial
setting up a simple calculation and interpreting the results.

\sphinxAtStartPar
See {\hyperref[\detokenize{cs_guide::doc}]{\sphinxcrossref{\DUrole{doc}{Preparing Cross Sections}}}} for a tutorial on the various
ways to obtain and manipulate cross sections for ThunderBoltz simulations.

\sphinxAtStartPar
See {\hyperref[\detokenize{multi_guide::doc}]{\sphinxcrossref{\DUrole{doc}{Running Multiple Calculations}}}} for a quick guide on
how to vary simulation parameters and easily run simulations in
parallel.

\sphinxAtStartPar
See {\hyperref[\detokenize{ext_guide::doc}]{\sphinxcrossref{\DUrole{doc}{Extracting Results}}}} for details on easily parsing,
plotting, and exporting data from many ThunderBoltz simulations at once.

\sphinxstepscope


\section{Quick Start Guide}
\label{\detokenize{quickstart:quick-start-guide}}\label{\detokenize{quickstart::doc}}

\subsection{Running the Simulation}
\label{\detokenize{quickstart:running-the-simulation}}
\sphinxAtStartPar
In this guide we will set up a simple calculation, run the program, and
interpret the results. This guide assumes that the python ThunderBoltz
package has already been {\hyperref[\detokenize{index::doc}]{\sphinxcrossref{\DUrole{doc}{Installed}}}}.

\sphinxAtStartPar
For the first example, we will set up a Helium gas calculation,
since the package already has a Helium model built\sphinxhyphen{}in. First, set up
a file that will hold the python script to drive the simulation.
In a new file \sphinxcode{\sphinxupquote{run\_example.py}}, write the following:

\begin{sphinxVerbatim}[commandchars=\\\{\}]
\PYG{k+kn}{import} \PYG{n+nn}{os} \PYG{c+c1}{\PYGZsh{} Operating system interface for making directories}

\PYG{k+kn}{from} \PYG{n+nn}{pytb} \PYG{k+kn}{import} \PYG{n}{ThunderBoltz} \PYG{c+c1}{\PYGZsh{} Import the main simulation object}
\PYG{k+kn}{from} \PYG{n+nn}{pytb}\PYG{n+nn}{.}\PYG{n+nn}{input} \PYG{k+kn}{import} \PYG{n}{He\PYGZus{}TB}  \PYG{c+c1}{\PYGZsh{} This is a built\PYGZhy{}in He model preset}

\PYG{c+c1}{\PYGZsh{} First we will make a folder in the current directory to house}
\PYG{c+c1}{\PYGZsh{} The simulation output / logging files.}
\PYG{n}{os}\PYG{o}{.}\PYG{n}{makedirs}\PYG{p}{(}\PYG{l+s+s2}{\PYGZdq{}}\PYG{l+s+s2}{example\PYGZus{}sim}\PYG{l+s+s2}{\PYGZdq{}}\PYG{p}{)}

\PYG{n}{tb} \PYG{o}{=} \PYG{n}{ThunderBoltz}\PYG{p}{(}

     \PYG{c+c1}{\PYGZsh{} Specify internal ThunderBoltz settings.}
     \PYG{n}{DT}\PYG{o}{=}\PYG{l+m+mf}{1e\PYGZhy{}10}\PYG{p}{,}             \PYG{c+c1}{\PYGZsh{} Time step of 0.1 ns.}
     \PYG{n}{NS}\PYG{o}{=}\PYG{l+m+mi}{30000}\PYG{p}{,}             \PYG{c+c1}{\PYGZsh{} Number of time steps.}
     \PYG{n}{L}\PYG{o}{=}\PYG{l+m+mf}{1e\PYGZhy{}6}\PYG{p}{,}               \PYG{c+c1}{\PYGZsh{} The cell size (m).}
     \PYG{n}{NP}\PYG{o}{=}\PYG{p}{[}\PYG{l+m+mi}{10000}\PYG{p}{,} \PYG{l+m+mi}{1000}\PYG{p}{]}\PYG{p}{,}     \PYG{c+c1}{\PYGZsh{} 1e4 electrons, 1e3 Helium macroparticles.}
     \PYG{n}{FV}\PYG{o}{=}\PYG{p}{[}\PYG{l+m+mi}{20000}\PYG{p}{,} \PYG{l+m+mi}{10000}\PYG{p}{,} \PYG{l+m+mi}{0}\PYG{p}{]}\PYG{p}{,} \PYG{c+c1}{\PYGZsh{} Dump the electron velocities on steps 20000 and 30000.}
     \PYG{c+c1}{\PYGZsh{} ... etc.}

     \PYG{c+c1}{\PYGZsh{} Specify additional python interface settings in the same way.}
     \PYG{n}{Ered}\PYG{o}{=}\PYG{l+m+mi}{100}\PYG{p}{,}       \PYG{c+c1}{\PYGZsh{} Fix the reduced field at 100 Townshend.}
     \PYG{n}{eesd}\PYG{o}{=}\PYG{l+s+s2}{\PYGZdq{}}\PYG{l+s+s2}{uniform}\PYG{l+s+s2}{\PYGZdq{}}\PYG{p}{,} \PYG{c+c1}{\PYGZsh{} Use the uniform electron energy sharing ionization model.}
     \PYG{n}{eadf}\PYG{o}{=}\PYG{l+s+s2}{\PYGZdq{}}\PYG{l+s+s2}{default}\PYG{l+s+s2}{\PYGZdq{}}\PYG{p}{,} \PYG{c+c1}{\PYGZsh{} Use isotropic elastic scattering.}
     \PYG{n}{egen}\PYG{o}{=}\PYG{k+kc}{False}\PYG{p}{,}     \PYG{c+c1}{\PYGZsh{} Do not generate secondary electrons in ionization events.}
     \PYG{c+c1}{\PYGZsh{} ... etc.}


     \PYG{c+c1}{\PYGZsh{}\PYGZsh{}\PYGZsh{} The package comes with a built\PYGZhy{}in Helium model that can be controlled}
     \PYG{c+c1}{\PYGZsh{}\PYGZsh{}\PYGZsh{} with the following parameters.}

     \PYG{c+c1}{\PYGZsh{} This indicates use of the built in He model. It will automatically}
     \PYG{c+c1}{\PYGZsh{} set up the masses, charges, and cross sections for a fixed\PYGZhy{}background}
     \PYG{c+c1}{\PYGZsh{} Helium calculation.}
     \PYG{n}{indeck}\PYG{o}{=}\PYG{n}{He\PYGZus{}TB}\PYG{p}{,}
     \PYG{c+c1}{\PYGZsh{} You may also specify up to what principle quantum number the}
     \PYG{c+c1}{\PYGZsh{} excitation cross sections will go.}
     \PYG{n}{n}\PYG{o}{=}\PYG{l+m+mi}{4}\PYG{p}{,}
     \PYG{c+c1}{\PYGZsh{} Or how many points to sample the cross section model.}
     \PYG{n}{nsamples}\PYG{o}{=}\PYG{l+m+mi}{300}\PYG{p}{,}

     \PYG{c+c1}{\PYGZsh{} Finally, specify a simulation directory where logging and output}
     \PYG{c+c1}{\PYGZsh{} files will be written.}
     \PYG{n}{directory}\PYG{o}{=}\PYG{l+s+s2}{\PYGZdq{}}\PYG{l+s+s2}{example\PYGZus{}sim}\PYG{l+s+s2}{\PYGZdq{}}\PYG{p}{,}
\PYG{p}{)}


\PYG{c+c1}{\PYGZsh{} At this point, the ThunderBoltz object has been configured and is ready}
\PYG{c+c1}{\PYGZsh{} to run. To do this, just call the \PYGZdq{}run\PYGZdq{} method. This will write the}
\PYG{c+c1}{\PYGZsh{} necessary input files, compile the program from package source into the}
\PYG{c+c1}{\PYGZsh{} simulation directory, and execute the program in a subprocess.}
\PYG{n}{tb}\PYG{o}{.}\PYG{n}{run}\PYG{p}{(}
    \PYG{c+c1}{\PYGZsh{} By default, all output is written into files and not stdout,}
    \PYG{c+c1}{\PYGZsh{} but data can also be printed to stdout by setting}
    \PYG{n}{std\PYGZus{}banner}\PYG{o}{=}\PYG{k+kc}{True}\PYG{p}{,}
\PYG{p}{)}

\PYG{c+c1}{\PYGZsh{} Once the calculation is finished, your python code will continue}
\PYG{c+c1}{\PYGZsh{} executing, and you can extract data from the ThunderBoltz object.}

\PYG{c+c1}{\PYGZsh{} This will extract step by step data for reaction counts/rates,}
\PYG{c+c1}{\PYGZsh{} electron mobility, drift velocity, and more.}
\PYG{n}{ts} \PYG{o}{=} \PYG{n}{tb}\PYG{o}{.}\PYG{n}{get\PYGZus{}timeseries}\PYG{p}{(}\PYG{p}{)}
\PYG{c+c1}{\PYGZsh{} This will extract the same quantities but time averaged during the}
\PYG{c+c1}{\PYGZsh{} steady state period of the run.}
\PYG{n}{steady\PYGZus{}state} \PYG{o}{=} \PYG{n}{tb}\PYG{o}{.}\PYG{n}{get\PYGZus{}ss\PYGZus{}params}\PYG{p}{(}\PYG{p}{)}

\PYG{c+c1}{\PYGZsh{} Sometimes it is easier to extract data once it is already been}
\PYG{c+c1}{\PYGZsh{} calculated. The next section will demonstrate how to asynchronously}
\PYG{c+c1}{\PYGZsh{} extract data after the calculation has finished in a new process.}
\end{sphinxVerbatim}

\sphinxAtStartPar
This script can be run from the command line by simply executing

\begin{sphinxVerbatim}[commandchars=\\\{\}]
python\PYG{+w}{ }run\PYGZus{}example.py
\end{sphinxVerbatim}

\begin{sphinxadmonition}{warning}{Warning:}
\sphinxAtStartPar
When running and rerunning calculations, ensure that the specified simulation
directory has no output files already present. ThunderBoltz will not
overwrite these files when running more calculations. This will preserve
your data, but prevent the python interface from being able to interpret
the results.
\end{sphinxadmonition}

\sphinxAtStartPar
For a full list available ThunderBoltz parameters, see {\hyperref[\detokenize{params::doc}]{\sphinxcrossref{\DUrole{doc}{Simulation Parameters}}}}.


\subsection{Interpreting the Results}
\label{\detokenize{quickstart:interpreting-the-results}}
\sphinxAtStartPar
Some calculations of interest may take several hours, and so it is
beneficial to run it once and explore the data later. It is easy
to recover the output data from the output files after the calculation
is finished like so:

\begin{sphinxVerbatim}[commandchars=\\\{\}]
\PYG{c+c1}{\PYGZsh{} Import some python plotting tools}
\PYG{k+kn}{import} \PYG{n+nn}{matplotlib}\PYG{n+nn}{.}\PYG{n+nn}{pyplot} \PYG{k}{as} \PYG{n+nn}{plt}

\PYG{c+c1}{\PYGZsh{} This will read single calculations and return ThunderBoltz objects.}
\PYG{k+kn}{from} \PYG{n+nn}{pytb} \PYG{k+kn}{import} \PYG{n}{read}

\PYG{c+c1}{\PYGZsh{} Ensure you are running this code from the same place as above, and}
\PYG{c+c1}{\PYGZsh{} just pass the location of the simulation directory.}
\PYG{n}{tb} \PYG{o}{=} \PYG{n}{read}\PYG{p}{(}\PYG{l+s+s2}{\PYGZdq{}}\PYG{l+s+s2}{example\PYGZus{}sim}\PYG{l+s+s2}{\PYGZdq{}}\PYG{p}{)}

\PYG{c+c1}{\PYGZsh{} Now all the same data will be available in the form of pandas DataFrames.}
\PYG{n}{timeseries} \PYG{o}{=} \PYG{n}{tb}\PYG{o}{.}\PYG{n}{get\PYGZus{}timeseries}\PYG{p}{(}\PYG{p}{)} \PYG{c+c1}{\PYGZsh{} Returns timeseries data in a DataFrame.}
\PYG{n}{steady\PYGZus{}state} \PYG{o}{=} \PYG{n}{tb}\PYG{o}{.}\PYG{n}{get\PYGZus{}ss\PYGZus{}params}\PYG{p}{(}\PYG{p}{)} \PYG{c+c1}{\PYGZsh{} Returns steady state data in a DataFrame.}
\PYG{n}{velocity\PYGZus{}data} \PYG{o}{=} \PYG{n}{tb}\PYG{o}{.}\PYG{n}{get\PYGZus{}vdfs}\PYG{p}{(}\PYG{l+s+s2}{\PYGZdq{}}\PYG{l+s+s2}{all}\PYG{l+s+s2}{\PYGZdq{}}\PYG{p}{)} \PYG{c+c1}{\PYGZsh{} Returns all velocity dump data in a DataFrame.}

\PYG{c+c1}{\PYGZsh{} These frames are convenient because they can be easily manipulated and}
\PYG{c+c1}{\PYGZsh{} exported}

\PYG{c+c1}{\PYGZsh{} To export to csv:}
\PYG{n}{timeseries}\PYG{o}{.}\PYG{n}{to\PYGZus{}csv}\PYG{p}{(}\PYG{l+s+s2}{\PYGZdq{}}\PYG{l+s+s2}{example\PYGZus{}sim/timeseries.csv}\PYG{l+s+s2}{\PYGZdq{}}\PYG{p}{,} \PYG{n}{index}\PYG{o}{=}\PYG{k+kc}{False}\PYG{p}{)}
\PYG{n}{steady\PYGZus{}state}\PYG{o}{.}\PYG{n}{to\PYGZus{}csv}\PYG{p}{(}\PYG{l+s+s2}{\PYGZdq{}}\PYG{l+s+s2}{example\PYGZus{}sim/steady.csv}\PYG{l+s+s2}{\PYGZdq{}}\PYG{p}{,} \PYG{n}{index}\PYG{o}{=}\PYG{k+kc}{False}\PYG{p}{)}

\PYG{c+c1}{\PYGZsh{} One can truncate the data row\PYGZhy{}wise. For example,}
\PYG{c+c1}{\PYGZsh{} the following will take data from last 20000 steps}
\PYG{c+c1}{\PYGZsh{} of the 30000 steps.}
\PYG{n}{trunc} \PYG{o}{=} \PYG{n}{timeseries}\PYG{p}{[}\PYG{n}{timeseries}\PYG{o}{.}\PYG{n}{step} \PYG{o}{\PYGZgt{}}\PYG{o}{=} \PYG{l+m+mi}{10000}\PYG{p}{]}\PYG{o}{.}\PYG{n}{copy}\PYG{p}{(}\PYG{p}{)}

\PYG{c+c1}{\PYGZsh{} Or only look at certain columns}

\PYG{c+c1}{\PYGZsh{} This will extract the mean electron energy (MEe),}
\PYG{c+c1}{\PYGZsh{} the reduced electron mobility (mobN),}
\PYG{c+c1}{\PYGZsh{} and the Townshend ionization coefficient (a\PYGZus{}n).}
\PYG{n}{transport\PYGZus{}params} \PYG{o}{=} \PYG{n}{timeseries}\PYG{p}{[}\PYG{p}{[}\PYG{l+s+s2}{\PYGZdq{}}\PYG{l+s+s2}{MEe}\PYG{l+s+s2}{\PYGZdq{}}\PYG{p}{,} \PYG{l+s+s2}{\PYGZdq{}}\PYG{l+s+s2}{mobN}\PYG{l+s+s2}{\PYGZdq{}}\PYG{p}{,} \PYG{l+s+s2}{\PYGZdq{}}\PYG{l+s+s2}{a\PYGZus{}n}\PYG{l+s+s2}{\PYGZdq{}}\PYG{p}{]}\PYG{p}{]}\PYG{o}{.}\PYG{n}{copy}\PYG{p}{(}\PYG{p}{)}


\PYG{c+c1}{\PYGZsh{} There are also some built\PYGZhy{}in plotting methods that}
\PYG{c+c1}{\PYGZsh{} can be accessed through the ThunderBoltz object.}

\PYG{c+c1}{\PYGZsh{} This will plot step by step data for any of the output}
\PYG{c+c1}{\PYGZsh{} parameters available in the time series table. Default}
\PYG{c+c1}{\PYGZsh{} is mean energy, mobility, and Townshend ionization coefficient}
\PYG{n}{tb}\PYG{o}{.}\PYG{n}{plot\PYGZus{}timeseries}\PYG{p}{(}\PYG{p}{)}

\PYG{c+c1}{\PYGZsh{} This will plot the rate coefficients for every process.}
\PYG{n}{tb}\PYG{o}{.}\PYG{n}{plot\PYGZus{}rates}\PYG{p}{(}\PYG{p}{)}

\PYG{c+c1}{\PYGZsh{} This will plot a joint plot of the electron velocity distribution function}
\PYG{n}{tb}\PYG{o}{.}\PYG{n}{plot\PYGZus{}vdfs}\PYG{p}{(}\PYG{p}{)}

\PYG{c+c1}{\PYGZsh{} This will show a GUI and is required to actually display the plots.}
\PYG{n}{plt}\PYG{o}{.}\PYG{n}{show}\PYG{p}{(}\PYG{p}{)}
\end{sphinxVerbatim}

\sphinxAtStartPar
For more details on the output parameter format, see
\DUrole{xref,std,std-ref}{output\_params}.

\sphinxstepscope


\section{Preparing Cross Sections}
\label{\detokenize{cs_guide:preparing-cross-sections}}\label{\detokenize{cs_guide::doc}}
\sphinxAtStartPar
There are a variety of ways to specify cross sections with the
ThunderBoltz interface. In the {\hyperref[\detokenize{quickstart::doc}]{\sphinxcrossref{\DUrole{doc}{Quick Start Guide}}}},
we used a built\sphinxhyphen{}in Helium cross section model. A more general approach
to preparing cross sections is with the {\hyperref[\detokenize{api/pytb.CrossSections:pytb.CrossSections}]{\sphinxcrossref{\sphinxcode{\sphinxupquote{CrossSections}}}}} object.


\subsection{Initializing the \sphinxstyleliteralintitle{\sphinxupquote{CrossSections}} Object}
\label{\detokenize{cs_guide:initializing-the-crosssections-object}}
\sphinxAtStartPar
There are three main ways to initialize a {\hyperref[\detokenize{api/pytb.CrossSections:pytb.CrossSections}]{\sphinxcrossref{\sphinxcode{\sphinxupquote{CrossSections}}}}} object:
\begin{enumerate}
\sphinxsetlistlabels{\arabic}{enumi}{enumii}{}{.}%
\item {} 
\sphinxAtStartPar
With cross section data from another ThunderBoltz run

\begin{sphinxVerbatim}[commandchars=\\\{\}]
\PYG{k+kn}{from} \PYG{n+nn}{pytb} \PYG{k+kn}{import} \PYG{n}{CrossSections}
\PYG{c+c1}{\PYGZsh{} Just specify the path to the simulation directory of a}
\PYG{c+c1}{\PYGZsh{} different ThunderBoltz run.}
\PYG{n}{cross\PYGZus{}sections} \PYG{o}{=} \PYG{n}{CrossSections}\PYG{p}{(}\PYG{n}{input\PYGZus{}path}\PYG{o}{=}\PYG{l+s+s2}{\PYGZdq{}}\PYG{l+s+s2}{path/to/thunderboltz\PYGZus{}sim\PYGZus{}dir}\PYG{l+s+s2}{\PYGZdq{}}\PYG{p}{)}
\end{sphinxVerbatim}

\end{enumerate}
\begin{quote}

\sphinxAtStartPar
Refer to the {\hyperref[\detokenize{api/pytb.CrossSections:pytb.CrossSections}]{\sphinxcrossref{\sphinxcode{\sphinxupquote{CrossSections}}}}} section of the {\hyperref[\detokenize{ref::doc}]{\sphinxcrossref{\DUrole{doc}{API Reference}}}} to ensure the
simulation data is set up correctly for interpretation by {\hyperref[\detokenize{api/pytb.CrossSections:pytb.CrossSections}]{\sphinxcrossref{\sphinxcode{\sphinxupquote{CrossSections}}}}}.
\end{quote}
\begin{enumerate}
\sphinxsetlistlabels{\arabic}{enumi}{enumii}{}{.}%
\item {} 
\sphinxAtStartPar
By reading from an \sphinxhref{https://nl.lxcat.net}{LXCat} text file extract.

\begin{sphinxVerbatim}[commandchars=\\\{\}]
\PYG{k+kn}{from} \PYG{n+nn}{pytb} \PYG{k+kn}{import} \PYG{n}{CrossSections}
\PYG{c+c1}{\PYGZsh{} First initialize an empty cross sections object}
\PYG{n}{cross\PYGZus{}sections} \PYG{o}{=} \PYG{n}{CrossSections}\PYG{p}{(}\PYG{p}{)}
\PYG{c+c1}{\PYGZsh{} Then reference a text file extract from LXCat}
\PYG{n}{cross\PYGZus{}sections}\PYG{o}{.}\PYG{n}{from\PYGZus{}LXCat}\PYG{p}{(}\PYG{l+s+s2}{\PYGZdq{}}\PYG{l+s+s2}{path/to/LXCat\PYGZus{}data.txt}\PYG{l+s+s2}{\PYGZdq{}}\PYG{p}{)}
\end{sphinxVerbatim}

\begin{sphinxadmonition}{note}{Note:}
\sphinxAtStartPar
For now, the LXCat parser assumes two species electron\sphinxhyphen{}gas
systems where all processes are between electrons and
gas macroparticles. If you wish to use LXCat data for other
purposes, you can alter the species indices to your liking
via \sphinxcode{\sphinxupquote{CrossSections.table}} after loading in LXCat data.
\end{sphinxadmonition}

\item {} 
\sphinxAtStartPar
By programmatically generating cross section data in python.

\sphinxAtStartPar
This approach involves the {\hyperref[\detokenize{api/pytb.Process:pytb.Process}]{\sphinxcrossref{\sphinxcode{\sphinxupquote{Process}}}}} object.

\begin{sphinxVerbatim}[commandchars=\\\{\}]
\PYG{k+kn}{from} \PYG{n+nn}{pytb} \PYG{k+kn}{import} \PYG{n}{CrossSections}
\PYG{k+kn}{from} \PYG{n+nn}{pytb} \PYG{k+kn}{import} \PYG{n}{Process}

\PYG{c+c1}{\PYGZsh{} Initialize an empty cross sections object}
\PYG{n}{cross\PYGZus{}sections} \PYG{o}{=} \PYG{n}{CrossSection}\PYG{p}{(}\PYG{p}{)}

\PYG{c+c1}{\PYGZsh{} Next make a few processes}

\PYG{c+c1}{\PYGZsh{} You can pass arbitrary tabulated data like so}
\PYG{n}{elastic\PYGZus{}data} \PYG{o}{=} \PYG{p}{[}
  \PYG{c+c1}{\PYGZsh{} [eV], [m\PYGZca{}2]}
    \PYG{p}{[}\PYG{l+m+mf}{0.0}\PYG{p}{,} \PYG{l+m+mf}{2e\PYGZhy{}20}\PYG{p}{]}\PYG{p}{,}
    \PYG{p}{[}\PYG{l+m+mf}{0.001}\PYG{p}{,} \PYG{l+m+mf}{2.1e\PYGZhy{}20}\PYG{p}{]}\PYG{p}{,}
    \PYG{p}{[}\PYG{l+m+mf}{.01}\PYG{p}{,} \PYG{l+m+mf}{3e\PYGZhy{}20}\PYG{p}{]}\PYG{p}{,}
    \PYG{p}{[}\PYG{l+m+mf}{10.0}\PYG{p}{,} \PYG{l+m+mf}{1e\PYGZhy{}19}\PYG{p}{]}\PYG{p}{,}
    \PYG{p}{[}\PYG{l+m+mi}{1000}\PYG{p}{,} \PYG{l+m+mf}{1e\PYGZhy{}18}\PYG{p}{]}\PYG{p}{,}
    \PYG{p}{[}\PYG{l+m+mi}{10000}\PYG{p}{,} \PYG{l+m+mf}{2e\PYGZhy{}19}\PYG{p}{]}\PYG{p}{,}
\PYG{p}{]}
\PYG{n}{elastic\PYGZus{}process} \PYG{o}{=} \PYG{n}{Process}\PYG{p}{(}
    \PYG{l+s+s2}{\PYGZdq{}}\PYG{l+s+s2}{Elastic}\PYG{l+s+s2}{\PYGZdq{}}\PYG{p}{,} \PYG{c+c1}{\PYGZsh{} The type of process}
    \PYG{n}{r1}\PYG{o}{=}\PYG{l+m+mi}{0}\PYG{p}{,} \PYG{c+c1}{\PYGZsh{} The first reactant species index}
    \PYG{n}{r2}\PYG{o}{=}\PYG{l+m+mi}{1}\PYG{p}{,} \PYG{c+c1}{\PYGZsh{} The second reactant species index}
    \PYG{n}{p1}\PYG{o}{=}\PYG{l+m+mi}{0}\PYG{p}{,} \PYG{c+c1}{\PYGZsh{} The first product species index}
    \PYG{n}{p2}\PYG{o}{=}\PYG{l+m+mi}{1}\PYG{p}{,} \PYG{c+c1}{\PYGZsh{} The second product species index}
    \PYG{n}{cs\PYGZus{}data}\PYG{o}{=}\PYG{n}{elastic\PYGZus{}data}\PYG{p}{,}
    \PYG{c+c1}{\PYGZsh{} This will determine the name of the}
    \PYG{c+c1}{\PYGZsh{} written cross section file and ideally should}
    \PYG{c+c1}{\PYGZsh{} be unique.}
    \PYG{n}{name}\PYG{o}{=}\PYG{l+s+s2}{\PYGZdq{}}\PYG{l+s+s2}{elastic\PYGZus{}example}\PYG{l+s+s2}{\PYGZdq{}}\PYG{p}{,}
\PYG{p}{)}
\PYG{c+c1}{\PYGZsh{} You can also pass data frames, or ndarrays if that is}
\PYG{c+c1}{\PYGZsh{} preferable}

\PYG{c+c1}{\PYGZsh{} Or, use an analytic form defined with a python}
\PYG{c+c1}{\PYGZsh{} function.}
\PYG{k+kn}{import} \PYG{n+nn}{numpy} \PYG{k}{as} \PYG{n+nn}{np} \PYG{c+c1}{\PYGZsh{} Import math functionality}
\PYG{k}{def} \PYG{n+nf}{inelastic\PYGZus{}model}\PYG{p}{(}\PYG{n}{energy}\PYG{p}{,} \PYG{n}{parameter}\PYG{p}{)}\PYG{p}{:}
    \PYG{c+c1}{\PYGZsh{} It\PYGZsq{}s okay to have conditional statements}
    \PYG{k}{if} \PYG{n}{energy} \PYG{o}{\PYGZlt{}} \PYG{l+m+mi}{5}\PYG{p}{:}
        \PYG{k}{return} \PYG{n}{parameter}

    \PYG{c+c1}{\PYGZsh{} And nonlinear functions}
    \PYG{k}{return} \PYG{n}{parameter}\PYG{o}{*}\PYG{n}{np}\PYG{o}{.}\PYG{n}{log}\PYG{p}{(}\PYG{n}{energy}\PYG{p}{)}\PYG{o}{/}\PYG{n}{energy}

\PYG{c+c1}{\PYGZsh{} You can parameterize your model}
\PYG{n}{cs\PYGZus{}mod\PYGZus{}1} \PYG{o}{=} \PYG{k}{lambda} \PYG{n}{e}\PYG{p}{:} \PYG{n}{inelastic\PYGZus{}model}\PYG{p}{(}\PYG{n}{e}\PYG{p}{,} \PYG{l+m+mf}{1e\PYGZhy{}20}\PYG{p}{)}
\PYG{n}{cs\PYGZus{}mod\PYGZus{}2} \PYG{o}{=} \PYG{k}{lambda} \PYG{n}{e}\PYG{p}{:} \PYG{n}{inelastic\PYGZus{}model}\PYG{p}{(}\PYG{n}{e}\PYG{p}{,} \PYG{l+m+mf}{2e\PYGZhy{}20}\PYG{p}{)}
\PYG{n}{cs\PYGZus{}mod\PYGZus{}3} \PYG{o}{=} \PYG{k}{lambda} \PYG{n}{e}\PYG{p}{:} \PYG{n}{inelastic\PYGZus{}model}\PYG{p}{(}\PYG{n}{e}\PYG{p}{,} \PYG{l+m+mf}{3e\PYGZhy{}20}\PYG{p}{)}

\PYG{c+c1}{\PYGZsh{} And create multiple cross sections}
\PYG{n}{inelastic\PYGZus{}1} \PYG{o}{=} \PYG{n}{Process}\PYG{p}{(}
    \PYG{l+s+s2}{\PYGZdq{}}\PYG{l+s+s2}{Inelastic}\PYG{l+s+s2}{\PYGZdq{}}\PYG{p}{,} \PYG{n}{threshold}\PYG{o}{=}\PYG{l+m+mf}{1.}\PYG{p}{,} \PYG{n}{cs\PYGZus{}func}\PYG{o}{=}\PYG{n}{cs\PYGZus{}mod\PYGZus{}1}\PYG{p}{,} \PYG{n}{name}\PYG{o}{=}\PYG{l+s+s2}{\PYGZdq{}}\PYG{l+s+s2}{inelastic1}\PYG{l+s+s2}{\PYGZdq{}}\PYG{p}{)}
\PYG{n}{inelastic\PYGZus{}2} \PYG{o}{=} \PYG{n}{Process}\PYG{p}{(}
    \PYG{l+s+s2}{\PYGZdq{}}\PYG{l+s+s2}{Inelastic}\PYG{l+s+s2}{\PYGZdq{}}\PYG{p}{,} \PYG{n}{threshold}\PYG{o}{=}\PYG{l+m+mf}{1.}\PYG{p}{,} \PYG{n}{cs\PYGZus{}func}\PYG{o}{=}\PYG{n}{cs\PYGZus{}mod\PYGZus{}2}\PYG{p}{,} \PYG{n}{name}\PYG{o}{=}\PYG{l+s+s2}{\PYGZdq{}}\PYG{l+s+s2}{inelastic2}\PYG{l+s+s2}{\PYGZdq{}}\PYG{p}{)}
\PYG{n}{inelastic\PYGZus{}3} \PYG{o}{=} \PYG{n}{Process}\PYG{p}{(}
    \PYG{l+s+s2}{\PYGZdq{}}\PYG{l+s+s2}{Inelastic}\PYG{l+s+s2}{\PYGZdq{}}\PYG{p}{,} \PYG{n}{threshold}\PYG{o}{=}\PYG{l+m+mf}{1.}\PYG{p}{,} \PYG{n}{cs\PYGZus{}func}\PYG{o}{=}\PYG{n}{cs\PYGZus{}mod\PYGZus{}3}\PYG{p}{,} \PYG{n}{name}\PYG{o}{=}\PYG{l+s+s2}{\PYGZdq{}}\PYG{l+s+s2}{inelastic3}\PYG{l+s+s2}{\PYGZdq{}}\PYG{p}{)}

\PYG{c+c1}{\PYGZsh{} Finally, you can create processes with differential cross section}
\PYG{c+c1}{\PYGZsh{} models, if they are available in your ThunderBoltz version.}
\PYG{n}{ionization} \PYG{o}{=} \PYG{n}{Process}\PYG{p}{(}\PYG{l+s+s2}{\PYGZdq{}}\PYG{l+s+s2}{Ionization}\PYG{l+s+s2}{\PYGZdq{}}\PYG{p}{,} \PYG{n}{threshold}\PYG{o}{=}\PYG{l+m+mf}{10.}\PYG{p}{,}
    \PYG{n}{cs\PYGZus{}func}\PYG{o}{=}\PYG{k}{lambda} \PYG{n}{e}\PYG{p}{:} \PYG{l+m+mf}{1e\PYGZhy{}19}\PYG{o}{*}\PYG{n}{np}\PYG{o}{.}\PYG{n}{log}\PYG{p}{(}\PYG{n}{e}\PYG{p}{)}\PYG{o}{/}\PYG{n}{e}\PYG{p}{,}
    \PYG{c+c1}{\PYGZsh{} This, for example, will add the equal energy sharing condition}
    \PYG{n}{differential\PYGZus{}process}\PYG{o}{=}\PYG{l+s+s2}{\PYGZdq{}}\PYG{l+s+s2}{equal}\PYG{l+s+s2}{\PYGZdq{}}\PYG{p}{,}
    \PYG{n}{name}\PYG{o}{=}\PYG{l+s+s2}{\PYGZdq{}}\PYG{l+s+s2}{ionization}\PYG{l+s+s2}{\PYGZdq{}}\PYG{p}{)}


\PYG{c+c1}{\PYGZsh{} You can add your process to the CrossSections object one at a time}
\PYG{n}{cross\PYGZus{}sections}\PYG{o}{.}\PYG{n}{add\PYGZus{}process}\PYG{p}{(}\PYG{n}{elastic\PYGZus{}process}\PYG{p}{)}
\PYG{c+c1}{\PYGZsh{} Or all at once}
\PYG{n}{cross\PYGZus{}sections}\PYG{o}{.}\PYG{n}{add\PYGZus{}processes}\PYG{p}{(}
    \PYG{p}{[}\PYG{n}{inelastic\PYGZus{}1}\PYG{p}{,} \PYG{n}{inelastic\PYGZus{}2}\PYG{p}{,} \PYG{n}{inelastic\PYGZus{}3}\PYG{p}{,} \PYG{n}{ionization}\PYG{p}{]}
\PYG{p}{)}
\end{sphinxVerbatim}

\begin{sphinxadmonition}{note}{Note:}
\sphinxAtStartPar
It is important to explicitly specify threshold values for
inelastic and superelastic processes because their values will
not be inferred from the cross section data.
\end{sphinxadmonition}

\end{enumerate}


\subsection{Viewing Your Cross Sections}
\label{\detokenize{cs_guide:viewing-your-cross-sections}}
\sphinxAtStartPar
When parsing data from external sources, it is important to ensure
that the correct data is being used in the intended context for the
simulation. You can view the reaction table for the model by
printing out the \sphinxcode{\sphinxupquote{table}} attribute.

\begin{sphinxVerbatim}[commandchars=\\\{\}]
\PYG{n+nb}{print}\PYG{p}{(}\PYG{n}{cross\PYGZus{}section}\PYG{o}{.}\PYG{n}{table}\PYG{p}{)}
\end{sphinxVerbatim}

\sphinxAtStartPar
And you can view the cross section data associated with each process
by printing out the \sphinxcode{\sphinxupquote{data}} attribute.

\begin{sphinxVerbatim}[commandchars=\\\{\}]
\PYG{n+nb}{print}\PYG{p}{(}\PYG{n}{cross\PYGZus{}section}\PYG{o}{.}\PYG{n}{data}\PYG{p}{)}
\end{sphinxVerbatim}

\sphinxAtStartPar
To view a plot of the cross section data, use the {\hyperref[\detokenize{api/pytb.CrossSections.plot_cs:pytb.CrossSections.plot_cs}]{\sphinxcrossref{\sphinxcode{\sphinxupquote{plot\_cs()}}}}} method.

\begin{sphinxVerbatim}[commandchars=\\\{\}]
\PYG{n}{cross\PYGZus{}section}\PYG{o}{.}\PYG{n}{plot\PYGZus{}cs}\PYG{p}{(}\PYG{p}{)}

\PYG{c+c1}{\PYGZsh{} Remember to show the plot at the end of plotting scripts}
\PYG{c+c1}{\PYGZsh{} Make sure to include the import statement \PYGZdq{}import matplotlib.pyplot as plt\PYGZdq{}}
\PYG{n}{plt}\PYG{o}{.}\PYG{n}{show}\PYG{p}{(}\PYG{p}{)}
\end{sphinxVerbatim}

\sphinxAtStartPar
See the API reference for plotting related quantities with the {\hyperref[\detokenize{api/pytb.CrossSections.plot_cs:pytb.CrossSections.plot_cs}]{\sphinxcrossref{\sphinxcode{\sphinxupquote{plot\_cs()}}}}}
method.


\subsection{Attaching the \sphinxstyleliteralintitle{\sphinxupquote{CrossSections}} Object}
\label{\detokenize{cs_guide:attaching-the-crosssections-object}}
\sphinxAtStartPar
Finally, attach the \sphinxcode{\sphinxupquote{CrossSections}} object to the main ThunderBoltz
object using the \sphinxcode{\sphinxupquote{cs}} keyword to use the cross section model within it.

\begin{sphinxVerbatim}[commandchars=\\\{\}]
\PYG{n}{tb} \PYG{o}{=} \PYG{n}{ThunderBoltz}\PYG{p}{(}
    \PYG{c+c1}{\PYGZsh{} ...}
    \PYG{n}{cs}\PYG{o}{=}\PYG{n}{cross\PYGZus{}sections}\PYG{p}{,}
    \PYG{c+c1}{\PYGZsh{} ...}
\PYG{p}{)}

\PYG{n}{tb}\PYG{o}{.}\PYG{n}{run}\PYG{p}{(}\PYG{p}{)}
\PYG{c+c1}{\PYGZsh{} ...}
\end{sphinxVerbatim}

\sphinxstepscope


\section{Running Multiple Calculations}
\label{\detokenize{multi_guide:running-multiple-calculations}}\label{\detokenize{multi_guide::doc}}

\subsection{In Sequence}
\label{\detokenize{multi_guide:in-sequence}}
\sphinxAtStartPar
You can change simulation parameters in the {\hyperref[\detokenize{api/pytb.ThunderBoltz:pytb.ThunderBoltz}]{\sphinxcrossref{\sphinxcode{\sphinxupquote{ThunderBoltz}}}}}
object and run the program again in a new directory. Use
the {\hyperref[\detokenize{api/pytb.ThunderBoltz.set_:pytb.ThunderBoltz.set_}]{\sphinxcrossref{\sphinxcode{\sphinxupquote{set\_()}}}}} method to update the desired parameters.

\sphinxAtStartPar
Suppose you wanted to run several calculation at various
field values. To do this, loop through the field values,
create new directories for the new calculation and run the
object like so:

\begin{sphinxVerbatim}[commandchars=\\\{\}]
\PYG{k+kn}{import} \PYG{n+nn}{os}
\PYG{k+kn}{import} \PYG{n+nn}{pytb}

\PYG{c+c1}{\PYGZsh{} Make a base directory for this ensemble of simulations}
\PYG{n}{os}\PYG{o}{.}\PYG{n}{makedirs}\PYG{p}{(}\PYG{l+s+s2}{\PYGZdq{}}\PYG{l+s+s2}{multi\PYGZus{}sim}\PYG{l+s+s2}{\PYGZdq{}}\PYG{p}{)}

\PYG{n}{tb} \PYG{o}{=} \PYG{n}{ThunderBoltz}\PYG{p}{(}\PYG{n}{indeck}\PYG{o}{=}\PYG{n}{pytb}\PYG{o}{.}\PYG{n}{input}\PYG{o}{.}\PYG{n}{He\PYGZus{}TB}\PYG{p}{)}

\PYG{n}{fields} \PYG{o}{=} \PYG{p}{[}\PYG{l+m+mi}{10}\PYG{p}{,} \PYG{l+m+mi}{100}\PYG{p}{,} \PYG{l+m+mi}{500}\PYG{p}{]}
\PYG{c+c1}{\PYGZsh{} Loop through the field values}
\PYG{k}{for} \PYG{n}{field} \PYG{o+ow}{in} \PYG{n}{fields}\PYG{p}{:}
    \PYG{c+c1}{\PYGZsh{} Create a new directory for this calculation}
    \PYG{n}{subdir} \PYG{o}{=} \PYG{n}{os}\PYG{o}{.}\PYG{n}{path}\PYG{o}{.}\PYG{n}{join}\PYG{p}{(}\PYG{l+s+s2}{\PYGZdq{}}\PYG{l+s+s2}{multi\PYGZus{}sim}\PYG{l+s+s2}{\PYGZdq{}}\PYG{p}{,} \PYG{l+s+sa}{f}\PYG{l+s+s2}{\PYGZdq{}}\PYG{l+s+si}{\PYGZob{}}\PYG{n}{field}\PYG{l+s+si}{\PYGZcb{}}\PYG{l+s+s2}{Td}\PYG{l+s+s2}{\PYGZdq{}}\PYG{p}{)}
    \PYG{n}{os}\PYG{o}{.}\PYG{n}{makedirs}\PYG{p}{(}\PYG{n}{subdir}\PYG{p}{)}
    \PYG{n}{tb}\PYG{o}{.}\PYG{n}{set\PYGZus{}}\PYG{p}{(}\PYG{n}{Ered}\PYG{o}{=}\PYG{n}{field}\PYG{p}{,} \PYG{n}{directory}\PYG{o}{=}\PYG{n}{subdir}\PYG{p}{)}
    \PYG{c+c1}{\PYGZsh{} Run the calculation}
    \PYG{n}{tb}\PYG{o}{.}\PYG{n}{run}\PYG{p}{(}\PYG{p}{)}
\end{sphinxVerbatim}

\sphinxAtStartPar
Each call to {\hyperref[\detokenize{api/pytb.ThunderBoltz.run:pytb.ThunderBoltz.run}]{\sphinxcrossref{\sphinxcode{\sphinxupquote{run()}}}}} will block until the
corresponding simulation is finished.


\subsection{In Parallel}
\label{\detokenize{multi_guide:in-parallel}}
\sphinxAtStartPar
Now suppose you would like to take advantage of multiple cores to run
several ThunderBoltz calculations at once. Though the internal kinetic
code is not (yet) parallelized, the python interface can run several
ThunderBoltz subprocesses in parallel like so:

\begin{sphinxVerbatim}[commandchars=\\\{\}]
\PYG{k+kn}{import} \PYG{n+nn}{os}
\PYG{k+kn}{import} \PYG{n+nn}{pytb}

\PYG{c+c1}{\PYGZsh{} Make a base directory for this ensemble of simulations}
\PYG{n}{base\PYGZus{}path} \PYG{o}{=} \PYG{l+s+s2}{\PYGZdq{}}\PYG{l+s+s2}{multi\PYGZus{}sim\PYGZus{}parallel}\PYG{l+s+s2}{\PYGZdq{}}
\PYG{n}{os}\PYG{o}{.}\PYG{n}{makedirs}\PYG{p}{(}\PYG{n}{base\PYGZus{}path}\PYG{p}{)}

\PYG{c+c1}{\PYGZsh{} Create the base object for the calculation}
\PYG{n}{tb} \PYG{o}{=} \PYG{n}{ThunderBoltz}\PYG{p}{(}\PYG{n}{indeck}\PYG{o}{=}\PYG{n}{pytb}\PYG{o}{.}\PYG{n}{input}\PYG{o}{.}\PYG{n}{He\PYGZus{}TB}\PYG{p}{)}

\PYG{n}{fields} \PYG{o}{=} \PYG{p}{[}\PYG{l+m+mi}{10}\PYG{p}{,} \PYG{l+m+mi}{100}\PYG{p}{,} \PYG{l+m+mi}{500}\PYG{p}{]}

\PYG{c+c1}{\PYGZsh{} This time use the DistributedPool context,}
\PYG{c+c1}{\PYGZsh{} passing the ThunderBoltz object like so}
\PYG{k}{with} \PYG{n}{DistributedPool}\PYG{p}{(}\PYG{n}{tb}\PYG{p}{)} \PYG{k}{as} \PYG{n}{pool}\PYG{p}{:}
    \PYG{c+c1}{\PYGZsh{} Loop through the field values}
    \PYG{k}{for} \PYG{n}{field} \PYG{o+ow}{in} \PYG{n}{fields}\PYG{p}{:}
        \PYG{c+c1}{\PYGZsh{} Create a new directory for this calculation}
        \PYG{n}{subdir} \PYG{o}{=} \PYG{n}{os}\PYG{o}{.}\PYG{n}{path}\PYG{o}{.}\PYG{n}{join}\PYG{p}{(}\PYG{n}{base\PYGZus{}path}\PYG{p}{,} \PYG{l+s+sa}{f}\PYG{l+s+s2}{\PYGZdq{}}\PYG{l+s+si}{\PYGZob{}}\PYG{n}{field}\PYG{l+s+si}{\PYGZcb{}}\PYG{l+s+s2}{Td}\PYG{l+s+s2}{\PYGZdq{}}\PYG{p}{)}
        \PYG{n}{os}\PYG{o}{.}\PYG{n}{makedirs}\PYG{p}{(}\PYG{n}{subdir}\PYG{p}{)}

        \PYG{c+c1}{\PYGZsh{} Rather than running with the ``ThunderBoltz`` object,}
        \PYG{c+c1}{\PYGZsh{} submit the changes to the pool, and it will automatically}
        \PYG{c+c1}{\PYGZsh{} run each each submitted calculation in parallel.}
        \PYG{n}{pool}\PYG{o}{.}\PYG{n}{submit}\PYG{p}{(}\PYG{n}{Ered}\PYG{o}{=}\PYG{n}{field}\PYG{p}{,} \PYG{n}{directory}\PYG{o}{=}\PYG{n}{subdir}\PYG{p}{)}

\PYG{c+c1}{\PYGZsh{} The DistributedPool context will wait for all the jobs to finish}
\PYG{c+c1}{\PYGZsh{} before continuing execution outside the \PYGZsq{}with\PYGZsq{} block.}
\end{sphinxVerbatim}

\begin{sphinxadmonition}{warning}{Warning:}
\sphinxAtStartPar
The forking process used to run multiple simulations has thusfar
only been tested on UNIX/LINUX operating systems.
\end{sphinxadmonition}

\begin{sphinxadmonition}{warning}{Warning:}
\sphinxAtStartPar
Ensure there is enough simultaneous memory for all jobs when running
them in parallel. See the section on
\DUrole{xref,std,std-ref}{Electron Growth and Memory Management}.
\end{sphinxadmonition}


\subsection{With a Job Manager}
\label{\detokenize{multi_guide:with-a-job-manager}}
\sphinxAtStartPar
If HPC resources are available to the user, the python API
includes a job manager compatible with the
\sphinxhref{https://slurm.schedmd.com/documentation.html}{SLURM} protocol.
The {\hyperref[\detokenize{api/pytb.parallel.SlurmManager:pytb.parallel.SlurmManager}]{\sphinxcrossref{\sphinxcode{\sphinxupquote{SlurmManager}}}}} context allows for many different calculations
to be split up among compute nodes, and further distributed across
cores. Use it as follows:

\begin{sphinxVerbatim}[commandchars=\\\{\}]
\PYG{k+kn}{import} \PYG{n+nn}{os}
\PYG{k+kn}{import} \PYG{n+nn}{pytb}

\PYG{c+c1}{\PYGZsh{} Make a base directory for this ensemble of simulations}
\PYG{n}{base\PYGZus{}path} \PYG{o}{=} \PYG{l+s+s2}{\PYGZdq{}}\PYG{l+s+s2}{multi\PYGZus{}sim\PYGZus{}slurm}\PYG{l+s+s2}{\PYGZdq{}}
\PYG{n}{os}\PYG{o}{.}\PYG{n}{makedirs}\PYG{p}{(}\PYG{n}{base\PYGZus{}path}\PYG{p}{)}

\PYG{c+c1}{\PYGZsh{} Create the base object for the calculation}
\PYG{n}{tb} \PYG{o}{=} \PYG{n}{ThunderBoltz}\PYG{p}{(}\PYG{n}{indeck}\PYG{o}{=}\PYG{n}{pytb}\PYG{o}{.}\PYG{n}{input}\PYG{o}{.}\PYG{n}{He\PYGZus{}TB}\PYG{p}{)}

\PYG{n}{fields} \PYG{o}{=} \PYG{p}{[}\PYG{l+m+mi}{10}\PYG{p}{,} \PYG{l+m+mi}{100}\PYG{p}{,} \PYG{l+m+mi}{500}\PYG{p}{]}

\PYG{c+c1}{\PYGZsh{} Configure SLURM parameters for your job}
\PYG{n}{slurm\PYGZus{}options} \PYG{o}{=} \PYG{p}{\PYGZob{}}
    \PYG{l+s+s2}{\PYGZdq{}}\PYG{l+s+s2}{account}\PYG{l+s+s2}{\PYGZdq{}}\PYG{p}{:} \PYG{l+s+s2}{\PYGZdq{}}\PYG{l+s+s2}{my\PYGZus{}account}\PYG{l+s+s2}{\PYGZdq{}}\PYG{p}{,}
    \PYG{l+s+s2}{\PYGZdq{}}\PYG{l+s+s2}{time}\PYG{l+s+s2}{\PYGZdq{}}\PYG{p}{:} \PYG{l+m+mi}{100}\PYG{p}{,} \PYG{c+c1}{\PYGZsh{} in minutes}
    \PYG{l+s+s2}{\PYGZdq{}}\PYG{l+s+s2}{job\PYGZhy{}name}\PYG{l+s+s2}{\PYGZdq{}}\PYG{p}{:} \PYG{l+s+s2}{\PYGZdq{}}\PYG{l+s+s2}{test\PYGZus{}slurm}\PYG{l+s+s2}{\PYGZdq{}}\PYG{p}{,}
    \PYG{l+s+s2}{\PYGZdq{}}\PYG{l+s+s2}{ntasks\PYGZhy{}per\PYGZhy{}node}\PYG{l+s+s2}{\PYGZdq{}}\PYG{p}{:} \PYG{l+m+mi}{8}\PYG{p}{,} \PYG{c+c1}{\PYGZsh{} Specify number of cores to use}
    \PYG{l+s+s2}{\PYGZdq{}}\PYG{l+s+s2}{qos}\PYG{l+s+s2}{\PYGZdq{}}\PYG{p}{:} \PYG{l+s+s2}{\PYGZdq{}}\PYG{l+s+s2}{debug}\PYG{l+s+s2}{\PYGZdq{}}\PYG{p}{,}
    \PYG{l+s+s2}{\PYGZdq{}}\PYG{l+s+s2}{reservation}\PYG{l+s+s2}{\PYGZdq{}}\PYG{p}{:} \PYG{l+s+s2}{\PYGZdq{}}\PYG{l+s+s2}{debug}\PYG{l+s+s2}{\PYGZdq{}}\PYG{p}{,}
\PYG{p}{\PYGZcb{}}

\PYG{c+c1}{\PYGZsh{} Use the SlurmManager Context, just like the DistributedPool context,}
\PYG{c+c1}{\PYGZsh{} but also give it your SLURM options.}
\PYG{k}{with} \PYG{n}{SlurmManager}\PYG{p}{(}\PYG{n}{tb}\PYG{p}{,} \PYG{n}{base\PYGZus{}path}\PYG{p}{,} \PYG{o}{*}\PYG{o}{*}\PYG{n}{slurm\PYGZus{}options}\PYG{p}{)} \PYG{k}{as} \PYG{n}{slurm}\PYG{p}{:}
    \PYG{c+c1}{\PYGZsh{} Loop through the field values}
    \PYG{k}{for} \PYG{n}{field} \PYG{o+ow}{in} \PYG{n}{fields}\PYG{p}{:}
        \PYG{c+c1}{\PYGZsh{} Create a new directory for this calculation}
        \PYG{n}{subdir} \PYG{o}{=} \PYG{n}{os}\PYG{o}{.}\PYG{n}{path}\PYG{o}{.}\PYG{n}{join}\PYG{p}{(}\PYG{n}{base\PYGZus{}path}\PYG{p}{,} \PYG{l+s+sa}{f}\PYG{l+s+s2}{\PYGZdq{}}\PYG{l+s+si}{\PYGZob{}}\PYG{n}{field}\PYG{l+s+si}{\PYGZcb{}}\PYG{l+s+s2}{Td}\PYG{l+s+s2}{\PYGZdq{}}\PYG{p}{)}
        \PYG{n}{os}\PYG{o}{.}\PYG{n}{makedirs}\PYG{p}{(}\PYG{n}{subdir}\PYG{p}{)}
        \PYG{c+c1}{\PYGZsh{} Use the slurm manager the same way as the pool, it will}
        \PYG{c+c1}{\PYGZsh{} handle node and core allocation internally.}
        \PYG{n}{slurm}\PYG{o}{.}\PYG{n}{submit}\PYG{p}{(}\PYG{n}{Ered}\PYG{o}{=}\PYG{n}{field}\PYG{p}{,} \PYG{n}{directory}\PYG{o}{=}\PYG{n}{subdir}\PYG{p}{)}
\end{sphinxVerbatim}

\sphinxAtStartPar
See \sphinxhref{https://docs.python.org/3/reference/expressions.html\#calls}{here}
for an explanation of the \sphinxcode{\sphinxupquote{**}} (unpacking) operator used
in the previous example.

\begin{sphinxadmonition}{note}{Note:}
\sphinxAtStartPar
This job manager currently only works for clusters that either
already have the gcc and python requirements installed on each
compute node, or clusters that use the
\sphinxhref{https://hpc-wiki.info/hpc/Modules}{Module System} to load
functionality.

\sphinxAtStartPar
The default behavior is to accomodate the module system as it
is common on most HPC machines. If you wish to avoid writing
\sphinxcode{\sphinxupquote{module load}} commands in the SLURM script, simply specify
\sphinxcode{\sphinxupquote{modules={[}{]}}} in the \sphinxcode{\sphinxupquote{SlurmManager}} constructor.
\end{sphinxadmonition}

\begin{sphinxadmonition}{warning}{Warning:}
\sphinxAtStartPar
Ensure there is enough memory for all parallel jobs when running
them in parallel.
\end{sphinxadmonition}

\sphinxstepscope


\section{Extracting Results}
\label{\detokenize{ext_guide:extracting-results}}\label{\detokenize{ext_guide::doc}}

\subsection{Reading a Single Calculation}
\label{\detokenize{ext_guide:reading-a-single-calculation}}
\sphinxAtStartPar
After a calculation is finished, you can easily read the output
data using the python API like so:

\begin{sphinxVerbatim}[commandchars=\\\{\}]
\PYG{k+kn}{import} \PYG{n+nn}{pytb}

\PYG{c+c1}{\PYGZsh{} Pass the location of the simulation directory to be read}
\PYG{n}{tb} \PYG{o}{=} \PYG{n}{pytb}\PYG{o}{.}\PYG{n}{read}\PYG{p}{(}\PYG{l+s+s2}{\PYGZdq{}}\PYG{l+s+s2}{path/to/previous/simulation\PYGZus{}directory}\PYG{l+s+s2}{\PYGZdq{}}\PYG{p}{)}
\end{sphinxVerbatim}

\sphinxAtStartPar
A single {\hyperref[\detokenize{api/pytb.ThunderBoltz:pytb.ThunderBoltz}]{\sphinxcrossref{\sphinxcode{\sphinxupquote{ThunderBoltz}}}}} object will be returned, with which
you can easily {\hyperref[\detokenize{ext_guide:exporting-data}]{\sphinxsamedocref{export}}} or {\hyperref[\detokenize{ext_guide:plotting-results}]{\sphinxsamedocref{plot}}}
output data.

\sphinxAtStartPar
When reading calculations in this way, you may or may not want to extract
cross section data from the simulation directory as well. To save on
runtime, cross section data is not read in by default. However, if you wanted
to read in cross section data from an old calculation and reuse that
data for other purposes, you can use the \sphinxcode{\sphinxupquote{read\_cs\_data}} argument:

\begin{sphinxVerbatim}[commandchars=\\\{\}]
\PYG{k+kn}{import} \PYG{n+nn}{pytb}

\PYG{n}{tb} \PYG{o}{=} \PYG{n}{pytb}\PYG{o}{.}\PYG{n}{read}\PYG{p}{(}\PYG{l+s+s2}{\PYGZdq{}}\PYG{l+s+s2}{path/to/previous/simulation\PYGZus{}directory}\PYG{l+s+s2}{\PYGZdq{}}\PYG{p}{,} \PYG{n}{read\PYGZus{}cs\PYGZus{}data}\PYG{o}{=}\PYG{k+kc}{True}\PYG{p}{)}

\PYG{c+c1}{\PYGZsh{} You will now see the cross section data has been loaded}
\PYG{n+nb}{print}\PYG{p}{(}\PYG{n}{tb}\PYG{o}{.}\PYG{n}{cs}\PYG{o}{.}\PYG{n}{data}\PYG{p}{)}
\end{sphinxVerbatim}


\subsection{Reading Many Calculations}
\label{\detokenize{ext_guide:reading-many-calculations}}
\sphinxAtStartPar
When running many calculations in various directories, it can be convenient to
read all of the output data at once. Imagine a directory structure like this:

\begin{sphinxVerbatim}[commandchars=\\\{\}]
path/to/base\PYGZus{}path
 /———sim1
     /———indeck\PYGZus{}file.in
     /———cross\PYGZus{}sections
     /———thunderboltz.out
     ...
 /———sim2
     ...
 /———sim3
     ...
 ...
\end{sphinxVerbatim}

\sphinxAtStartPar
where several ThunderBoltz calculations are stored in one base directory
located at \sphinxcode{\sphinxupquote{path/to/base\_path}}. You locate and extract all relevant
ThunderBoltz data out of a directory tree using the {\hyperref[\detokenize{api/pytb.tb.query_tree:pytb.tb.query_tree}]{\sphinxcrossref{\sphinxcode{\sphinxupquote{query\_tree()}}}}}
function:

\begin{sphinxVerbatim}[commandchars=\\\{\}]
\PYG{k+kn}{import} \PYG{n+nn}{pytb}

\PYG{n}{tbs} \PYG{o}{=} \PYG{n}{pytb}\PYG{o}{.}\PYG{n}{query\PYGZus{}tree}\PYG{p}{(}\PYG{l+s+s2}{\PYGZdq{}}\PYG{l+s+s2}{path/to/base\PYGZus{}path}\PYG{l+s+s2}{\PYGZdq{}}\PYG{p}{)}

\PYG{c+c1}{\PYGZsh{} Now you can access each of the simulation objects}
\PYG{c+c1}{\PYGZsh{} separately. For example:}

\PYG{c+c1}{\PYGZsh{} View the time series data from the first read calculation.}
\PYG{n+nb}{print}\PYG{p}{(}\PYG{n}{tbs}\PYG{p}{[}\PYG{l+m+mi}{0}\PYG{p}{]}\PYG{o}{.}\PYG{n}{get\PYGZus{}timeseries}\PYG{p}{)}

\PYG{c+c1}{\PYGZsh{} Plot the last velocity dump data from the fourth read calculation.}
\PYG{n}{tbs}\PYG{p}{[}\PYG{l+m+mi}{3}\PYG{p}{]}\PYG{o}{.}\PYG{n}{plot\PYGZus{}vdfs}\PYG{p}{(}\PYG{p}{)}
\end{sphinxVerbatim}

\sphinxAtStartPar
See the {\hyperref[\detokenize{api/pytb.tb.query_tree:pytb.tb.query_tree}]{\sphinxcrossref{\sphinxcode{\sphinxupquote{query\_tree()}}}}} API reference to learn about options for filtering
criteria and automatically merging data from several calculations.

\sphinxAtStartPar
As with the {\hyperref[\detokenize{ext_guide:reading-a-single-calculation}]{\sphinxsamedocref{single calculation}}} case,
you can request the cross section data by providing the \sphinxcode{\sphinxupquote{read\_cs\_data}}
argument:

\begin{sphinxVerbatim}[commandchars=\\\{\}]
\PYG{k+kn}{import} \PYG{n+nn}{pytb}

\PYG{n}{tbs} \PYG{o}{=} \PYG{n}{pytb}\PYG{o}{.}\PYG{n}{query\PYGZus{}tree}\PYG{p}{(}\PYG{l+s+s2}{\PYGZdq{}}\PYG{l+s+s2}{path/to/previous/simulation\PYGZus{}directory}\PYG{l+s+s2}{\PYGZdq{}}\PYG{p}{,} \PYG{n}{read\PYGZus{}cs\PYGZus{}data}\PYG{o}{=}\PYG{k+kc}{True}\PYG{p}{)}

\PYG{c+c1}{\PYGZsh{} Now each of the calculations will have cross section model data attached to them.}
\PYG{c+c1}{\PYGZsh{} For example, this will print the collision table for the 3rd read in calculation.}
\PYG{n+nb}{print}\PYG{p}{(}\PYG{n}{tb}\PYG{p}{[}\PYG{l+m+mi}{2}\PYG{p}{]}\PYG{o}{.}\PYG{n}{cs}\PYG{o}{.}\PYG{n}{table}\PYG{p}{)}
\end{sphinxVerbatim}


\subsection{Accessing Data}
\label{\detokenize{ext_guide:accessing-data}}
\sphinxAtStartPar
Either after a calculation has finished, or after reading output data as shown
above, all data can be extracted from the {\hyperref[\detokenize{api/pytb.ThunderBoltz:pytb.ThunderBoltz}]{\sphinxcrossref{\sphinxcode{\sphinxupquote{ThunderBoltz}}}}} object:

\sphinxAtStartPar
Time\sphinxhyphen{}dependent data for the attributes found in
{\hyperref[\detokenize{api/pytb.parameters.OutputParameters:pytb.parameters.OutputParameters}]{\sphinxcrossref{\sphinxcode{\sphinxupquote{OutputParameters}}}}} can be accessed with
{\hyperref[\detokenize{api/pytb.ThunderBoltz.get_timeseries:pytb.ThunderBoltz.get_timeseries}]{\sphinxcrossref{\sphinxcode{\sphinxupquote{get\_timeseries()}}}}}:

\begin{sphinxVerbatim}[commandchars=\\\{\}]
\PYG{n}{data} \PYG{o}{=} \PYG{n}{tb}\PYG{o}{.}\PYG{n}{get\PYGZus{}timeseries}\PYG{p}{(}\PYG{p}{)}
\end{sphinxVerbatim}

\sphinxAtStartPar
Time\sphinxhyphen{}averaged data for the attributes found in
{\hyperref[\detokenize{api/pytb.parameters.OutputParameters:pytb.parameters.OutputParameters}]{\sphinxcrossref{\sphinxcode{\sphinxupquote{OutputParameters}}}}} can be accessed with
{\hyperref[\detokenize{api/pytb.ThunderBoltz.get_ss_params:pytb.ThunderBoltz.get_ss_params}]{\sphinxcrossref{\sphinxcode{\sphinxupquote{get\_ss\_params()}}}}}:

\begin{sphinxVerbatim}[commandchars=\\\{\}]
\PYG{n}{data} \PYG{o}{=} \PYG{n}{tb}\PYG{o}{.}\PYG{n}{get\PYGZus{}ss\PYGZus{}params}\PYG{p}{(}\PYG{p}{)}
\end{sphinxVerbatim}

\sphinxAtStartPar
This method will also compute standard deviations over the
steady\sphinxhyphen{}state interval for each parameter in a new column
with a “\_std” suffix added to the column name.

\begin{sphinxadmonition}{warning}{Warning:}
\sphinxAtStartPar
Currently, the last quarter of the timeseries data is assumed to be
in steady\sphinxhyphen{}state by default when calculating these steady\sphinxhyphen{}state parameters.
Please verify that this is true by viewing the figures produced by
\sphinxcode{\sphinxupquote{plot\_timeseries()}}. Otherwise, run the simulation for longer,
or provide your own appropriate criteria via the \sphinxcode{\sphinxupquote{ss\_func}} option when
calling {\hyperref[\detokenize{api/pytb.ThunderBoltz.get_ss_params:pytb.ThunderBoltz.get_ss_params}]{\sphinxcrossref{\sphinxcode{\sphinxupquote{get\_ss\_params()}}}}}.
\end{sphinxadmonition}

\sphinxAtStartPar
Output parameters for the attributes found in
{\hyperref[\detokenize{api/pytb.parameters.ParticleParameters:pytb.parameters.ParticleParameters}]{\sphinxcrossref{\sphinxcode{\sphinxupquote{ParticleParameters}}}}} can be accessed with
{\hyperref[\detokenize{api/pytb.ThunderBoltz.get_particle_tables:pytb.ThunderBoltz.get_particle_tables}]{\sphinxcrossref{\sphinxcode{\sphinxupquote{get\_particle\_tables()}}}}}:

\begin{sphinxVerbatim}[commandchars=\\\{\}]
\PYG{n}{data} \PYG{o}{=} \PYG{n}{tb}\PYG{o}{.}\PYG{n}{get\PYGZus{}particle\PYGZus{}tables}\PYG{p}{(}\PYG{p}{)}

\PYG{c+c1}{\PYGZsh{} For example, this will write the mean energy, and}
\PYG{c+c1}{\PYGZsh{} each of the mean displacement components to a csv}
\PYG{c+c1}{\PYGZsh{} called \PYGZdq{}R\PYGZus{}export.csv\PYGZdq{}}
\PYG{n}{data}\PYG{o}{.}\PYG{n}{to\PYGZus{}csv}\PYG{p}{(}\PYG{l+s+s2}{\PYGZdq{}}\PYG{l+s+s2}{R\PYGZus{}export.csv}\PYG{l+s+s2}{\PYGZdq{}}\PYG{p}{,} \PYG{n}{index}\PYG{o}{=}\PYG{k+kc}{False}\PYG{p}{)}
\end{sphinxVerbatim}


\subsection{Exporting Data}
\label{\detokenize{ext_guide:exporting-data}}
\sphinxAtStartPar
Once data is in the form of a \sphinxhref{http://pandas.pydata.org/pandas-docs/dev/reference/api/pandas.DataFrame.html\#pandas.DataFrame}{\sphinxcode{\sphinxupquote{DataFrame}}}, it is easy
to export it to other formats. See the
\sphinxhref{https://pandas.pydata.org/docs/user\_guide/io.html}{Pandas I/O Guide}
for extensive options for converting from the \sphinxcode{\sphinxupquote{DataFrame}} object.
The simplest option is to convert the data to a csv:

\begin{sphinxVerbatim}[commandchars=\\\{\}]
\PYG{c+c1}{\PYGZsh{} This will write the data into a new file called \PYGZdq{}my\PYGZus{}new\PYGZus{}file.csv\PYGZdq{}}
\PYG{n}{data}\PYG{o}{.}\PYG{n}{to\PYGZus{}csv}\PYG{p}{(}\PYG{l+s+s2}{\PYGZdq{}}\PYG{l+s+s2}{my\PYGZus{}new\PYGZus{}file.csv}\PYG{l+s+s2}{\PYGZdq{}}\PYG{p}{,} \PYG{n}{index}\PYG{o}{=}\PYG{k+kc}{False}\PYG{p}{)}
\end{sphinxVerbatim}

\begin{sphinxadmonition}{note}{Note:}
\sphinxAtStartPar
When exporting data to the csv format from a pandas
DataFrame, it is usually most convenient to pass \sphinxcode{\sphinxupquote{index}} = \sphinxcode{\sphinxupquote{False}}
to prevent \sphinxhref{http://pandas.pydata.org/pandas-docs/dev/reference/api/pandas.DataFrame.to\_csv.html\#pandas.DataFrame.to\_csv}{\sphinxcode{\sphinxupquote{to\_csv()}}} from writing
the index (usually just an enumeration of the rows) into the
first column of the csv.
\end{sphinxadmonition}


\subsection{Plotting Results}
\label{\detokenize{ext_guide:plotting-results}}
\sphinxAtStartPar
The {\hyperref[\detokenize{api/pytb.ThunderBoltz:pytb.ThunderBoltz}]{\sphinxcrossref{\sphinxcode{\sphinxupquote{ThunderBoltz}}}}} API offers functions for
automatically plotting results. See the documentation for the following
functions


\begin{savenotes}\sphinxattablestart
\sphinxthistablewithglobalstyle
\sphinxthistablewithnovlinesstyle
\centering
\begin{tabulary}{\linewidth}[t]{\X{1}{2}\X{1}{2}}
\sphinxtoprule
\sphinxtableatstartofbodyhook
\sphinxAtStartPar
{\hyperref[\detokenize{api/pytb.ThunderBoltz.plot_timeseries:pytb.ThunderBoltz.plot_timeseries}]{\sphinxcrossref{\sphinxcode{\sphinxupquote{pytb.ThunderBoltz.plot\_timeseries}}}}}({[}series, ...{]})
&
\sphinxAtStartPar
Create a diagnostic plot of ThunderBoltz time series data.
\\
\sphinxhline
\sphinxAtStartPar
{\hyperref[\detokenize{api/pytb.ThunderBoltz.plot_rates:pytb.ThunderBoltz.plot_rates}]{\sphinxcrossref{\sphinxcode{\sphinxupquote{pytb.ThunderBoltz.plot\_rates}}}}}({[}save, stamp, ...{]})
&
\sphinxAtStartPar
Create a diagnostic plot of ThunderBoltz time series data.
\\
\sphinxhline
\sphinxAtStartPar
{\hyperref[\detokenize{api/pytb.ThunderBoltz.plot_edf_comps:pytb.ThunderBoltz.plot_edf_comps}]{\sphinxcrossref{\sphinxcode{\sphinxupquote{pytb.ThunderBoltz.plot\_edf\_comps}}}}}({[}steps, ...{]})
&
\sphinxAtStartPar
Plot the directional components of the energy distribution function.
\\
\sphinxhline
\sphinxAtStartPar
{\hyperref[\detokenize{api/pytb.ThunderBoltz.plot_edfs:pytb.ThunderBoltz.plot_edfs}]{\sphinxcrossref{\sphinxcode{\sphinxupquote{pytb.ThunderBoltz.plot\_edfs}}}}}({[}steps, ...{]})
&
\sphinxAtStartPar
Plot the electron total energy distribution function, optionally include the provided cross sections for comparison.
\\
\sphinxhline
\sphinxAtStartPar
{\hyperref[\detokenize{api/pytb.ThunderBoltz.plot_cs:pytb.ThunderBoltz.plot_cs}]{\sphinxcrossref{\sphinxcode{\sphinxupquote{pytb.ThunderBoltz.plot\_cs}}}}}({[}ax, legend, ...{]})
&
\sphinxAtStartPar
Plot the cross sections models.
\\
\sphinxbottomrule
\end{tabulary}
\sphinxtableafterendhook\par
\sphinxattableend\end{savenotes}

\sphinxAtStartPar
These functions will plot the data into \sphinxhref{https://matplotlib.org/stable/api/figure\_api.html\#matplotlib.figure.Figure}{\sphinxcode{\sphinxupquote{Figure}}} objects,
but in order to see the plots in a GUI, you must import the plotting library
and include the line \sphinxcode{\sphinxupquote{plt.show()}} after calling plotting methods like so:

\begin{sphinxVerbatim}[commandchars=\\\{\}]
\PYG{k+kn}{import} \PYG{n+nn}{pytb}

\PYG{c+c1}{\PYGZsh{} This will import the plotting library}
\PYG{k+kn}{import} \PYG{n+nn}{matplotlib}\PYG{n+nn}{.}\PYG{n+nn}{pyplot} \PYG{k}{as} \PYG{n+nn}{plt}

\PYG{c+c1}{\PYGZsh{} Either read in data, or run calculations}
\PYG{n}{tb} \PYG{o}{=} \PYG{n}{pytb}\PYG{o}{.}\PYG{n}{read}\PYG{p}{(}\PYG{l+s+s2}{\PYGZdq{}}\PYG{l+s+s2}{path/to/simulations\PYGZus{}to\PYGZus{}plot}\PYG{l+s+s2}{\PYGZdq{}}\PYG{p}{,} \PYG{n}{read\PYGZus{}cs\PYGZus{}data}\PYG{o}{=}\PYG{k+kc}{True}\PYG{p}{)}

\PYG{c+c1}{\PYGZsh{} Call plotting methods}
\PYG{n}{tb}\PYG{o}{.}\PYG{n}{plot\PYGZus{}cs}\PYG{p}{(}\PYG{p}{)}

\PYG{c+c1}{\PYGZsh{} Show the plots and load a GUI}
\PYG{n}{plt}\PYG{o}{.}\PYG{n}{show}\PYG{p}{(}\PYG{p}{)}
\end{sphinxVerbatim}

\sphinxAtStartPar
Alternatively, you may specify a directory within which to
save a pdf file of the plot when calling any \sphinxcode{\sphinxupquote{ThunderBoltz.plot\_*}} method.

\begin{sphinxVerbatim}[commandchars=\\\{\}]
\PYG{n}{tb}\PYG{o}{.}\PYG{n}{plot\PYGZus{}cs}\PYG{p}{(}\PYG{n}{save}\PYG{o}{=}\PYG{l+s+s2}{\PYGZdq{}}\PYG{l+s+s2}{path/to/figure\PYGZus{}directory}\PYG{l+s+s2}{\PYGZdq{}}\PYG{p}{)}
\end{sphinxVerbatim}


\chapter{Benchmark Testing}
\label{\detokenize{index:benchmark-testing}}
\sphinxAtStartPar
See {\hyperref[\detokenize{bm::doc}]{\sphinxcrossref{\DUrole{doc}{Benchmark Testing}}}} to run code that reproduces
the results found in the paper.

\sphinxstepscope


\section{Benchmark Testing}
\label{\detokenize{bm:benchmark-testing}}\label{\detokenize{bm::doc}}
\sphinxAtStartPar
There are three benchmark tests available in the
\sphinxhref{https://gitlab.com/Mczammit/thunderboltz}{repository}.
These benchmark simulations are described in detail in Sect. III of the ThunderBoltz paper.
The resulting calculations and figures from these benchmark tests can be compared directly
to the figures given in the paper. Each can be imported from the \sphinxcode{\sphinxupquote{run.py}} python module.


\subsection{Onsager Relation}
\label{\detokenize{bm:onsager-relation}}
\sphinxAtStartPar
The Onsager relation %
\begin{footnote}[1]\sphinxAtStartFootnote
Light, J. C., Ross, J., \& Shuler, K. E. (1969). Rate coefficients,
reaction cross sections and microscopic reversibility. Kinetic
Processes in Gases and Plasmas, 314, 281.
%
\end{footnote} predicts the kinetic rates of the following
chemical reactions between arbitrary heavy particles,
\begin{equation*}
\begin{split}A \rightleftarrows B \rightleftarrows C \rightleftarrows A.\end{split}
\end{equation*}
\sphinxAtStartPar
Based on the equilibrium condition, \(n_i k_{ij}=n_jk_{ji}\),
the rate constants \(k_{ij}\) have the analytic solution
\begin{equation*}
\begin{split}k(T)= 2d^2\left(\frac{2\pi k_B T}{m_r}\right)^{1/2}
\exp\left(
    \frac{-E_a}{k_B T}\right)\left(1+\frac{E_a}{k_B T}
\right).\end{split}
\end{equation*}
\sphinxAtStartPar
To run this system in in ThunderBoltz, run the prepared function either
in a python script:

\begin{sphinxVerbatim}[commandchars=\\\{\}]
\PYG{c+c1}{\PYGZsh{} Run this from within the repository root}
\PYG{k+kn}{import} \PYG{n+nn}{run}
\PYG{n}{run}\PYG{o}{.}\PYG{n}{onsager\PYGZus{}relation}\PYG{p}{(}\PYG{p}{)}
\end{sphinxVerbatim}

\sphinxAtStartPar
or directly from the command line:

\begin{sphinxVerbatim}[commandchars=\\\{\}]
python\PYG{+w}{ }\PYGZhy{}c\PYG{+w}{ }\PYG{l+s+s2}{\PYGZdq{}import run; run.onsager\PYGZus{}relation()\PYGZdq{}}
\end{sphinxVerbatim}

\sphinxAtStartPar
The resulting calculation will automatically run in the
directory \sphinxcode{\sphinxupquote{simulations/onsager\_relation}}.

\sphinxAtStartPar
Once the simulation has finished, run the following on the command line
(or in a python script) to view a time evolution of the species densities,
reaction rates, and absolute rates.

\begin{sphinxVerbatim}[commandchars=\\\{\}]
python\PYG{+w}{ }\PYGZhy{}c\PYG{+w}{ }\PYG{l+s+s2}{\PYGZdq{}import visualize; visualize.plot\PYGZus{}onsager()\PYGZdq{}}
\end{sphinxVerbatim}

\sphinxAtStartPar
This will automatically save a pdf of the plot in the \sphinxcode{\sphinxupquote{simulations}}
directory.


\subsection{Ikuta\sphinxhyphen{}Sugai}
\label{\detokenize{bm:ikuta-sugai}}
\sphinxAtStartPar
The Ikuta\sphinxhyphen{}Sugai benchmark problem tests electron transport
in crossed electric and magnetic fields.

\sphinxAtStartPar
To run this system in ThunderBoltz and compare it to the analytic
theory presented by Ness %
\begin{footnote}[2]\sphinxAtStartFootnote
K F Ness 1994 J. Phys. D: Appl. Phys. 27 1848.
%
\end{footnote}, run the prepared function either in a python
script of directly from the command line:

\begin{sphinxVerbatim}[commandchars=\\\{\}]
python\PYG{+w}{ }\PYGZhy{}c\PYG{+w}{ }\PYG{l+s+s2}{\PYGZdq{}import run; run.ikuta\PYGZus{}sugai()\PYGZdq{}}
\end{sphinxVerbatim}

\sphinxAtStartPar
Once the simulation has finished, run the following command to view
the effect of the magnetic field on the average velocity moments
and mean energy of the particles in comparison to Ness:

\begin{sphinxVerbatim}[commandchars=\\\{\}]
python\PYG{+w}{ }\PYGZhy{}c\PYG{+w}{ }\PYG{l+s+s2}{\PYGZdq{}import visualize; visualize.plot\PYGZus{}ikuta\PYGZus{}sugai()\PYGZdq{}}
\end{sphinxVerbatim}

\sphinxAtStartPar
This will automatically save a pdf of the plot in the \sphinxcode{\sphinxupquote{simulations}}
directory.


\subsection{He Transport}
\label{\detokenize{bm:he-transport}}
\sphinxAtStartPar
Here we generate comparisons of bulk and flux electron mobility, \(\mu N\), and
Townshend ionization coefficient, \(\alpha / N\), at various reduced fields.
We compare ThunderBoltz results to the two\sphinxhyphen{}term Boltzmann equation solver, BOLSIG,
as well as some swarm experiments.

\sphinxAtStartPar
To simulate this system in ThunderBoltz run the prepared function either in a
python script or directly from the command line:

\begin{sphinxVerbatim}[commandchars=\\\{\}]
python\PYG{+w}{ }\PYGZhy{}c\PYG{+w}{ }\PYG{l+s+s2}{\PYGZdq{}import run; run.He\PYGZus{}transport()\PYGZdq{}}
\end{sphinxVerbatim}

\sphinxAtStartPar
Once the simulation has finished, run the following command to view
the reduced Townshend ionization coefficient and the reduced electron mobility
as a function of reduced electric field:

\begin{sphinxVerbatim}[commandchars=\\\{\}]
python\PYG{+w}{ }\PYGZhy{}c\PYG{+w}{ }\PYG{l+s+s2}{\PYGZdq{}import visualize; visualize.plot\PYGZus{}He\PYGZus{}transport()\PYGZdq{}}
\end{sphinxVerbatim}

\sphinxAtStartPar
This will automatically save a pdf of the plot in the \sphinxcode{\sphinxupquote{simulations}}
directory. To view a plot comparing the individual reaction rate coefficients of
ThunderBoltz and BOLSIG, run the following:

\begin{sphinxVerbatim}[commandchars=\\\{\}]
python\PYG{+w}{ }\PYGZhy{}c\PYG{+w}{ }\PYG{l+s+s2}{\PYGZdq{}import visualize; visualize.rate\PYGZus{}comp()\PYGZdq{}}
\end{sphinxVerbatim}


\chapter{Simulation Parameters}
\label{\detokenize{index:simulation-parameters}}
\sphinxAtStartPar
Review the {\hyperref[\detokenize{params::doc}]{\sphinxcrossref{\DUrole{doc}{Simulation Parameters}}}} for information on
the default behavior of the code, the available input options, and
details regarding output parameter definitions and interpretations.

\sphinxstepscope


\section{Simulation Parameters}
\label{\detokenize{params:simulation-parameters}}\label{\detokenize{params::doc}}

\subsection{Input Parameters}
\label{\detokenize{params:input-parameters}}

\begin{savenotes}\sphinxattablestart
\sphinxthistablewithglobalstyle
\sphinxthistablewithnovlinesstyle
\centering
\begin{tabulary}{\linewidth}[t]{\X{1}{2}\X{1}{2}}
\sphinxtoprule
\sphinxtableatstartofbodyhook
\sphinxAtStartPar
{\hyperref[\detokenize{api/pytb.parameters.TBParameters:pytb.parameters.TBParameters}]{\sphinxcrossref{\sphinxcode{\sphinxupquote{pytb.parameters.TBParameters}}}}}()
&
\sphinxAtStartPar
The ThunderBoltz simulation settings and their default values.
\\
\sphinxhline
\sphinxAtStartPar
{\hyperref[\detokenize{api/pytb.parameters.WrapParameters:pytb.parameters.WrapParameters}]{\sphinxcrossref{\sphinxcode{\sphinxupquote{pytb.parameters.WrapParameters}}}}}()
&
\sphinxAtStartPar
Additional Python interface settings and their defaults.
\\
\sphinxbottomrule
\end{tabulary}
\sphinxtableafterendhook\par
\sphinxattableend\end{savenotes}

\sphinxstepscope


\subsubsection{pytb.parameters.TBParameters}
\label{\detokenize{api/pytb.parameters.TBParameters:pytb-parameters-tbparameters}}\label{\detokenize{api/pytb.parameters.TBParameters::doc}}\index{TBParameters (class in pytb.parameters)@\spxentry{TBParameters}\spxextra{class in pytb.parameters}}

\begin{fulllineitems}
\phantomsection\label{\detokenize{api/pytb.parameters.TBParameters:pytb.parameters.TBParameters}}
\pysigstartsignatures
\pysigline{\sphinxbfcode{\sphinxupquote{class\DUrole{w,w}{  }}}\sphinxcode{\sphinxupquote{pytb.parameters.}}\sphinxbfcode{\sphinxupquote{TBParameters}}}
\pysigstopsignatures
\sphinxAtStartPar
The ThunderBoltz simulation settings and their
default values.
\subsubsection*{Attributes}


\begin{savenotes}\sphinxattablestart
\sphinxthistablewithglobalstyle
\sphinxthistablewithnovlinesstyle
\centering
\begin{tabulary}{\linewidth}[t]{\X{1}{2}\X{1}{2}}
\sphinxtoprule
\sphinxtableatstartofbodyhook
\sphinxAtStartPar
\sphinxcode{\sphinxupquote{B}}
&
\sphinxAtStartPar
(list{[}int{]}) Magnetic field vector (Tesla), default is \sphinxcode{\sphinxupquote{{[}0.0, 0.0, 0.0{]}}}.
\\
\sphinxhline
\sphinxAtStartPar
\sphinxcode{\sphinxupquote{CO}}
&
\sphinxAtStartPar
(str) Collision ordering, options are \sphinxcode{\sphinxupquote{"default"}} | \sphinxcode{\sphinxupquote{"Random"}} | \sphinxcode{\sphinxupquote{"Reverse"}}.
\\
\sphinxhline
\sphinxAtStartPar
\sphinxcode{\sphinxupquote{CR}}
&
\sphinxAtStartPar
(int) If \sphinxcode{\sphinxupquote{1}}, then the remainder of \(N_{\rm pairs}\) is carried into the next \(N_{\rm pairs}\) evaluation of the same process, default is \sphinxcode{\sphinxupquote{1}}.
\\
\sphinxhline
\sphinxAtStartPar
\sphinxcode{\sphinxupquote{DT}}
&
\sphinxAtStartPar
(float) Time increment interval (s), default is \sphinxcode{\sphinxupquote{5e\sphinxhyphen{}12}}.
\\
\sphinxhline
\sphinxAtStartPar
\sphinxcode{\sphinxupquote{E}}
&
\sphinxAtStartPar
(float) Electric field in z\sphinxhyphen{}direction (V/m), default is \sphinxcode{\sphinxupquote{\sphinxhyphen{}24640.0}}.
\\
\sphinxhline
\sphinxAtStartPar
\sphinxcode{\sphinxupquote{ET}}
&
\sphinxAtStartPar
(float) E\sphinxhyphen{}field oscillation frequency (Hz), default is \sphinxcode{\sphinxupquote{0}}.
\\
\sphinxhline
\sphinxAtStartPar
\sphinxcode{\sphinxupquote{EX}}
&
\sphinxAtStartPar
(int) Cross section extrapolation — options are \sphinxcode{\sphinxupquote{0}} (extrapolated cross sections are set to \(0\) m \(^2\)), or \sphinxcode{\sphinxupquote{1}} (linearly extrapolated from last two points), default is \sphinxcode{\sphinxupquote{0}}.
\\
\sphinxhline
\sphinxAtStartPar
\sphinxcode{\sphinxupquote{FV}}
&
\sphinxAtStartPar
(list{[}int{]}) Output velocity dump settings {[}start, stride, species ID{]}, default is \sphinxcode{\sphinxupquote{{[}1000, 1000000, 0{]}}}.
\\
\sphinxhline
\sphinxAtStartPar
\sphinxcode{\sphinxupquote{L}}
&
\sphinxAtStartPar
(float) Cell length (m), default is \sphinxcode{\sphinxupquote{1e\sphinxhyphen{}6}}.
\\
\sphinxhline
\sphinxAtStartPar
\sphinxcode{\sphinxupquote{LV}}
&
\sphinxAtStartPar
(list{[}str,int{]}) Optionally load particle velocities from a comma separated text file, default is \sphinxcode{\sphinxupquote{None}}; specify the name of the file at index \sphinxcode{\sphinxupquote{0}} and the particle species index it applies to at index \sphinxcode{\sphinxupquote{1}}.
\\
\sphinxhline
\sphinxAtStartPar
\sphinxcode{\sphinxupquote{MEM}}
&
\sphinxAtStartPar
(float) Request memory (GB) for particle arrays.
\\
\sphinxhline
\sphinxAtStartPar
\sphinxcode{\sphinxupquote{MP}}
&
\sphinxAtStartPar
(list{[}float{]}) Mass of each particle species (amu), default is \sphinxcode{\sphinxupquote{{[}5.4857e\sphinxhyphen{}4, 28.0{]}}}.
\\
\sphinxhline
\sphinxAtStartPar
\sphinxcode{\sphinxupquote{NP}}
&
\sphinxAtStartPar
(list{[}int{]}) Number of particles for each species, default is \sphinxcode{\sphinxupquote{{[}10000, 1000{]}}}.
\\
\sphinxhline
\sphinxAtStartPar
\sphinxcode{\sphinxupquote{NS}}
&
\sphinxAtStartPar
(int) Number of time steps, default is \sphinxcode{\sphinxupquote{1000001}}.
\\
\sphinxhline
\sphinxAtStartPar
\sphinxcode{\sphinxupquote{OS}}
&
\sphinxAtStartPar
(int) Time step stride for output parameters, default is \sphinxcode{\sphinxupquote{100}}.
\\
\sphinxhline
\sphinxAtStartPar
\sphinxcode{\sphinxupquote{QP}}
&
\sphinxAtStartPar
(list{[}int{]}) Charge (elementary units) of each particle species, default is \sphinxcode{\sphinxupquote{{[}\sphinxhyphen{}1.0, 0.0{]}}}.
\\
\sphinxhline
\sphinxAtStartPar
\sphinxcode{\sphinxupquote{SE}}
&
\sphinxAtStartPar
(int) When using a SLURM manager on HPC, auto dump particle velocity data before job allocation runs out — options are \sphinxcode{\sphinxupquote{0}} (don\textquotesingle{}t auto dump) | \sphinxcode{\sphinxupquote{1}} (dump using SLURM setup).
\\
\sphinxhline
\sphinxAtStartPar
\sphinxcode{\sphinxupquote{SP}}
&
\sphinxAtStartPar
(int) Number of species, default is \sphinxcode{\sphinxupquote{2}}.
\\
\sphinxhline
\sphinxAtStartPar
\sphinxcode{\sphinxupquote{TP}}
&
\sphinxAtStartPar
(list{[}float{]}) Temperature (eV) of each particle species, default is \sphinxcode{\sphinxupquote{{[}0.0, 0.0259{]}}}.
\\
\sphinxhline
\sphinxAtStartPar
\sphinxcode{\sphinxupquote{VS}}
&
\sphinxAtStartPar
(int) Number of random samples used to find \(\max_\epsilon (v\sigma(\epsilon))\) for each process, default is \sphinxcode{\sphinxupquote{1000}}.
\\
\sphinxhline
\sphinxAtStartPar
\sphinxcode{\sphinxupquote{VV}}
&
\sphinxAtStartPar
(list{[}float{]}) Flow velocity for each particle, default is \sphinxcode{\sphinxupquote{{[}0.0, 0.0{]}}}.
\\
\sphinxbottomrule
\end{tabulary}
\sphinxtableafterendhook\par
\sphinxattableend\end{savenotes}
\subsubsection*{Methods}


\begin{savenotes}\sphinxattablestart
\sphinxthistablewithglobalstyle
\sphinxthistablewithnovlinesstyle
\centering
\begin{tabulary}{\linewidth}[t]{\X{1}{2}\X{1}{2}}
\sphinxtoprule
\sphinxtableatstartofbodyhook
\sphinxAtStartPar
{\hyperref[\detokenize{api/pytb.parameters.TBParameters.get_params:pytb.parameters.TBParameters.get_params}]{\sphinxcrossref{\sphinxcode{\sphinxupquote{get\_params}}}}}()
&
\sphinxAtStartPar
Return the set of parameters and their default values as a python dictionary.
\\
\sphinxbottomrule
\end{tabulary}
\sphinxtableafterendhook\par
\sphinxattableend\end{savenotes}

\sphinxstepscope


\paragraph{pytb.parameters.TBParameters.get\_params}
\label{\detokenize{api/pytb.parameters.TBParameters.get_params:pytb-parameters-tbparameters-get-params}}\label{\detokenize{api/pytb.parameters.TBParameters.get_params::doc}}\index{get\_params() (pytb.parameters.TBParameters method)@\spxentry{get\_params()}\spxextra{pytb.parameters.TBParameters method}}

\begin{fulllineitems}
\phantomsection\label{\detokenize{api/pytb.parameters.TBParameters.get_params:pytb.parameters.TBParameters.get_params}}
\pysigstartsignatures
\pysiglinewithargsret{\sphinxcode{\sphinxupquote{TBParameters.}}\sphinxbfcode{\sphinxupquote{get\_params}}}{}{}
\pysigstopsignatures
\sphinxAtStartPar
Return the set of parameters and their default values
as a python dictionary.

\end{fulllineitems}


\end{fulllineitems}


\sphinxstepscope


\subsubsection{pytb.parameters.WrapParameters}
\label{\detokenize{api/pytb.parameters.WrapParameters:pytb-parameters-wrapparameters}}\label{\detokenize{api/pytb.parameters.WrapParameters::doc}}\index{WrapParameters (class in pytb.parameters)@\spxentry{WrapParameters}\spxextra{class in pytb.parameters}}

\begin{fulllineitems}
\phantomsection\label{\detokenize{api/pytb.parameters.WrapParameters:pytb.parameters.WrapParameters}}
\pysigstartsignatures
\pysigline{\sphinxbfcode{\sphinxupquote{class\DUrole{w,w}{  }}}\sphinxcode{\sphinxupquote{pytb.parameters.}}\sphinxbfcode{\sphinxupquote{WrapParameters}}}
\pysigstopsignatures
\sphinxAtStartPar
Additional Python interface settings and
their defaults. An asterisk (*) at the beginning
of the description indicates a parameter that is
specific to the built\sphinxhyphen{}in He model.
\subsubsection*{Attributes}


\begin{savenotes}\sphinxattablestart
\sphinxthistablewithglobalstyle
\sphinxthistablewithnovlinesstyle
\centering
\begin{tabulary}{\linewidth}[t]{\X{1}{2}\X{1}{2}}
\sphinxtoprule
\sphinxtableatstartofbodyhook
\sphinxAtStartPar
\sphinxcode{\sphinxupquote{Bred}}
&
\sphinxAtStartPar
(list{[}float{]}) Specify reduced magnetic field (Hx), default is \sphinxcode{\sphinxupquote{None}}.
\\
\sphinxhline
\sphinxAtStartPar
\sphinxcode{\sphinxupquote{DE}}
&
\sphinxAtStartPar
(float) The maximum change in energy (eV) of a hypothetical electron per time step, default is \sphinxcode{\sphinxupquote{0.1}}.
\\
\sphinxhline
\sphinxAtStartPar
\sphinxcode{\sphinxupquote{ECS}}
&
\sphinxAtStartPar
{\color{red}\bfseries{}*}(str or None) The total elastic cross section model for the He built\sphinxhyphen{}in — options are \sphinxcode{\sphinxupquote{"ICS"}} or \sphinxcode{\sphinxupquote{"MTCS"}}, default is \sphinxcode{\sphinxupquote{"ICS"}} if an anisotropic angular distribution function is used and \sphinxcode{\sphinxupquote{"MTCS"}} if an isotropic angular distribution function is used.
\\
\sphinxhline
\sphinxAtStartPar
\sphinxcode{\sphinxupquote{EP\_0}}
&
\sphinxAtStartPar
(float) The initial energy (eV) of a hypothetical electron per time step, default is \sphinxcode{\sphinxupquote{10}}.
\\
\sphinxhline
\sphinxAtStartPar
\sphinxcode{\sphinxupquote{Ered}}
&
\sphinxAtStartPar
(float) Specify reduced electric field (Td), default is \sphinxcode{\sphinxupquote{None}}.
\\
\sphinxhline
\sphinxAtStartPar
\sphinxcode{\sphinxupquote{NN}}
&
\sphinxAtStartPar
(int) Number of background neutral gas particles (assumed gas species is of index 1), default is \sphinxcode{\sphinxupquote{None}}.
\\
\sphinxhline
\sphinxAtStartPar
\sphinxcode{\sphinxupquote{Nmin}}
&
\sphinxAtStartPar
(float) Minimum number of pseudo pairs to be generated by the smallest cross section of interest, default is \sphinxcode{\sphinxupquote{1.0}}.
\\
\sphinxhline
\sphinxAtStartPar
\sphinxcode{\sphinxupquote{analytic\_cs}}
&
\sphinxAtStartPar
{\color{red}\bfseries{}*}(str or bool) For the built\sphinxhyphen{}in \sphinxcode{\sphinxupquote{pytb.input.He\_TB}} Helium model, use either tabulated data, analytic fits, or a mix of both, options are \sphinxcode{\sphinxupquote{False}} | \sphinxcode{\sphinxupquote{True}} | \sphinxcode{\sphinxupquote{"mixed"}}.
\\
\sphinxhline
\sphinxAtStartPar
\sphinxcode{\sphinxupquote{autostep}}
&
\sphinxAtStartPar
(bool) Flag to calculate \sphinxcode{\sphinxupquote{DT}} / \sphinxcode{\sphinxupquote{NP}} / \sphinxcode{\sphinxupquote{E}} from \sphinxcode{\sphinxupquote{Ered}} / \sphinxcode{\sphinxupquote{L}} / \sphinxcode{\sphinxupquote{pct\_ion}} / \sphinxcode{\sphinxupquote{DE}} / \sphinxcode{\sphinxupquote{EP\_0}}, default is \sphinxcode{\sphinxupquote{False}}.
\\
\sphinxhline
\sphinxAtStartPar
\sphinxcode{\sphinxupquote{downsample}}
&
\sphinxAtStartPar
(bool) If specified with \sphinxcode{\sphinxupquote{vdf\_init}}, truncate the init file such that NP is satisfied, default is \sphinxcode{\sphinxupquote{False}}.
\\
\sphinxhline
\sphinxAtStartPar
\sphinxcode{\sphinxupquote{duration}}
&
\sphinxAtStartPar
(float) Run simulation until duration (s) is complete, default is \sphinxcode{\sphinxupquote{None}}.
\\
\sphinxbottomrule
\end{tabulary}
%%%%% SPLIT TABLE %%%%%
\begin{tabulary}{\linewidth}[t]{\X{1}{2}\X{1}{2}}
\sphinxtoprule
\sphinxtableatstartofbodyhook
\sphinxAtStartPar
\sphinxcode{\sphinxupquote{eadf}}
&
\sphinxAtStartPar
(str) Elastic angular distribution function model — options are \sphinxcode{\sphinxupquote{"default"}}, or \sphinxcode{\sphinxupquote{"He\_Park"}}.
\\
\sphinxhline
\sphinxAtStartPar
\sphinxcode{\sphinxupquote{eesd}}
&
\sphinxAtStartPar
(str) Electron energy sharing distribution model — options are \sphinxcode{\sphinxupquote{"default"}} (one takes all) | \sphinxcode{\sphinxupquote{"equal"}} | \sphinxcode{\sphinxupquote{"uniform"}}.
\\
\sphinxhline
\sphinxAtStartPar
\sphinxcode{\sphinxupquote{egen}}
&
\sphinxAtStartPar
{\color{red}\bfseries{}*}(bool) Allow secondary electron generation for the ionization model.
\\
\sphinxhline
\sphinxAtStartPar
\sphinxcode{\sphinxupquote{fixed\_background}}
&
\sphinxAtStartPar
(bool) Flag to append "FixedParticle2" to each of the reaction types in the indeck
\\
\sphinxhline
\sphinxAtStartPar
\sphinxcode{\sphinxupquote{gas\_index}}
&
\sphinxAtStartPar
(int) The index of the neutral gas species (if applicable), default is \sphinxcode{\sphinxupquote{1}}.
\\
\sphinxhline
\sphinxAtStartPar
\sphinxcode{\sphinxupquote{indeck}}
&
\sphinxAtStartPar
(callable or str) Function for auto\sphinxhyphen{}generating indeck and CS object or string path to directory with CS data and indeck, default is \sphinxcode{\sphinxupquote{None}}.
\\
\sphinxhline
\sphinxAtStartPar
\sphinxcode{\sphinxupquote{mix\_thresh}}
&
\sphinxAtStartPar
{\color{red}\bfseries{}*}(float) If \sphinxcode{\sphinxupquote{analytic\_cs}} is \sphinxcode{\sphinxupquote{"mixed"}}, use numerical data at energies lower than this threshold value (in eV), and use analytic data at higher energies.
\\
\sphinxhline
\sphinxAtStartPar
\sphinxcode{\sphinxupquote{n}}
&
\sphinxAtStartPar
{\color{red}\bfseries{}*}(int) For the built\sphinxhyphen{}in \sphinxcode{\sphinxupquote{pytb.input.He\_TB}} Helium model, include CCC excitation processes from the ground state to (up to and including) states with principle quantum number \sphinxcode{\sphinxupquote{n}}.
\\
\sphinxhline
\sphinxAtStartPar
\sphinxcode{\sphinxupquote{nsamples}}
&
\sphinxAtStartPar
{\color{red}\bfseries{}*}(int) For the built\sphinxhyphen{}in \sphinxcode{\sphinxupquote{pytb.input.He\_TB}}, this specifies the number of tabulated cross section values for analytic sampling.
\\
\sphinxhline
\sphinxAtStartPar
\sphinxcode{\sphinxupquote{pc\_scale}}
&
\sphinxAtStartPar
(dict{[}str\sphinxhyphen{}\textgreater{}float{]}) Multiply any property involved in the \(N_{\rm min}\) constraint by a constant after the constraint is imposed; useful for convergence testing, default is \sphinxcode{\sphinxupquote{\{\}}} (do nothing).
\\
\sphinxhline
\sphinxAtStartPar
\sphinxcode{\sphinxupquote{pct\_ion}}
&
\sphinxAtStartPar
(float) Set the ratio \(\frac{N_{\rm e}}{N_{\rm gas}}\), default is \sphinxcode{\sphinxupquote{None}}.
\\
\sphinxhline
\sphinxAtStartPar
\sphinxcode{\sphinxupquote{vdf\_init}}
&
\sphinxAtStartPar
(list{[}str or int or 2D\sphinxhyphen{}array, int{]}) Initialize particles with velocity data — a string at index \sphinxcode{\sphinxupquote{0}} will read velocity data from that file path; an int at index \sphinxcode{\sphinxupquote{0}} will attempt to reinitialize a previous calculation from a that time step if that step dump file is available; an array or DataFrame of shape (NP,3) at index \sphinxcode{\sphinxupquote{0}} will create a velocity initialization file with the provided data, the value at index 1 should represent the species type.
\\
\sphinxbottomrule
\end{tabulary}
\sphinxtableafterendhook\par
\sphinxattableend\end{savenotes}
\subsubsection*{Methods}


\begin{savenotes}\sphinxattablestart
\sphinxthistablewithglobalstyle
\sphinxthistablewithnovlinesstyle
\centering
\begin{tabulary}{\linewidth}[t]{\X{1}{2}\X{1}{2}}
\sphinxtoprule
\sphinxtableatstartofbodyhook
\sphinxAtStartPar
{\hyperref[\detokenize{api/pytb.parameters.WrapParameters.get_params:pytb.parameters.WrapParameters.get_params}]{\sphinxcrossref{\sphinxcode{\sphinxupquote{get\_params}}}}}()
&
\sphinxAtStartPar
Return the set of parameters and their default values as a python dictionary.
\\
\sphinxbottomrule
\end{tabulary}
\sphinxtableafterendhook\par
\sphinxattableend\end{savenotes}

\sphinxstepscope


\paragraph{pytb.parameters.WrapParameters.get\_params}
\label{\detokenize{api/pytb.parameters.WrapParameters.get_params:pytb-parameters-wrapparameters-get-params}}\label{\detokenize{api/pytb.parameters.WrapParameters.get_params::doc}}\index{get\_params() (pytb.parameters.WrapParameters method)@\spxentry{get\_params()}\spxextra{pytb.parameters.WrapParameters method}}

\begin{fulllineitems}
\phantomsection\label{\detokenize{api/pytb.parameters.WrapParameters.get_params:pytb.parameters.WrapParameters.get_params}}
\pysigstartsignatures
\pysiglinewithargsret{\sphinxcode{\sphinxupquote{WrapParameters.}}\sphinxbfcode{\sphinxupquote{get\_params}}}{}{}
\pysigstopsignatures
\sphinxAtStartPar
Return the set of parameters and their default values
as a python dictionary.

\end{fulllineitems}


\end{fulllineitems}



\subsection{Output Parameters}
\label{\detokenize{params:output-parameters}}\label{\detokenize{params:output-params}}

\begin{savenotes}\sphinxattablestart
\sphinxthistablewithglobalstyle
\sphinxthistablewithnovlinesstyle
\centering
\begin{tabulary}{\linewidth}[t]{\X{1}{2}\X{1}{2}}
\sphinxtoprule
\sphinxtableatstartofbodyhook
\sphinxAtStartPar
{\hyperref[\detokenize{api/pytb.parameters.OutputParameters:pytb.parameters.OutputParameters}]{\sphinxcrossref{\sphinxcode{\sphinxupquote{pytb.parameters.OutputParameters}}}}}()
&
\sphinxAtStartPar
A listing of the main output parameters of the simulation, these keywords are the named columns of the time series and steady state data frames returned by {\hyperref[\detokenize{api/pytb.ThunderBoltz.get_timeseries:pytb.ThunderBoltz.get_timeseries}]{\sphinxcrossref{\sphinxcode{\sphinxupquote{get\_timeseries()}}}}} and {\hyperref[\detokenize{api/pytb.ThunderBoltz.get_ss_params:pytb.ThunderBoltz.get_ss_params}]{\sphinxcrossref{\sphinxcode{\sphinxupquote{get\_ss\_params()}}}}} respectively.
\\
\sphinxhline
\sphinxAtStartPar
{\hyperref[\detokenize{api/pytb.parameters.ParticleParameters:pytb.parameters.ParticleParameters}]{\sphinxcrossref{\sphinxcode{\sphinxupquote{pytb.parameters.ParticleParameters}}}}}()
&
\sphinxAtStartPar
A listing of species dependent properties that can be accessed by {\hyperref[\detokenize{api/pytb.ThunderBoltz.get_particle_tables:pytb.ThunderBoltz.get_particle_tables}]{\sphinxcrossref{\sphinxcode{\sphinxupquote{get\_particle\_tables()}}}}}, which returns a list of data tables (one for each species) where each column of data is labeled with one of the following keywords.
\\
\sphinxbottomrule
\end{tabulary}
\sphinxtableafterendhook\par
\sphinxattableend\end{savenotes}

\sphinxstepscope


\subsubsection{pytb.parameters.OutputParameters}
\label{\detokenize{api/pytb.parameters.OutputParameters:pytb-parameters-outputparameters}}\label{\detokenize{api/pytb.parameters.OutputParameters::doc}}\index{OutputParameters (class in pytb.parameters)@\spxentry{OutputParameters}\spxextra{class in pytb.parameters}}

\begin{fulllineitems}
\phantomsection\label{\detokenize{api/pytb.parameters.OutputParameters:pytb.parameters.OutputParameters}}
\pysigstartsignatures
\pysigline{\sphinxbfcode{\sphinxupquote{class\DUrole{w,w}{  }}}\sphinxcode{\sphinxupquote{pytb.parameters.}}\sphinxbfcode{\sphinxupquote{OutputParameters}}}
\pysigstopsignatures
\sphinxAtStartPar
A listing of the main output parameters of the simulation, these keywords
are the named columns of the time series and steady state data frames
returned by {\hyperref[\detokenize{api/pytb.ThunderBoltz.get_timeseries:pytb.ThunderBoltz.get_timeseries}]{\sphinxcrossref{\sphinxcode{\sphinxupquote{get\_timeseries()}}}}} and
{\hyperref[\detokenize{api/pytb.ThunderBoltz.get_ss_params:pytb.ThunderBoltz.get_ss_params}]{\sphinxcrossref{\sphinxcode{\sphinxupquote{get\_ss\_params()}}}}} respectively. These data tables also
include the {\hyperref[\detokenize{api/pytb.parameters.ParticleParameters:pytb.parameters.ParticleParameters}]{\sphinxcrossref{\sphinxcode{\sphinxupquote{ParticleParameters}}}}} of the species at index \(0\).
The steady state parameters returned by {\hyperref[\detokenize{api/pytb.ThunderBoltz.get_ss_params:pytb.ThunderBoltz.get_ss_params}]{\sphinxcrossref{\sphinxcode{\sphinxupquote{get\_ss\_params()}}}}}
will also include standard deviations for each parameter indicated by an
added “\_std” suffix.
\subsubsection*{Attributes}


\begin{savenotes}\sphinxattablestart
\sphinxthistablewithglobalstyle
\sphinxthistablewithnovlinesstyle
\centering
\begin{tabulary}{\linewidth}[t]{\X{1}{2}\X{1}{2}}
\sphinxtoprule
\sphinxtableatstartofbodyhook
\sphinxAtStartPar
\sphinxcode{\sphinxupquote{E}}
&
\sphinxAtStartPar
(float) The electric field component (V/m) in the \(z\) direction, which can change in AC scenarios.
\\
\sphinxhline
\sphinxAtStartPar
\sphinxcode{\sphinxupquote{MEe}}
&
\sphinxAtStartPar
(float) The mean energy (eV) of the species at index \(0\) (usually electrons), computed as \(\langle\epsilon\rangle = \frac{m_0}{2N_0}\sum_{i=1}^{N_0}v_{0i}^2\) where \(m_0\) and \(N_0\) are the mass and particle count of the \(0^{\rm th}\) species, and \(v_{0i}\) is the velocity vector of the \(i^{\rm th}\) particle of species \(0\).
\\
\sphinxhline
\sphinxAtStartPar
\sphinxcode{\sphinxupquote{a\_n}}
&
\sphinxAtStartPar
(float) The reduced flux Townshend ionization coefficient \((\text{m}^2)\) of the species at index \(0\), computed as \(\frac{\alpha^f}{n_{\rm gas}} = \frac{1}{n_0 n_{\rm gas}}\frac{dC_{\rm ion}}{dt}\times \left( \frac{1}{N_0}\sum_{i=1}^{N_0}v_{\parallel,0i}\right)^{-1}\) where \(N_0\) is the particle count of the \(0^{\rm th}\) species, \(C_{\rm ion}\) is the count of ionization events, \(n_0\) and \(n_{\rm gas}\) are the \(0^{\rm th}\) and background gas densities repectively, and \(v_{\parallel,0i}\) is velocity component parallel to the E field vector of the \(i^{\rm th}\) th particle of species \(0\).
\\
\sphinxhline
\sphinxAtStartPar
\sphinxcode{\sphinxupquote{a\_n\_bulk}}
&
\sphinxAtStartPar
(float) The reduced bulk Townshend ionization coefficient \((\text{m}^2)\) of the species at index \(0\), computed as \(\frac{\alpha^b}{n_{\rm gas}} = \frac{1}{n_0 n_{\rm gas}}\frac{dC_{\rm ion}}{dt}\times\left( \frac{d}{dt} \left[\frac{1}{N_0}\sum_{i=1}^{N_0}r_{\parallel,0i}\right] \right)^{-1}\), where \(N_0\) is the particle count of the \(0^{\rm th}\) species, \(C_{\rm ion}\) is the count of ionization events, \(n_0\) and \(n_{\rm gas}\) are the \(0^{\rm th}\) and background gas densities respectively, and \(r_{\parallel,0i}\) is displacement component parallel to the E field vector of the \(i^{\rm th}\) th particle of species \(0\).
\\
\sphinxhline
\sphinxAtStartPar
\sphinxcode{\sphinxupquote{k\_1}}
&
\sphinxAtStartPar
(float) The rate coefficient for the first reaction, computed as \(\frac{1}{n_0n_{\rm gas}} \frac{dC_{\rm 1}}{dt}\).
\\
\sphinxhline
\sphinxAtStartPar
\sphinxcode{\sphinxupquote{k\_ion}}
&
\sphinxAtStartPar
(float) The ionization rate coefficient, computed as \(\frac{1}{n_0n_{\rm gas}} \frac{dC_{\rm ion}}{dt}\).
\\
\sphinxhline
\sphinxAtStartPar
\sphinxcode{\sphinxupquote{mobN}}
&
\sphinxAtStartPar
(float) The reduced flux mobility \((\text{V}^{-1}\text{m}^{-1}\text{s}^{-1})\) of the species at index \(0\), computed as \(\mu^fn_{\rm gas}=\frac{n_{\rm gas}}{E}\frac{1}{N_0}\sum_{i=1}^{N_0}v_{\parallel,0i}\) where \(N_0\) is the particle count of the \(0^{\rm th}\) species, \(E\) is field magnitude, \(n_{\rm gas}\) is the density of the background gas, and \(v_{\parallel,0i}\) is the velocity component parallel to the E field vector of the \(i^{\rm th}\) th particle of species \(0\).
\\
\sphinxhline
\sphinxAtStartPar
\sphinxcode{\sphinxupquote{mobN\_bulk}}
&
\sphinxAtStartPar
(float) The reduced bulk mobility \((\text{V}^{-1}\text{m}^{-1}\text{s}^{-1})\) of the species at index \(0\), computed as \(\mu^b n_{\rm gas}=\frac{n_{\rm gas}}{E}\frac{d}{dt}\left[\frac{1}{N_0}\sum_{i=1}^{N_0}r_{\parallel,0i}\right]\) where \(N_0\) is the particle count of the \(0^{\rm th}\) species, \(E\) is field magnitude, \(n_{\rm gas}\) is the density of the background gas, and \(r_{\parallel,0i}\) is displacement component parallel to the E field vector of the \(i^{\rm th}\) th particle of species \(0\).
\\
\sphinxhline
\sphinxAtStartPar
\sphinxcode{\sphinxupquote{n\_gas}}
&
\sphinxAtStartPar
(float) \(n_{\rm gas}\), the number density of the background gas.
\\
\sphinxhline
\sphinxAtStartPar
\sphinxcode{\sphinxupquote{step}}
&
\sphinxAtStartPar
(int) The number of time steps elapsed in the simulation, with \sphinxcode{\sphinxupquote{t}} \(=0\) corresponding to \sphinxcode{\sphinxupquote{step}} \(=0\), and with \sphinxcode{\sphinxupquote{t}} = \sphinxcode{\sphinxupquote{DT}} corresponding to \sphinxcode{\sphinxupquote{step}} \(=1\).
\\
\sphinxhline
\sphinxAtStartPar
\sphinxcode{\sphinxupquote{t}}
&
\sphinxAtStartPar
(float) The time (s) elapsed in the simulation.
\\
\sphinxbottomrule
\end{tabulary}
\sphinxtableafterendhook\par
\sphinxattableend\end{savenotes}
\subsubsection*{Methods}


\begin{savenotes}\sphinxattablestart
\sphinxthistablewithglobalstyle
\sphinxthistablewithnovlinesstyle
\centering
\begin{tabulary}{\linewidth}[t]{\X{1}{2}\X{1}{2}}
\sphinxtoprule
\sphinxtableatstartofbodyhook
\sphinxAtStartPar
{\hyperref[\detokenize{api/pytb.parameters.OutputParameters.get_params:pytb.parameters.OutputParameters.get_params}]{\sphinxcrossref{\sphinxcode{\sphinxupquote{get\_params}}}}}()
&
\sphinxAtStartPar
Return the set of parameters and their default values as a python dictionary.
\\
\sphinxbottomrule
\end{tabulary}
\sphinxtableafterendhook\par
\sphinxattableend\end{savenotes}

\sphinxstepscope


\paragraph{pytb.parameters.OutputParameters.get\_params}
\label{\detokenize{api/pytb.parameters.OutputParameters.get_params:pytb-parameters-outputparameters-get-params}}\label{\detokenize{api/pytb.parameters.OutputParameters.get_params::doc}}\index{get\_params() (pytb.parameters.OutputParameters method)@\spxentry{get\_params()}\spxextra{pytb.parameters.OutputParameters method}}

\begin{fulllineitems}
\phantomsection\label{\detokenize{api/pytb.parameters.OutputParameters.get_params:pytb.parameters.OutputParameters.get_params}}
\pysigstartsignatures
\pysiglinewithargsret{\sphinxcode{\sphinxupquote{OutputParameters.}}\sphinxbfcode{\sphinxupquote{get\_params}}}{}{}
\pysigstopsignatures
\sphinxAtStartPar
Return the set of parameters and their default values
as a python dictionary.

\end{fulllineitems}


\end{fulllineitems}


\sphinxstepscope


\subsubsection{pytb.parameters.ParticleParameters}
\label{\detokenize{api/pytb.parameters.ParticleParameters:pytb-parameters-particleparameters}}\label{\detokenize{api/pytb.parameters.ParticleParameters::doc}}\index{ParticleParameters (class in pytb.parameters)@\spxentry{ParticleParameters}\spxextra{class in pytb.parameters}}

\begin{fulllineitems}
\phantomsection\label{\detokenize{api/pytb.parameters.ParticleParameters:pytb.parameters.ParticleParameters}}
\pysigstartsignatures
\pysigline{\sphinxbfcode{\sphinxupquote{class\DUrole{w,w}{  }}}\sphinxcode{\sphinxupquote{pytb.parameters.}}\sphinxbfcode{\sphinxupquote{ParticleParameters}}}
\pysigstopsignatures
\sphinxAtStartPar
A listing of species dependent properties that can be accessed by
{\hyperref[\detokenize{api/pytb.ThunderBoltz.get_particle_tables:pytb.ThunderBoltz.get_particle_tables}]{\sphinxcrossref{\sphinxcode{\sphinxupquote{get\_particle\_tables()}}}}}, which returns a list of data tables
(one for each species) where each column of data is labeled with one of the
following keywords.
\subsubsection*{Attributes}


\begin{savenotes}\sphinxattablestart
\sphinxthistablewithglobalstyle
\sphinxthistablewithnovlinesstyle
\centering
\begin{tabulary}{\linewidth}[t]{\X{1}{2}\X{1}{2}}
\sphinxtoprule
\sphinxtableatstartofbodyhook
\sphinxAtStartPar
\sphinxcode{\sphinxupquote{Ki}}
&
\sphinxAtStartPar
(float) The total kinetic energy (eV).
\\
\sphinxhline
\sphinxAtStartPar
\sphinxcode{\sphinxupquote{Mi}}
&
\sphinxAtStartPar
(float) The mean kinetic energy (eV).
\\
\sphinxhline
\sphinxAtStartPar
\sphinxcode{\sphinxupquote{Ni}}
&
\sphinxAtStartPar
(float) The number density (m \(^{-3}\)).
\\
\sphinxhline
\sphinxAtStartPar
\sphinxcode{\sphinxupquote{Rxi}}
&
\sphinxAtStartPar
(float) The mean x component of all particle displacements (m).
\\
\sphinxhline
\sphinxAtStartPar
\sphinxcode{\sphinxupquote{Ryi}}
&
\sphinxAtStartPar
(float) The mean y component of all particle displacements (m).
\\
\sphinxhline
\sphinxAtStartPar
\sphinxcode{\sphinxupquote{Rzi}}
&
\sphinxAtStartPar
(float) The mean z component of all particle displacements (m).
\\
\sphinxhline
\sphinxAtStartPar
\sphinxcode{\sphinxupquote{Txi}}
&
\sphinxAtStartPar
(float) The mean x component temperature (eV).
\\
\sphinxhline
\sphinxAtStartPar
\sphinxcode{\sphinxupquote{Tyi}}
&
\sphinxAtStartPar
(float) The mean y component temperature (eV).
\\
\sphinxhline
\sphinxAtStartPar
\sphinxcode{\sphinxupquote{Tzi}}
&
\sphinxAtStartPar
(float) The mean z component temperature (eV).
\\
\sphinxhline
\sphinxAtStartPar
\sphinxcode{\sphinxupquote{Vxi}}
&
\sphinxAtStartPar
(float) The mean x component velocity (m/s).
\\
\sphinxhline
\sphinxAtStartPar
\sphinxcode{\sphinxupquote{Vyi}}
&
\sphinxAtStartPar
(float) The mean y component velocity (m/s).
\\
\sphinxhline
\sphinxAtStartPar
\sphinxcode{\sphinxupquote{Vzi}}
&
\sphinxAtStartPar
(float) The mean z component velocity (m/s).
\\
\sphinxhline
\sphinxAtStartPar
\sphinxcode{\sphinxupquote{step}}
&
\sphinxAtStartPar
(int) The number of time steps elapsed in the simulation, with \sphinxcode{\sphinxupquote{t}} \(=0\) corresponding to \sphinxcode{\sphinxupquote{step}} \(=0\).
\\
\sphinxhline
\sphinxAtStartPar
\sphinxcode{\sphinxupquote{t}}
&
\sphinxAtStartPar
(float) The time (s) elapsed in the simulation.
\\
\sphinxbottomrule
\end{tabulary}
\sphinxtableafterendhook\par
\sphinxattableend\end{savenotes}
\subsubsection*{Methods}


\begin{savenotes}\sphinxattablestart
\sphinxthistablewithglobalstyle
\sphinxthistablewithnovlinesstyle
\centering
\begin{tabulary}{\linewidth}[t]{\X{1}{2}\X{1}{2}}
\sphinxtoprule
\sphinxtableatstartofbodyhook
\sphinxAtStartPar
{\hyperref[\detokenize{api/pytb.parameters.ParticleParameters.get_params:pytb.parameters.ParticleParameters.get_params}]{\sphinxcrossref{\sphinxcode{\sphinxupquote{get\_params}}}}}()
&
\sphinxAtStartPar
Return the set of parameters and their default values as a python dictionary.
\\
\sphinxbottomrule
\end{tabulary}
\sphinxtableafterendhook\par
\sphinxattableend\end{savenotes}

\sphinxstepscope


\paragraph{pytb.parameters.ParticleParameters.get\_params}
\label{\detokenize{api/pytb.parameters.ParticleParameters.get_params:pytb-parameters-particleparameters-get-params}}\label{\detokenize{api/pytb.parameters.ParticleParameters.get_params::doc}}\index{get\_params() (pytb.parameters.ParticleParameters method)@\spxentry{get\_params()}\spxextra{pytb.parameters.ParticleParameters method}}

\begin{fulllineitems}
\phantomsection\label{\detokenize{api/pytb.parameters.ParticleParameters.get_params:pytb.parameters.ParticleParameters.get_params}}
\pysigstartsignatures
\pysiglinewithargsret{\sphinxcode{\sphinxupquote{ParticleParameters.}}\sphinxbfcode{\sphinxupquote{get\_params}}}{}{}
\pysigstopsignatures
\sphinxAtStartPar
Return the set of parameters and their default values
as a python dictionary.

\end{fulllineitems}


\end{fulllineitems}



\subsection{Electron Growth and Memory Management}
\label{\detokenize{params:electron-growth-and-memory-management}}\label{\detokenize{params:memory}}
\sphinxAtStartPar
Depending on the ionization model and field strength,
ThunderBoltz may generate a large number of electrons.
In these cases, the appropriate amount of memory must be
allocated. The correct amount will be allocated automatically
in scenarios where no ionization process is used,
or when the \sphinxcode{\sphinxupquote{IonizationNoEgen}} model is used. This amount
will be allocated based on the sum of all \sphinxcode{\sphinxupquote{NP}} elements
times 4.

\sphinxAtStartPar
However, in scenarios where there is significant electron generation,
i.e. at high \(E\) fields with the \sphinxcode{\sphinxupquote{Ionization}} model on,
the default memory settings are not sufficient and the simulation
will exit with the error “Too many particles!”. To prevent this
specify the \sphinxcode{\sphinxupquote{MEM}} flag in the {\hyperref[\detokenize{api/pytb.ThunderBoltz:pytb.ThunderBoltz}]{\sphinxcrossref{\sphinxcode{\sphinxupquote{ThunderBoltz}}}}} constructor:

\begin{sphinxVerbatim}[commandchars=\\\{\}]
\PYG{n}{tb} \PYG{o}{=} \PYG{n}{ThunderBoltz}\PYG{p}{(}
     \PYG{c+c1}{\PYGZsh{} For example, using the Helium model.}
     \PYG{n}{indeck}\PYG{o}{=}\PYG{n}{pytb}\PYG{o}{.}\PYG{n}{input}\PYG{o}{.}\PYG{n}{He\PYGZus{}TB}\PYG{p}{,}
     \PYG{c+c1}{\PYGZsh{} This will turn on electron generation for the Helium model}
     \PYG{c+c1}{\PYGZsh{} i.e. this will ensure the \PYGZdq{}Ionization\PYGZdq{} collision model is}
     \PYG{c+c1}{\PYGZsh{} used in the generated indeck.}
     \PYG{n}{egen}\PYG{o}{=}\PYG{k+kc}{True}\PYG{p}{,}
     \PYG{c+c1}{\PYGZsh{} Now we must set the MEM flag, since we will be generating}
     \PYG{c+c1}{\PYGZsh{} a lot of electrons.}
     \PYG{n}{MEM} \PYG{o}{=} \PYG{l+m+mi}{10}\PYG{p}{,} \PYG{c+c1}{\PYGZsh{} in GB}
\PYG{p}{)}
\end{sphinxVerbatim}

\sphinxAtStartPar
MEM will accept any float representing the number of gigabytes
to be made available to the particle arrays.

\begin{sphinxadmonition}{warning}{Warning:}
\sphinxAtStartPar
If the value of \sphinxcode{\sphinxupquote{MEM}} is more than the actual number of
available GB, then the simulation will still run, but will
exit with a segmentation fault once too many particles are
created.
\end{sphinxadmonition}

\begin{sphinxadmonition}{warning}{Warning:}
\sphinxAtStartPar
When using multiple cores on the same machine / node, ensure
that each process has enough memory requested and that
the sum of memory requests does not exceed the available
pool of RAM.
\end{sphinxadmonition}


\chapter{API Reference}
\label{\detokenize{index:api-reference}}
\sphinxAtStartPar
See the full {\hyperref[\detokenize{ref::doc}]{\sphinxcrossref{\DUrole{doc}{API Reference}}}} for full documentation
of the ThunderBoltz programming interface.

\sphinxstepscope


\section{API Reference}
\label{\detokenize{ref:api-reference}}\label{\detokenize{ref::doc}}

\begin{savenotes}\sphinxattablestart
\sphinxthistablewithglobalstyle
\sphinxthistablewithnovlinesstyle
\centering
\begin{tabulary}{\linewidth}[t]{\X{1}{2}\X{1}{2}}
\sphinxtoprule
\sphinxtableatstartofbodyhook
\sphinxAtStartPar
{\hyperref[\detokenize{api/pytb.ThunderBoltz:pytb.ThunderBoltz}]{\sphinxcrossref{\sphinxcode{\sphinxupquote{ThunderBoltz}}}}}({[}directory, cs, out\_file, ...{]})
&
\sphinxAtStartPar
ThunderBoltz 0D DSMC simulation wrapper.
\\
\sphinxbottomrule
\end{tabulary}
\sphinxtableafterendhook\par
\sphinxattableend\end{savenotes}

\sphinxstepscope


\subsection{pytb.ThunderBoltz}
\label{\detokenize{api/pytb.ThunderBoltz:pytb-thunderboltz}}\label{\detokenize{api/pytb.ThunderBoltz::doc}}\index{ThunderBoltz (class in pytb)@\spxentry{ThunderBoltz}\spxextra{class in pytb}}

\begin{fulllineitems}
\phantomsection\label{\detokenize{api/pytb.ThunderBoltz:pytb.ThunderBoltz}}
\pysigstartsignatures
\pysiglinewithargsret{\sphinxbfcode{\sphinxupquote{class\DUrole{w,w}{  }}}\sphinxcode{\sphinxupquote{pytb.}}\sphinxbfcode{\sphinxupquote{ThunderBoltz}}}{\sphinxparam{\DUrole{n,n}{directory}\DUrole{o,o}{=}\DUrole{default_value}{None}}, \sphinxparam{\DUrole{n,n}{cs}\DUrole{o,o}{=}\DUrole{default_value}{None}}, \sphinxparam{\DUrole{n,n}{out\_file}\DUrole{o,o}{=}\DUrole{default_value}{None}}, \sphinxparam{\DUrole{n,n}{monitor}\DUrole{o,o}{=}\DUrole{default_value}{None}}, \sphinxparam{\DUrole{n,n}{live}\DUrole{o,o}{=}\DUrole{default_value}{None}}, \sphinxparam{\DUrole{n,n}{live\_rate}\DUrole{o,o}{=}\DUrole{default_value}{None}}, \sphinxparam{\DUrole{n,n}{ts\_plot\_params}\DUrole{o,o}{=}\DUrole{default_value}{None}}, \sphinxparam{\DUrole{o,o}{**}\DUrole{n,n}{params}}}{}
\pysigstopsignatures
\sphinxAtStartPar
ThunderBoltz 0D DSMC simulation wrapper.
\begin{quote}\begin{description}
\sphinxlineitem{Parameters}\begin{itemize}
\item {} 
\sphinxAtStartPar
\sphinxstyleliteralstrong{\sphinxupquote{directory}} (\sphinxhref{https://docs.python.org/3/library/stdtypes.html\#str}{\sphinxstyleliteralemphasis{\sphinxupquote{str}}}) \textendash{} The path to a directory that will host
ThunderBoltz compiled, source, input, and output files.

\item {} 
\sphinxAtStartPar
\sphinxstyleliteralstrong{\sphinxupquote{cs}} ({\hyperref[\detokenize{api/pytb.CrossSections:pytb.CrossSections}]{\sphinxcrossref{\sphinxstyleliteralemphasis{\sphinxupquote{CrossSections}}}}}) \textendash{} The set of cross section information
required for this simulation. Optionally supplied as an alternative
to \sphinxcode{\sphinxupquote{indeck}}, default is an empty CrossSections object.

\item {} 
\sphinxAtStartPar
\sphinxstyleliteralstrong{\sphinxupquote{out\_file}} (\sphinxhref{https://docs.python.org/3/library/stdtypes.html\#str}{\sphinxstyleliteralemphasis{\sphinxupquote{str}}}) \textendash{} Optional file base name for ThunderBoltz stdout buffer,
default is \sphinxcode{\sphinxupquote{"thunderboltz"}}.

\item {} 
\sphinxAtStartPar
\sphinxstyleliteralstrong{\sphinxupquote{monitor}} (\sphinxhref{https://docs.python.org/3/library/functions.html\#bool}{\sphinxstyleliteralemphasis{\sphinxupquote{bool}}}) \textendash{} Runtime flag, when set to \sphinxcode{\sphinxupquote{True}} an empty \sphinxcode{\sphinxupquote{monitor}}
file will be generated in the simulation directory.
Deleting this file will cause the ThunderBoltz process to exit,
but allow the wrapper to continue execution. This is useful
performing several simulation calculations sequentially,
but manual exit is required for each one, or if post processing
is required immediately after ThunderBoltz exits.

\item {} 
\sphinxAtStartPar
\sphinxstyleliteralstrong{\sphinxupquote{live}} (\sphinxhref{https://docs.python.org/3/library/functions.html\#bool}{\sphinxstyleliteralemphasis{\sphinxupquote{bool}}}) \textendash{} Run and update time series plotting GUI during simulation.

\item {} 
\sphinxAtStartPar
\sphinxstyleliteralstrong{\sphinxupquote{ts\_plot\_params}} (\sphinxhref{https://docs.python.org/3/library/stdtypes.html\#list}{\sphinxstyleliteralemphasis{\sphinxupquote{list}}}\sphinxstyleliteralemphasis{\sphinxupquote{{[}}}\sphinxhref{https://docs.python.org/3/library/stdtypes.html\#str}{\sphinxstyleliteralemphasis{\sphinxupquote{str}}}\sphinxstyleliteralemphasis{\sphinxupquote{{]}}}) \textendash{} The default output parameters to be plotted
by {\hyperref[\detokenize{api/pytb.ThunderBoltz.plot_timeseries:pytb.ThunderBoltz.plot_timeseries}]{\sphinxcrossref{\sphinxcode{\sphinxupquote{plot\_timeseries()}}}}}.

\item {} 
\sphinxAtStartPar
\sphinxstyleliteralstrong{\sphinxupquote{**params}} \textendash{} pytb and ThunderBoltz simulation parameters. Any attributes of
{\hyperref[\detokenize{api/pytb.parameters.TBParameters:pytb.parameters.TBParameters}]{\sphinxcrossref{\sphinxcode{\sphinxupquote{TBParameters}}}}} or
{\hyperref[\detokenize{api/pytb.parameters.WrapParameters:pytb.parameters.WrapParameters}]{\sphinxcrossref{\sphinxcode{\sphinxupquote{WrapParameters}}}}} can be passed here.

\end{itemize}

\end{description}\end{quote}
\subsubsection*{Attributes}


\begin{savenotes}\sphinxattablestart
\sphinxthistablewithglobalstyle
\sphinxthistablewithnovlinesstyle
\centering
\begin{tabulary}{\linewidth}[t]{\X{1}{2}\X{1}{2}}
\sphinxtoprule
\sphinxtableatstartofbodyhook
\sphinxAtStartPar
\sphinxcode{\sphinxupquote{logfile}}
&
\sphinxAtStartPar
Name of the simulation output file produced by \sphinxcode{\sphinxupquote{pytb}}
\\
\sphinxhline
\sphinxAtStartPar
\sphinxcode{\sphinxupquote{output\_files}}
&
\sphinxAtStartPar
Files to be read when tabular data is requested
\\
\sphinxhline
\sphinxAtStartPar
\sphinxcode{\sphinxupquote{ts\_plot\_params}}
&
\sphinxAtStartPar
Time series plot parameters
\\
\sphinxhline
\sphinxAtStartPar
\sphinxcode{\sphinxupquote{particle\_tables}}
&
\sphinxAtStartPar
Particle\sphinxhyphen{}specific times series data
\\
\sphinxhline
\sphinxAtStartPar
\sphinxcode{\sphinxupquote{kinetic\_table}}
&
\sphinxAtStartPar
Banner output data
\\
\sphinxhline
\sphinxAtStartPar
\sphinxcode{\sphinxupquote{timeseries}}
&
\sphinxAtStartPar
All tick\sphinxhyphen{}by\sphinxhyphen{}tick simulation data
\\
\sphinxhline
\sphinxAtStartPar
\sphinxcode{\sphinxupquote{vdfs}}
&
\sphinxAtStartPar
Particle velocity dump data
\\
\sphinxhline
\sphinxAtStartPar
\sphinxcode{\sphinxupquote{vdf\_init\_data}}
&
\sphinxAtStartPar
Particle velocity data intended for particle initialization
\\
\sphinxhline
\sphinxAtStartPar
\sphinxcode{\sphinxupquote{time\_conv}}
&
\sphinxAtStartPar
Time step at which steady\sphinxhyphen{}state calculations are considered converged
\\
\sphinxhline
\sphinxAtStartPar
\sphinxcode{\sphinxupquote{counts}}
&
\sphinxAtStartPar
Table of collision counts
\\
\sphinxhline
\sphinxAtStartPar
\sphinxcode{\sphinxupquote{elapsed\_time}}
&
\sphinxAtStartPar
Elapsed wall\sphinxhyphen{}clock time of calculation
\\
\sphinxhline
\sphinxAtStartPar
\sphinxcode{\sphinxupquote{runtime\_start}}
&
\sphinxAtStartPar
Date/Time of calculation start
\\
\sphinxhline
\sphinxAtStartPar
\sphinxcode{\sphinxupquote{runtime\_end}}
&
\sphinxAtStartPar
Date/Time of calculation end
\\
\sphinxhline
\sphinxAtStartPar
\sphinxcode{\sphinxupquote{out\_file}}
&
\sphinxAtStartPar
Name for ThunderBoltz stdout file
\\
\sphinxhline
\sphinxAtStartPar
\sphinxcode{\sphinxupquote{live}}
&
\sphinxAtStartPar
Run and update time series plotting GUI during simulation.
\\
\sphinxhline
\sphinxAtStartPar
\sphinxcode{\sphinxupquote{live\_rate}}
&
\sphinxAtStartPar
Run and update reaction rate plotting GUI during simulation.
\\
\sphinxhline
\sphinxAtStartPar
\sphinxcode{\sphinxupquote{ts\_fig}}
&
\sphinxAtStartPar
The figure object for the time series plot.
\\
\sphinxhline
\sphinxAtStartPar
\sphinxcode{\sphinxupquote{rate\_fig}}
&
\sphinxAtStartPar
The figure object for the rate plot
\\
\sphinxhline
\sphinxAtStartPar
\sphinxcode{\sphinxupquote{callbacks}}
&
\sphinxAtStartPar
List of functions that are called every time banner output is updated.
\\
\sphinxhline
\sphinxAtStartPar
\sphinxcode{\sphinxupquote{directory}}
&
\sphinxAtStartPar
Simulation directory
\\
\sphinxhline
\sphinxAtStartPar
\sphinxcode{\sphinxupquote{err\_stack}}
&
\sphinxAtStartPar
Recorded thunderboltz warnings read in from output files
\\
\sphinxhline
\sphinxAtStartPar
\sphinxcode{\sphinxupquote{monitor}}
&
\sphinxAtStartPar
Option to create temp file during run that causes safe exit upon deletion
\\
\sphinxbottomrule
\end{tabulary}
\sphinxtableafterendhook\par
\sphinxattableend\end{savenotes}
\subsubsection*{Methods}


\begin{savenotes}\sphinxattablestart
\sphinxthistablewithglobalstyle
\sphinxthistablewithnovlinesstyle
\centering
\begin{tabulary}{\linewidth}[t]{\X{1}{2}\X{1}{2}}
\sphinxtoprule
\sphinxtableatstartofbodyhook
\sphinxAtStartPar
{\hyperref[\detokenize{api/pytb.ThunderBoltz.add_callback:pytb.ThunderBoltz.add_callback}]{\sphinxcrossref{\sphinxcode{\sphinxupquote{add\_callback}}}}}(f)
&
\sphinxAtStartPar
Add a function to the list of functions that will be called during banner output.
\\
\sphinxhline
\sphinxAtStartPar
{\hyperref[\detokenize{api/pytb.ThunderBoltz.compile_debug:pytb.ThunderBoltz.compile_debug}]{\sphinxcrossref{\sphinxcode{\sphinxupquote{compile\_debug}}}}}()
&
\sphinxAtStartPar
Prepare all files and compile with \sphinxhyphen{}g debug flag.
\\
\sphinxhline
\sphinxAtStartPar
{\hyperref[\detokenize{api/pytb.ThunderBoltz.compile_from:pytb.ThunderBoltz.compile_from}]{\sphinxcrossref{\sphinxcode{\sphinxupquote{compile\_from}}}}}(src\_path{[}, debug{]})
&
\sphinxAtStartPar
Copy TB files from src\_path and compile in simulation directory.
\\
\sphinxhline
\sphinxAtStartPar
{\hyperref[\detokenize{api/pytb.ThunderBoltz.describe_dist:pytb.ThunderBoltz.describe_dist}]{\sphinxcrossref{\sphinxcode{\sphinxupquote{describe\_dist}}}}}({[}steps, sample\_cap{]})
&
\sphinxAtStartPar
Generate percentile and count statistics of the electron velocity / energy distribution for various time steps.
\\
\sphinxhline
\sphinxAtStartPar
{\hyperref[\detokenize{api/pytb.ThunderBoltz.get_directory:pytb.ThunderBoltz.get_directory}]{\sphinxcrossref{\sphinxcode{\sphinxupquote{get\_directory}}}}}()
&
\sphinxAtStartPar
Return the path of the current simulation.
\\
\sphinxhline
\sphinxAtStartPar
{\hyperref[\detokenize{api/pytb.ThunderBoltz.get_edfs:pytb.ThunderBoltz.get_edfs}]{\sphinxcrossref{\sphinxcode{\sphinxupquote{get\_edfs}}}}}({[}steps, sample\_cap{]})
&
\sphinxAtStartPar
Read the electron velocity distribution functions and return the component and total energy distributions within a ThunderBoltz calculation.
\\
\sphinxhline
\sphinxAtStartPar
{\hyperref[\detokenize{api/pytb.ThunderBoltz.get_etrans:pytb.ThunderBoltz.get_etrans}]{\sphinxcrossref{\sphinxcode{\sphinxupquote{get\_etrans}}}}}()
&
\sphinxAtStartPar
Return the energy weighted counts of each reaction computed for each time step.
\\
\sphinxbottomrule
\end{tabulary}
%%%%% SPLIT TABLE %%%%%
\begin{tabulary}{\linewidth}[t]{\X{1}{2}\X{1}{2}}
\sphinxtoprule
\sphinxtableatstartofbodyhook
\sphinxAtStartPar
{\hyperref[\detokenize{api/pytb.ThunderBoltz.get_particle_tables:pytb.ThunderBoltz.get_particle_tables}]{\sphinxcrossref{\sphinxcode{\sphinxupquote{get\_particle\_tables}}}}}()
&
\sphinxAtStartPar
Return the particle table data for each species in a list.
\\
\sphinxhline
\sphinxAtStartPar
{\hyperref[\detokenize{api/pytb.ThunderBoltz.get_sim_param:pytb.ThunderBoltz.get_sim_param}]{\sphinxcrossref{\sphinxcode{\sphinxupquote{get\_sim\_param}}}}}(key)
&
\sphinxAtStartPar
Return the value of a simulation parameter.
\\
\sphinxhline
\sphinxAtStartPar
{\hyperref[\detokenize{api/pytb.ThunderBoltz.get_ss_params:pytb.ThunderBoltz.get_ss_params}]{\sphinxcrossref{\sphinxcode{\sphinxupquote{get\_ss\_params}}}}}({[}ss\_func{]})
&
\sphinxAtStartPar
Get steady\sphinxhyphen{}state transport parameter values by averaging last section of time series.
\\
\sphinxhline
\sphinxAtStartPar
{\hyperref[\detokenize{api/pytb.ThunderBoltz.get_timeseries:pytb.ThunderBoltz.get_timeseries}]{\sphinxcrossref{\sphinxcode{\sphinxupquote{get\_timeseries}}}}}()
&
\sphinxAtStartPar
Collect the relevant time series data from a ThunderBoltz simulation directory and add input parameter columns.
\\
\sphinxhline
\sphinxAtStartPar
{\hyperref[\detokenize{api/pytb.ThunderBoltz.get_vdfs:pytb.ThunderBoltz.get_vdfs}]{\sphinxcrossref{\sphinxcode{\sphinxupquote{get\_vdfs}}}}}({[}steps, sample\_cap, particle\_type, v{]})
&
\sphinxAtStartPar
Read the electron velocities arrays within a ThunderBoltz calculation.
\\
\sphinxhline
\sphinxAtStartPar
{\hyperref[\detokenize{api/pytb.ThunderBoltz.plot_cs:pytb.ThunderBoltz.plot_cs}]{\sphinxcrossref{\sphinxcode{\sphinxupquote{plot\_cs}}}}}({[}ax, legend, vsig, thresholds, save{]})
&
\sphinxAtStartPar
Plot the cross sections models.
\\
\sphinxhline
\sphinxAtStartPar
{\hyperref[\detokenize{api/pytb.ThunderBoltz.plot_edf_comps:pytb.ThunderBoltz.plot_edf_comps}]{\sphinxcrossref{\sphinxcode{\sphinxupquote{plot\_edf\_comps}}}}}({[}steps, sample\_cap, bins, ...{]})
&
\sphinxAtStartPar
Plot the directional components of the energy distribution function.
\\
\sphinxhline
\sphinxAtStartPar
{\hyperref[\detokenize{api/pytb.ThunderBoltz.plot_edfs:pytb.ThunderBoltz.plot_edfs}]{\sphinxcrossref{\sphinxcode{\sphinxupquote{plot\_edfs}}}}}({[}steps, sample\_cap, bins, ...{]})
&
\sphinxAtStartPar
Plot the electron total energy distribution function, optionally include the provided cross sections for comparison.
\\
\sphinxhline
\sphinxAtStartPar
{\hyperref[\detokenize{api/pytb.ThunderBoltz.plot_rates:pytb.ThunderBoltz.plot_rates}]{\sphinxcrossref{\sphinxcode{\sphinxupquote{plot\_rates}}}}}({[}save, stamp, v, update{]})
&
\sphinxAtStartPar
Create a diagnostic plot of ThunderBoltz time series data.
\\
\sphinxhline
\sphinxAtStartPar
{\hyperref[\detokenize{api/pytb.ThunderBoltz.plot_timeseries:pytb.ThunderBoltz.plot_timeseries}]{\sphinxcrossref{\sphinxcode{\sphinxupquote{plot\_timeseries}}}}}({[}series, save, stamp, v, update{]})
&
\sphinxAtStartPar
Create a diagnostic plot of ThunderBoltz time series data.
\\
\sphinxhline
\sphinxAtStartPar
{\hyperref[\detokenize{api/pytb.ThunderBoltz.plot_vdfs:pytb.ThunderBoltz.plot_vdfs}]{\sphinxcrossref{\sphinxcode{\sphinxupquote{plot\_vdfs}}}}}({[}steps, save, bins, sample\_cap{]})
&
\sphinxAtStartPar
Plot the joint distribution heat map between the x\sphinxhyphen{}y and x\sphinxhyphen{}z velocities.
\\
\sphinxhline
\sphinxAtStartPar
{\hyperref[\detokenize{api/pytb.ThunderBoltz.read:pytb.ThunderBoltz.read}]{\sphinxcrossref{\sphinxcode{\sphinxupquote{read}}}}}({[}directory, read\_input, read\_cs\_data, only{]})
&
\sphinxAtStartPar
Read the simulation directory of a ThunderBoltz run, possibly all of its input and output files.
\\
\sphinxhline
\sphinxAtStartPar
{\hyperref[\detokenize{api/pytb.ThunderBoltz.read_log:pytb.ThunderBoltz.read_log}]{\sphinxcrossref{\sphinxcode{\sphinxupquote{read\_log}}}}}(logfile)
&
\sphinxAtStartPar
Read json file from simulation directory.
\\
\sphinxhline
\sphinxAtStartPar
{\hyperref[\detokenize{api/pytb.ThunderBoltz.read_particle_table:pytb.ThunderBoltz.read_particle_table}]{\sphinxcrossref{\sphinxcode{\sphinxupquote{read\_particle\_table}}}}}(i)
&
\sphinxAtStartPar
Read species specific output data, including density, velocity, displacement, energy, and temperature.
\\
\sphinxhline
\sphinxAtStartPar
{\hyperref[\detokenize{api/pytb.ThunderBoltz.read_stdout:pytb.ThunderBoltz.read_stdout}]{\sphinxcrossref{\sphinxcode{\sphinxupquote{read\_stdout}}}}}(fname)
&
\sphinxAtStartPar
Read the banner output data.
\\
\sphinxhline
\sphinxAtStartPar
{\hyperref[\detokenize{api/pytb.ThunderBoltz.read_tb_params:pytb.ThunderBoltz.read_tb_params}]{\sphinxcrossref{\sphinxcode{\sphinxupquote{read\_tb\_params}}}}}(fname{[}, ignore{]})
&
\sphinxAtStartPar
Takes file name of an input deck, updates the simulation parameters and returns the simulation parameters which were read from the file \sphinxcode{\sphinxupquote{fname}}.
\\
\sphinxhline
\sphinxAtStartPar
{\hyperref[\detokenize{api/pytb.ThunderBoltz.reset:pytb.ThunderBoltz.reset}]{\sphinxcrossref{\sphinxcode{\sphinxupquote{reset}}}}}()
&
\sphinxAtStartPar
Reset output data for a new run.
\\
\sphinxhline
\sphinxAtStartPar
{\hyperref[\detokenize{api/pytb.ThunderBoltz.run:pytb.ThunderBoltz.run}]{\sphinxcrossref{\sphinxcode{\sphinxupquote{run}}}}}({[}src\_path, bin\_path, out\_file, monitor, ...{]})
&
\sphinxAtStartPar
Execute with the current parameters in the simulation directory.
\\
\sphinxhline
\sphinxAtStartPar
{\hyperref[\detokenize{api/pytb.ThunderBoltz.set_:pytb.ThunderBoltz.set_}]{\sphinxcrossref{\sphinxcode{\sphinxupquote{set\_}}}}}(**p)
&
\sphinxAtStartPar
Update parameters, call appropriate functions ensuring input parameters are self\sphinxhyphen{}consistent.
\\
\sphinxhline
\sphinxAtStartPar
{\hyperref[\detokenize{api/pytb.ThunderBoltz.set_fixed_tracking:pytb.ThunderBoltz.set_fixed_tracking}]{\sphinxcrossref{\sphinxcode{\sphinxupquote{set\_fixed\_tracking}}}}}()
&
\sphinxAtStartPar
Change all reaction species indices of differing reactant values to be between only particle 0 and 1 (e.g.
\\
\sphinxhline
\sphinxAtStartPar
{\hyperref[\detokenize{api/pytb.ThunderBoltz.set_ts_plot_params:pytb.ThunderBoltz.set_ts_plot_params}]{\sphinxcrossref{\sphinxcode{\sphinxupquote{set\_ts\_plot\_params}}}}}(params)
&
\sphinxAtStartPar
Set the default series plotted by {\hyperref[\detokenize{api/pytb.ThunderBoltz.plot_timeseries:pytb.ThunderBoltz.plot_timeseries}]{\sphinxcrossref{\sphinxcode{\sphinxupquote{plot\_timeseries()}}}}}
\\
\sphinxhline
\sphinxAtStartPar
{\hyperref[\detokenize{api/pytb.ThunderBoltz.to_pickleable:pytb.ThunderBoltz.to_pickleable}]{\sphinxcrossref{\sphinxcode{\sphinxupquote{to\_pickleable}}}}}()
&
\sphinxAtStartPar
Return a picklable version of this object.
\\
\sphinxhline
\sphinxAtStartPar
{\hyperref[\detokenize{api/pytb.ThunderBoltz.write_input:pytb.ThunderBoltz.write_input}]{\sphinxcrossref{\sphinxcode{\sphinxupquote{write\_input}}}}}(directory)
&
\sphinxAtStartPar
Write all the input files into a directory with the current settings.
\\
\sphinxbottomrule
\end{tabulary}
\sphinxtableafterendhook\par
\sphinxattableend\end{savenotes}

\sphinxstepscope


\subsubsection{pytb.ThunderBoltz.add\_callback}
\label{\detokenize{api/pytb.ThunderBoltz.add_callback:pytb-thunderboltz-add-callback}}\label{\detokenize{api/pytb.ThunderBoltz.add_callback::doc}}\index{add\_callback() (pytb.ThunderBoltz method)@\spxentry{add\_callback()}\spxextra{pytb.ThunderBoltz method}}

\begin{fulllineitems}
\phantomsection\label{\detokenize{api/pytb.ThunderBoltz.add_callback:pytb.ThunderBoltz.add_callback}}
\pysigstartsignatures
\pysiglinewithargsret{\sphinxcode{\sphinxupquote{ThunderBoltz.}}\sphinxbfcode{\sphinxupquote{add\_callback}}}{\sphinxparam{\DUrole{n,n}{f}}}{}
\pysigstopsignatures
\sphinxAtStartPar
Add a function to the list of functions that will be
called during banner output.
\begin{quote}\begin{description}
\sphinxlineitem{Parameters}
\sphinxAtStartPar
\sphinxstyleliteralstrong{\sphinxupquote{f}} (\sphinxstyleliteralemphasis{\sphinxupquote{callable}}\sphinxstyleliteralemphasis{\sphinxupquote{{[}}}\sphinxstyleliteralemphasis{\sphinxupquote{,}}\sphinxstyleliteralemphasis{\sphinxupquote{{]}}}) \textendash{} The function that will be called. It
should accept no arguments and return no arguments.

\end{description}\end{quote}

\end{fulllineitems}


\sphinxstepscope


\subsubsection{pytb.ThunderBoltz.compile\_debug}
\label{\detokenize{api/pytb.ThunderBoltz.compile_debug:pytb-thunderboltz-compile-debug}}\label{\detokenize{api/pytb.ThunderBoltz.compile_debug::doc}}\index{compile\_debug() (pytb.ThunderBoltz method)@\spxentry{compile\_debug()}\spxextra{pytb.ThunderBoltz method}}

\begin{fulllineitems}
\phantomsection\label{\detokenize{api/pytb.ThunderBoltz.compile_debug:pytb.ThunderBoltz.compile_debug}}
\pysigstartsignatures
\pysiglinewithargsret{\sphinxcode{\sphinxupquote{ThunderBoltz.}}\sphinxbfcode{\sphinxupquote{compile\_debug}}}{}{}
\pysigstopsignatures
\sphinxAtStartPar
Prepare all files and compile with \sphinxhyphen{}g debug flag.

\end{fulllineitems}


\sphinxstepscope


\subsubsection{pytb.ThunderBoltz.compile\_from}
\label{\detokenize{api/pytb.ThunderBoltz.compile_from:pytb-thunderboltz-compile-from}}\label{\detokenize{api/pytb.ThunderBoltz.compile_from::doc}}\index{compile\_from() (pytb.ThunderBoltz method)@\spxentry{compile\_from()}\spxextra{pytb.ThunderBoltz method}}

\begin{fulllineitems}
\phantomsection\label{\detokenize{api/pytb.ThunderBoltz.compile_from:pytb.ThunderBoltz.compile_from}}
\pysigstartsignatures
\pysiglinewithargsret{\sphinxcode{\sphinxupquote{ThunderBoltz.}}\sphinxbfcode{\sphinxupquote{compile\_from}}}{\sphinxparam{\DUrole{n,n}{src\_path}}, \sphinxparam{\DUrole{n,n}{debug}\DUrole{o,o}{=}\DUrole{default_value}{False}}}{}
\pysigstopsignatures
\sphinxAtStartPar
Copy TB files from src\_path and compile
in simulation directory.
\begin{quote}\begin{description}
\sphinxlineitem{Parameters}\begin{itemize}
\item {} 
\sphinxAtStartPar
\sphinxstyleliteralstrong{\sphinxupquote{src\_path}} (\sphinxhref{https://docs.python.org/3/library/stdtypes.html\#str}{\sphinxstyleliteralemphasis{\sphinxupquote{str}}}) \textendash{} The location of the source files
to compile from.

\item {} 
\sphinxAtStartPar
\sphinxstyleliteralstrong{\sphinxupquote{debug}} (\sphinxhref{https://docs.python.org/3/library/functions.html\#bool}{\sphinxstyleliteralemphasis{\sphinxupquote{bool}}}) \textendash{} If \sphinxcode{\sphinxupquote{True}}, compile C++ with the \sphinxcode{\sphinxupquote{\sphinxhyphen{}g}}
debug flag.

\end{itemize}

\sphinxlineitem{Raises}
\sphinxAtStartPar
\sphinxhref{https://docs.python.org/3/library/exceptions.html\#RuntimeError}{\sphinxstyleliteralstrong{\sphinxupquote{RuntimeError}}} \textendash{} if there is a compilation issue.

\end{description}\end{quote}

\end{fulllineitems}


\sphinxstepscope


\subsubsection{pytb.ThunderBoltz.describe\_dist}
\label{\detokenize{api/pytb.ThunderBoltz.describe_dist:pytb-thunderboltz-describe-dist}}\label{\detokenize{api/pytb.ThunderBoltz.describe_dist::doc}}\index{describe\_dist() (pytb.ThunderBoltz method)@\spxentry{describe\_dist()}\spxextra{pytb.ThunderBoltz method}}

\begin{fulllineitems}
\phantomsection\label{\detokenize{api/pytb.ThunderBoltz.describe_dist:pytb.ThunderBoltz.describe_dist}}
\pysigstartsignatures
\pysiglinewithargsret{\sphinxcode{\sphinxupquote{ThunderBoltz.}}\sphinxbfcode{\sphinxupquote{describe\_dist}}}{\sphinxparam{\DUrole{n,n}{steps}\DUrole{o,o}{=}\DUrole{default_value}{\textquotesingle{}last\textquotesingle{}}}, \sphinxparam{\DUrole{n,n}{sample\_cap}\DUrole{o,o}{=}\DUrole{default_value}{500000}}}{}
\pysigstopsignatures
\sphinxAtStartPar
Generate percentile and count statistics of the electron
velocity / energy distribution for various time steps.
\begin{quote}\begin{description}
\sphinxlineitem{Parameters}\begin{itemize}
\item {} 
\sphinxAtStartPar
\sphinxstyleliteralstrong{\sphinxupquote{steps}} (\sphinxhref{https://docs.python.org/3/library/stdtypes.html\#str}{\sphinxstyleliteralemphasis{\sphinxupquote{str}}}\sphinxstyleliteralemphasis{\sphinxupquote{, }}\sphinxhref{https://docs.python.org/3/library/stdtypes.html\#list}{\sphinxstyleliteralemphasis{\sphinxupquote{list}}}\sphinxstyleliteralemphasis{\sphinxupquote{{[}}}\sphinxhref{https://docs.python.org/3/library/functions.html\#int}{\sphinxstyleliteralemphasis{\sphinxupquote{int}}}\sphinxstyleliteralemphasis{\sphinxupquote{{]}}}\sphinxstyleliteralemphasis{\sphinxupquote{, or }}\sphinxhref{https://docs.python.org/3/library/functions.html\#int}{\sphinxstyleliteralemphasis{\sphinxupquote{int}}}) \textendash{} 
\sphinxAtStartPar
Options for which time steps to
read:
\begin{itemize}
\item {} 
\sphinxAtStartPar
\sphinxcode{\sphinxupquote{"last"}}: Only read the VDF of the last time step

\item {} 
\sphinxAtStartPar
\sphinxcode{\sphinxupquote{"first"}}: Only read the VDF of the first time step

\item {} 
\sphinxAtStartPar
\sphinxcode{\sphinxupquote{"all"}}: Read a separate VDF for each time step.

\item {} 
\sphinxAtStartPar
\sphinxcode{\sphinxupquote{list{[}int{]}}}: Read VDF for each time step included in list.

\item {} 
\sphinxAtStartPar
\sphinxcode{\sphinxupquote{int}}: read VDF at one specific time step.

\end{itemize}


\item {} 
\sphinxAtStartPar
\sphinxstyleliteralstrong{\sphinxupquote{sample\_cap}} (\sphinxhref{https://docs.python.org/3/library/functions.html\#int}{\sphinxstyleliteralemphasis{\sphinxupquote{int}}}) \textendash{} Limit the number of samples read from the dump
file for very large files. Default is 500000. If bool(sample\_cap)
evaluates to \sphinxcode{\sphinxupquote{False}}, then no cap will be imposed.

\end{itemize}

\sphinxlineitem{Returns}
\sphinxAtStartPar
A table with statistical descriptions of
the velocity and energy distributions.

\sphinxlineitem{Return type}
\sphinxAtStartPar
\sphinxhref{http://pandas.pydata.org/pandas-docs/dev/reference/api/pandas.DataFrame.html\#pandas.DataFrame}{\sphinxcode{\sphinxupquote{pandas.DataFrame}}}

\end{description}\end{quote}

\end{fulllineitems}


\sphinxstepscope


\subsubsection{pytb.ThunderBoltz.get\_directory}
\label{\detokenize{api/pytb.ThunderBoltz.get_directory:pytb-thunderboltz-get-directory}}\label{\detokenize{api/pytb.ThunderBoltz.get_directory::doc}}\index{get\_directory() (pytb.ThunderBoltz method)@\spxentry{get\_directory()}\spxextra{pytb.ThunderBoltz method}}

\begin{fulllineitems}
\phantomsection\label{\detokenize{api/pytb.ThunderBoltz.get_directory:pytb.ThunderBoltz.get_directory}}
\pysigstartsignatures
\pysiglinewithargsret{\sphinxcode{\sphinxupquote{ThunderBoltz.}}\sphinxbfcode{\sphinxupquote{get\_directory}}}{}{}
\pysigstopsignatures
\sphinxAtStartPar
Return the path of the current simulation.

\end{fulllineitems}


\sphinxstepscope


\subsubsection{pytb.ThunderBoltz.get\_edfs}
\label{\detokenize{api/pytb.ThunderBoltz.get_edfs:pytb-thunderboltz-get-edfs}}\label{\detokenize{api/pytb.ThunderBoltz.get_edfs::doc}}\index{get\_edfs() (pytb.ThunderBoltz method)@\spxentry{get\_edfs()}\spxextra{pytb.ThunderBoltz method}}

\begin{fulllineitems}
\phantomsection\label{\detokenize{api/pytb.ThunderBoltz.get_edfs:pytb.ThunderBoltz.get_edfs}}
\pysigstartsignatures
\pysiglinewithargsret{\sphinxcode{\sphinxupquote{ThunderBoltz.}}\sphinxbfcode{\sphinxupquote{get\_edfs}}}{\sphinxparam{\DUrole{n,n}{steps}\DUrole{o,o}{=}\DUrole{default_value}{\textquotesingle{}last\textquotesingle{}}}, \sphinxparam{\DUrole{n,n}{sample\_cap}\DUrole{o,o}{=}\DUrole{default_value}{500000}}}{}
\pysigstopsignatures
\sphinxAtStartPar
Read the electron velocity distribution functions and return
the component and total energy distributions within a
ThunderBoltz calculation. Energy units are in eV. Invokes
{\hyperref[\detokenize{api/pytb.ThunderBoltz.get_vdfs:pytb.ThunderBoltz.get_vdfs}]{\sphinxcrossref{\sphinxcode{\sphinxupquote{get\_vdfs()}}}}}.
\begin{quote}\begin{description}
\sphinxlineitem{Parameters}\begin{itemize}
\item {} 
\sphinxAtStartPar
\sphinxstyleliteralstrong{\sphinxupquote{steps}} (\sphinxhref{https://docs.python.org/3/library/stdtypes.html\#str}{\sphinxstyleliteralemphasis{\sphinxupquote{str}}}\sphinxstyleliteralemphasis{\sphinxupquote{, }}\sphinxhref{https://docs.python.org/3/library/stdtypes.html\#list}{\sphinxstyleliteralemphasis{\sphinxupquote{list}}}\sphinxstyleliteralemphasis{\sphinxupquote{{[}}}\sphinxhref{https://docs.python.org/3/library/functions.html\#int}{\sphinxstyleliteralemphasis{\sphinxupquote{int}}}\sphinxstyleliteralemphasis{\sphinxupquote{{]}}}\sphinxstyleliteralemphasis{\sphinxupquote{, or }}\sphinxhref{https://docs.python.org/3/library/functions.html\#int}{\sphinxstyleliteralemphasis{\sphinxupquote{int}}}) \textendash{} 
\sphinxAtStartPar
Options for which time steps to
read:
\begin{itemize}
\item {} 
\sphinxAtStartPar
\sphinxcode{\sphinxupquote{"last"}}: Only read the VDF of the last time step

\item {} 
\sphinxAtStartPar
\sphinxcode{\sphinxupquote{"first"}}: Only read the VDF of the first time step

\item {} 
\sphinxAtStartPar
\sphinxcode{\sphinxupquote{"all"}}: Read a separate VDF for each time step.

\item {} 
\sphinxAtStartPar
\sphinxcode{\sphinxupquote{list{[}int{]}}}: Read VDF for each time step included in list.

\item {} 
\sphinxAtStartPar
\sphinxcode{\sphinxupquote{int}}: read VDF at one specific time step.

\end{itemize}


\item {} 
\sphinxAtStartPar
\sphinxstyleliteralstrong{\sphinxupquote{sample\_cap}} (\sphinxhref{https://docs.python.org/3/library/functions.html\#int}{\sphinxstyleliteralemphasis{\sphinxupquote{int}}}) \textendash{} Limit the number of samples read from the dump
file for very large files. Default is 500000. If bool(sample\_cap)
evaluates to \sphinxcode{\sphinxupquote{False}}, then no cap will be imposed.

\end{itemize}

\sphinxlineitem{Returns}
\sphinxAtStartPar
\begin{description}
\sphinxlineitem{A table with the signed and unsigned}
\sphinxAtStartPar
energy components of each particle.

\end{description}


\sphinxlineitem{Return type}
\sphinxAtStartPar
\sphinxhref{http://pandas.pydata.org/pandas-docs/dev/reference/api/pandas.DataFrame.html\#pandas.DataFrame}{\sphinxcode{\sphinxupquote{pandas.DataFrame}}}

\end{description}\end{quote}

\end{fulllineitems}


\sphinxstepscope


\subsubsection{pytb.ThunderBoltz.get\_etrans}
\label{\detokenize{api/pytb.ThunderBoltz.get_etrans:pytb-thunderboltz-get-etrans}}\label{\detokenize{api/pytb.ThunderBoltz.get_etrans::doc}}\index{get\_etrans() (pytb.ThunderBoltz method)@\spxentry{get\_etrans()}\spxextra{pytb.ThunderBoltz method}}

\begin{fulllineitems}
\phantomsection\label{\detokenize{api/pytb.ThunderBoltz.get_etrans:pytb.ThunderBoltz.get_etrans}}
\pysigstartsignatures
\pysiglinewithargsret{\sphinxcode{\sphinxupquote{ThunderBoltz.}}\sphinxbfcode{\sphinxupquote{get\_etrans}}}{}{}
\pysigstopsignatures
\sphinxAtStartPar
Return the energy weighted counts of each reaction computed
for each time step.
\begin{quote}\begin{description}
\sphinxlineitem{Returns}
\sphinxAtStartPar
A table of each process and the
energy transfer through that channel as a proportion
to the total energy transfer.

\sphinxlineitem{Return type}
\sphinxAtStartPar
\sphinxhref{http://pandas.pydata.org/pandas-docs/dev/reference/api/pandas.DataFrame.html\#pandas.DataFrame}{\sphinxcode{\sphinxupquote{pandas.DataFrame}}}

\end{description}\end{quote}

\end{fulllineitems}


\sphinxstepscope


\subsubsection{pytb.ThunderBoltz.get\_particle\_tables}
\label{\detokenize{api/pytb.ThunderBoltz.get_particle_tables:pytb-thunderboltz-get-particle-tables}}\label{\detokenize{api/pytb.ThunderBoltz.get_particle_tables::doc}}\index{get\_particle\_tables() (pytb.ThunderBoltz method)@\spxentry{get\_particle\_tables()}\spxextra{pytb.ThunderBoltz method}}

\begin{fulllineitems}
\phantomsection\label{\detokenize{api/pytb.ThunderBoltz.get_particle_tables:pytb.ThunderBoltz.get_particle_tables}}
\pysigstartsignatures
\pysiglinewithargsret{\sphinxcode{\sphinxupquote{ThunderBoltz.}}\sphinxbfcode{\sphinxupquote{get\_particle\_tables}}}{}{}
\pysigstopsignatures
\sphinxAtStartPar
Return the particle table data for each species
in a list.
\begin{quote}\begin{description}
\sphinxlineitem{Returns}
\sphinxAtStartPar
The list of particle
table data. Each table with have columns with keywords
matching the attributes of
{\hyperref[\detokenize{api/pytb.parameters.ParticleParameters:pytb.parameters.ParticleParameters}]{\sphinxcrossref{\sphinxcode{\sphinxupquote{ParticleParameters}}}}}.

\sphinxlineitem{Return type}
\sphinxAtStartPar
list{[} \sphinxhref{http://pandas.pydata.org/pandas-docs/dev/reference/api/pandas.DataFrame.html\#pandas.DataFrame}{\sphinxcode{\sphinxupquote{pandas.DataFrame}}}{]}

\end{description}\end{quote}

\end{fulllineitems}


\sphinxstepscope


\subsubsection{pytb.ThunderBoltz.get\_sim\_param}
\label{\detokenize{api/pytb.ThunderBoltz.get_sim_param:pytb-thunderboltz-get-sim-param}}\label{\detokenize{api/pytb.ThunderBoltz.get_sim_param::doc}}\index{get\_sim\_param() (pytb.ThunderBoltz method)@\spxentry{get\_sim\_param()}\spxextra{pytb.ThunderBoltz method}}

\begin{fulllineitems}
\phantomsection\label{\detokenize{api/pytb.ThunderBoltz.get_sim_param:pytb.ThunderBoltz.get_sim_param}}
\pysigstartsignatures
\pysiglinewithargsret{\sphinxcode{\sphinxupquote{ThunderBoltz.}}\sphinxbfcode{\sphinxupquote{get\_sim\_param}}}{\sphinxparam{\DUrole{n,n}{key}}}{}
\pysigstopsignatures
\sphinxAtStartPar
Return the value of a simulation parameter.
\begin{quote}\begin{description}
\sphinxlineitem{Raises}
\sphinxAtStartPar
\sphinxhref{https://docs.python.org/3/library/exceptions.html\#IndexError}{\sphinxstyleliteralstrong{\sphinxupquote{IndexError}}} \textendash{} if the parameter is not set.

\end{description}\end{quote}

\end{fulllineitems}


\sphinxstepscope


\subsubsection{pytb.ThunderBoltz.get\_ss\_params}
\label{\detokenize{api/pytb.ThunderBoltz.get_ss_params:pytb-thunderboltz-get-ss-params}}\label{\detokenize{api/pytb.ThunderBoltz.get_ss_params::doc}}\index{get\_ss\_params() (pytb.ThunderBoltz method)@\spxentry{get\_ss\_params()}\spxextra{pytb.ThunderBoltz method}}

\begin{fulllineitems}
\phantomsection\label{\detokenize{api/pytb.ThunderBoltz.get_ss_params:pytb.ThunderBoltz.get_ss_params}}
\pysigstartsignatures
\pysiglinewithargsret{\sphinxcode{\sphinxupquote{ThunderBoltz.}}\sphinxbfcode{\sphinxupquote{get\_ss\_params}}}{\sphinxparam{\DUrole{n,n}{ss\_func}\DUrole{o,o}{=}\DUrole{default_value}{None}}}{}
\pysigstopsignatures
\sphinxAtStartPar
Get steady\sphinxhyphen{}state transport parameter values by averaging last section
of time series. By default, the last fourth of the available data is
considered to be steady\sphinxhyphen{}state. Standard deviations over this interval
will be computed for each parameter in a new column with a “\_std”
suffix added to the column name.
\begin{quote}\begin{description}
\sphinxlineitem{Parameters}
\sphinxAtStartPar
\sphinxstyleliteralstrong{\sphinxupquote{ss\_func}} (callable{[}\sphinxhref{http://pandas.pydata.org/pandas-docs/dev/reference/api/pandas.DataFrame.html\#pandas.DataFrame}{\sphinxcode{\sphinxupquote{pandas.DataFrame}}}, \sphinxhref{http://pandas.pydata.org/pandas-docs/dev/reference/api/pandas.DataFrame.html\#pandas.DataFrame}{\sphinxcode{\sphinxupquote{pandas.DataFrame}}}{]}) \textendash{} A function that takes in the numerical time series data and returns a
truncated set of time series data that is considered to be at
steady state.

\sphinxlineitem{Returns}
\sphinxAtStartPar
The aggregated steady\sphinxhyphen{}state data for
each output parameter along with columns specifying the input
parameters.

\sphinxlineitem{Return type}
\sphinxAtStartPar
\sphinxhref{http://pandas.pydata.org/pandas-docs/dev/reference/api/pandas.DataFrame.html\#pandas.DataFrame}{\sphinxcode{\sphinxupquote{pandas.DataFrame}}}

\sphinxlineitem{Raises}
\sphinxAtStartPar
\sphinxhref{https://docs.python.org/3/library/exceptions.html\#RuntimeWarning}{\sphinxstyleliteralstrong{\sphinxupquote{RuntimeWarning}}} \textendash{} if not enough steps are available to compute steady
    state statistics.

\end{description}\end{quote}

\begin{sphinxadmonition}{warning}{Warning:}
\sphinxAtStartPar
Currently, the last quarter of the time series data is assumed to be
in steady\sphinxhyphen{}state by default when calculating these steady\sphinxhyphen{}state parameters.
One can verify that this is true by viewing the figures produced by
{\hyperref[\detokenize{api/pytb.ThunderBoltz.plot_timeseries:pytb.ThunderBoltz.plot_timeseries}]{\sphinxcrossref{\sphinxcode{\sphinxupquote{plot\_timeseries()}}}}}. Otherwise, one may run the simulation for longer,
or provide the appropriate criteria via \sphinxcode{\sphinxupquote{ss\_func}}.
\end{sphinxadmonition}

\end{fulllineitems}


\sphinxstepscope


\subsubsection{pytb.ThunderBoltz.get\_timeseries}
\label{\detokenize{api/pytb.ThunderBoltz.get_timeseries:pytb-thunderboltz-get-timeseries}}\label{\detokenize{api/pytb.ThunderBoltz.get_timeseries::doc}}\index{get\_timeseries() (pytb.ThunderBoltz method)@\spxentry{get\_timeseries()}\spxextra{pytb.ThunderBoltz method}}

\begin{fulllineitems}
\phantomsection\label{\detokenize{api/pytb.ThunderBoltz.get_timeseries:pytb.ThunderBoltz.get_timeseries}}
\pysigstartsignatures
\pysiglinewithargsret{\sphinxcode{\sphinxupquote{ThunderBoltz.}}\sphinxbfcode{\sphinxupquote{get\_timeseries}}}{}{}
\pysigstopsignatures
\sphinxAtStartPar
Collect the relevant time series data from a ThunderBoltz simulation
directory and add input parameter columns.
\begin{quote}\begin{description}
\sphinxlineitem{Returns}
\sphinxAtStartPar
The table of time series data.

\sphinxlineitem{Return type}
\sphinxAtStartPar
\sphinxhref{http://pandas.pydata.org/pandas-docs/dev/reference/api/pandas.DataFrame.html\#pandas.DataFrame}{\sphinxcode{\sphinxupquote{pandas.DataFrame}}}

\end{description}\end{quote}

\end{fulllineitems}


\sphinxstepscope


\subsubsection{pytb.ThunderBoltz.get\_vdfs}
\label{\detokenize{api/pytb.ThunderBoltz.get_vdfs:pytb-thunderboltz-get-vdfs}}\label{\detokenize{api/pytb.ThunderBoltz.get_vdfs::doc}}\index{get\_vdfs() (pytb.ThunderBoltz method)@\spxentry{get\_vdfs()}\spxextra{pytb.ThunderBoltz method}}

\begin{fulllineitems}
\phantomsection\label{\detokenize{api/pytb.ThunderBoltz.get_vdfs:pytb.ThunderBoltz.get_vdfs}}
\pysigstartsignatures
\pysiglinewithargsret{\sphinxcode{\sphinxupquote{ThunderBoltz.}}\sphinxbfcode{\sphinxupquote{get\_vdfs}}}{\sphinxparam{\DUrole{n,n}{steps}\DUrole{o,o}{=}\DUrole{default_value}{\textquotesingle{}last\textquotesingle{}}}, \sphinxparam{\DUrole{n,n}{sample\_cap}\DUrole{o,o}{=}\DUrole{default_value}{500000}}, \sphinxparam{\DUrole{n,n}{particle\_type}\DUrole{o,o}{=}\DUrole{default_value}{0}}, \sphinxparam{\DUrole{n,n}{v}\DUrole{o,o}{=}\DUrole{default_value}{0}}}{}
\pysigstopsignatures
\sphinxAtStartPar
Read the electron velocities arrays within a ThunderBoltz calculation.
Velocity units are in m/s. If velocity dump files are found
corresponding to \sphinxcode{\sphinxupquote{steps}}, update \sphinxcode{\sphinxupquote{vdfs}}.
\begin{quote}\begin{description}
\sphinxlineitem{Parameters}\begin{itemize}
\item {} 
\sphinxAtStartPar
\sphinxstyleliteralstrong{\sphinxupquote{steps}} (\sphinxhref{https://docs.python.org/3/library/stdtypes.html\#str}{\sphinxstyleliteralemphasis{\sphinxupquote{str}}}\sphinxstyleliteralemphasis{\sphinxupquote{, }}\sphinxhref{https://docs.python.org/3/library/stdtypes.html\#list}{\sphinxstyleliteralemphasis{\sphinxupquote{list}}}\sphinxstyleliteralemphasis{\sphinxupquote{{[}}}\sphinxhref{https://docs.python.org/3/library/functions.html\#int}{\sphinxstyleliteralemphasis{\sphinxupquote{int}}}\sphinxstyleliteralemphasis{\sphinxupquote{{]}}}\sphinxstyleliteralemphasis{\sphinxupquote{, or }}\sphinxhref{https://docs.python.org/3/library/functions.html\#int}{\sphinxstyleliteralemphasis{\sphinxupquote{int}}}) \textendash{} 
\sphinxAtStartPar
Options for which time steps to
read:
\begin{itemize}
\item {} 
\sphinxAtStartPar
\sphinxcode{\sphinxupquote{"last"}}: Only read the VDF of the last time step

\item {} 
\sphinxAtStartPar
\sphinxcode{\sphinxupquote{"first"}}: Only read the VDF of the first time step

\item {} 
\sphinxAtStartPar
\sphinxcode{\sphinxupquote{"all"}}: Read a separate VDF for each time step.

\item {} 
\sphinxAtStartPar
\sphinxcode{\sphinxupquote{list{[}int{]}}}: Read VDF for each time step included in list.

\item {} 
\sphinxAtStartPar
\sphinxcode{\sphinxupquote{int}}: read VDF at one specific time step.

\end{itemize}


\item {} 
\sphinxAtStartPar
\sphinxstyleliteralstrong{\sphinxupquote{sample\_cap}} (\sphinxhref{https://docs.python.org/3/library/functions.html\#int}{\sphinxstyleliteralemphasis{\sphinxupquote{int}}}) \textendash{} Limit the number of samples read from the dump
file for very large files. Default is 500000. If bool(sample\_cap)
evaluates to \sphinxcode{\sphinxupquote{False}}, then no cap will be imposed.

\item {} 
\sphinxAtStartPar
\sphinxstyleliteralstrong{\sphinxupquote{particle\_type}} (\sphinxhref{https://docs.python.org/3/library/stdtypes.html\#str}{\sphinxstyleliteralemphasis{\sphinxupquote{str}}}\sphinxstyleliteralemphasis{\sphinxupquote{, }}\sphinxhref{https://docs.python.org/3/library/stdtypes.html\#list}{\sphinxstyleliteralemphasis{\sphinxupquote{list}}}\sphinxstyleliteralemphasis{\sphinxupquote{{[}}}\sphinxhref{https://docs.python.org/3/library/functions.html\#int}{\sphinxstyleliteralemphasis{\sphinxupquote{int}}}\sphinxstyleliteralemphasis{\sphinxupquote{{]}}}\sphinxstyleliteralemphasis{\sphinxupquote{, or }}\sphinxhref{https://docs.python.org/3/library/functions.html\#int}{\sphinxstyleliteralemphasis{\sphinxupquote{int}}}) \textendash{} 
\sphinxAtStartPar
Specify which kinds of species data
should be read from.
\begin{itemize}
\item {} 
\sphinxAtStartPar
\sphinxcode{\sphinxupquote{int}}: The particle type to read. Default is \sphinxcode{\sphinxupquote{0}}.

\item {} 
\sphinxAtStartPar
\sphinxcode{\sphinxupquote{list{[}int{]}}}: A set of particle types to read.

\item {} 
\sphinxAtStartPar
\sphinxcode{\sphinxupquote{"all"}}: Read all particle types.

\end{itemize}


\item {} 
\sphinxAtStartPar
\sphinxstyleliteralstrong{\sphinxupquote{v}} (\sphinxhref{https://docs.python.org/3/library/functions.html\#int}{\sphinxstyleliteralemphasis{\sphinxupquote{int}}}) \textendash{} Verbosity \textendash{} 0: silent, 1: print file paths before reading.

\end{itemize}

\sphinxlineitem{Returns}
\sphinxAtStartPar
The particle velocity dump data.

\sphinxlineitem{Return type}
\sphinxAtStartPar
\sphinxhref{http://pandas.pydata.org/pandas-docs/dev/reference/api/pandas.DataFrame.html\#pandas.DataFrame}{\sphinxcode{\sphinxupquote{pandas.DataFrame}}}

\end{description}\end{quote}

\begin{sphinxadmonition}{warning}{Warning:}
\sphinxAtStartPar
Large files are truncated to the first \sphinxcode{\sphinxupquote{sample\_cap}} lines.
It is assumed that the particle ordering in the dump files
is not correlated with any velocity statistics, but
this may not be the case when \sphinxcode{\sphinxupquote{egen}} is on. In that case, ensure
the entire velocity dump file is being read.
\end{sphinxadmonition}

\end{fulllineitems}


\sphinxstepscope


\subsubsection{pytb.ThunderBoltz.plot\_cs}
\label{\detokenize{api/pytb.ThunderBoltz.plot_cs:pytb-thunderboltz-plot-cs}}\label{\detokenize{api/pytb.ThunderBoltz.plot_cs::doc}}\index{plot\_cs() (pytb.ThunderBoltz method)@\spxentry{plot\_cs()}\spxextra{pytb.ThunderBoltz method}}

\begin{fulllineitems}
\phantomsection\label{\detokenize{api/pytb.ThunderBoltz.plot_cs:pytb.ThunderBoltz.plot_cs}}
\pysigstartsignatures
\pysiglinewithargsret{\sphinxcode{\sphinxupquote{ThunderBoltz.}}\sphinxbfcode{\sphinxupquote{plot\_cs}}}{\sphinxparam{\DUrole{n,n}{ax}\DUrole{o,o}{=}\DUrole{default_value}{None}}, \sphinxparam{\DUrole{n,n}{legend}\DUrole{o,o}{=}\DUrole{default_value}{True}}, \sphinxparam{\DUrole{n,n}{vsig}\DUrole{o,o}{=}\DUrole{default_value}{False}}, \sphinxparam{\DUrole{n,n}{thresholds}\DUrole{o,o}{=}\DUrole{default_value}{False}}, \sphinxparam{\DUrole{n,n}{save}\DUrole{o,o}{=}\DUrole{default_value}{None}}, \sphinxparam{\DUrole{o,o}{**}\DUrole{n,n}{plot\_args}}}{}
\pysigstopsignatures
\sphinxAtStartPar
Plot the cross sections models.
\begin{quote}\begin{description}
\sphinxlineitem{Parameters}\begin{itemize}
\item {} 
\sphinxAtStartPar
\sphinxstyleliteralstrong{\sphinxupquote{ax}} (\sphinxhref{https://matplotlib.org/stable/api/\_as\_gen/matplotlib.axes.Axes.html\#matplotlib.axes.Axes}{\sphinxcode{\sphinxupquote{Axes}}} or None) \textendash{} Optional
axes object to plot on top of, default is None.
If ax is None, then a new figure and ax object
will be created.

\item {} 
\sphinxAtStartPar
\sphinxstyleliteralstrong{\sphinxupquote{legend}} (\sphinxhref{https://docs.python.org/3/library/functions.html\#bool}{\sphinxstyleliteralemphasis{\sphinxupquote{bool}}}) \textendash{} Activate axes legend if true,
default is True.

\item {} 
\sphinxAtStartPar
\sphinxstyleliteralstrong{\sphinxupquote{vsig}} (\sphinxhref{https://docs.python.org/3/library/functions.html\#bool}{\sphinxstyleliteralemphasis{\sphinxupquote{bool}}}) \textendash{} Plot \(\sqrt{\frac{2\epsilon}{m_{\rm e}}}\sigma(\epsilon)\)
rather than \(\sigma(\epsilon)\).

\item {} 
\sphinxAtStartPar
\sphinxstyleliteralstrong{\sphinxupquote{thresholds}} (\sphinxhref{https://docs.python.org/3/library/functions.html\#bool}{\sphinxstyleliteralemphasis{\sphinxupquote{bool}}}) \textendash{} if True, plot energy units in thresholds.

\item {} 
\sphinxAtStartPar
\sphinxstyleliteralstrong{\sphinxupquote{save}} (\sphinxhref{https://docs.python.org/3/library/stdtypes.html\#str}{\sphinxstyleliteralemphasis{\sphinxupquote{str}}}) \textendash{} Optional location of the directory to save the plot in.

\item {} 
\sphinxAtStartPar
\sphinxstyleliteralstrong{\sphinxupquote{**plot\_args}} \textendash{} Optional arguments passed to Axes.plot().

\end{itemize}

\sphinxlineitem{Returns}
\sphinxAtStartPar
The axes object of the plot.

\sphinxlineitem{Return type}
\sphinxAtStartPar
\sphinxhref{https://matplotlib.org/stable/api/\_as\_gen/matplotlib.axes.Axes.html\#matplotlib.axes.Axes}{\sphinxcode{\sphinxupquote{matplotlib.axes.Axes}}}

\end{description}\end{quote}

\end{fulllineitems}


\sphinxstepscope


\subsubsection{pytb.ThunderBoltz.plot\_edf\_comps}
\label{\detokenize{api/pytb.ThunderBoltz.plot_edf_comps:pytb-thunderboltz-plot-edf-comps}}\label{\detokenize{api/pytb.ThunderBoltz.plot_edf_comps::doc}}\index{plot\_edf\_comps() (pytb.ThunderBoltz method)@\spxentry{plot\_edf\_comps()}\spxextra{pytb.ThunderBoltz method}}

\begin{fulllineitems}
\phantomsection\label{\detokenize{api/pytb.ThunderBoltz.plot_edf_comps:pytb.ThunderBoltz.plot_edf_comps}}
\pysigstartsignatures
\pysiglinewithargsret{\sphinxcode{\sphinxupquote{ThunderBoltz.}}\sphinxbfcode{\sphinxupquote{plot\_edf\_comps}}}{\sphinxparam{\DUrole{n,n}{steps}\DUrole{o,o}{=}\DUrole{default_value}{\textquotesingle{}last\textquotesingle{}}}, \sphinxparam{\DUrole{n,n}{sample\_cap}\DUrole{o,o}{=}\DUrole{default_value}{500000}}, \sphinxparam{\DUrole{n,n}{bins}\DUrole{o,o}{=}\DUrole{default_value}{100}}, \sphinxparam{\DUrole{n,n}{maxwellian}\DUrole{o,o}{=}\DUrole{default_value}{True}}, \sphinxparam{\DUrole{n,n}{save}\DUrole{o,o}{=}\DUrole{default_value}{None}}}{}
\pysigstopsignatures
\sphinxAtStartPar
Plot the directional components of the energy distribution function.
\begin{quote}\begin{description}
\sphinxlineitem{Parameters}\begin{itemize}
\item {} 
\sphinxAtStartPar
\sphinxstyleliteralstrong{\sphinxupquote{steps}} (\sphinxhref{https://docs.python.org/3/library/stdtypes.html\#str}{\sphinxstyleliteralemphasis{\sphinxupquote{str}}}\sphinxstyleliteralemphasis{\sphinxupquote{, }}\sphinxhref{https://docs.python.org/3/library/stdtypes.html\#list}{\sphinxstyleliteralemphasis{\sphinxupquote{list}}}\sphinxstyleliteralemphasis{\sphinxupquote{{[}}}\sphinxhref{https://docs.python.org/3/library/functions.html\#int}{\sphinxstyleliteralemphasis{\sphinxupquote{int}}}\sphinxstyleliteralemphasis{\sphinxupquote{{]}}}\sphinxstyleliteralemphasis{\sphinxupquote{, or }}\sphinxhref{https://docs.python.org/3/library/functions.html\#int}{\sphinxstyleliteralemphasis{\sphinxupquote{int}}}) \textendash{} 
\sphinxAtStartPar
Options for which time steps to
read:
\begin{itemize}
\item {} 
\sphinxAtStartPar
\sphinxcode{\sphinxupquote{"last"}}: Only read the VDF of the last time step

\item {} 
\sphinxAtStartPar
\sphinxcode{\sphinxupquote{"first"}}: Only read the VDF of the first time step

\item {} 
\sphinxAtStartPar
\sphinxcode{\sphinxupquote{"all"}}: Read a separate VDF for each time step.

\item {} 
\sphinxAtStartPar
\sphinxcode{\sphinxupquote{list{[}int{]}}}: Read VDF for each time step included in list.

\item {} 
\sphinxAtStartPar
\sphinxcode{\sphinxupquote{int}}: read VDF at one specific time step.

\end{itemize}


\item {} 
\sphinxAtStartPar
\sphinxstyleliteralstrong{\sphinxupquote{sample\_cap}} (\sphinxhref{https://docs.python.org/3/library/functions.html\#int}{\sphinxstyleliteralemphasis{\sphinxupquote{int}}}) \textendash{} Limit the number of samples read from the dump
file for very large files. Default is 500000. If bool(sample\_cap)
evaluates to \sphinxcode{\sphinxupquote{False}}, then no cap will be imposed.

\item {} 
\sphinxAtStartPar
\sphinxstyleliteralstrong{\sphinxupquote{bins}} (\sphinxhref{https://docs.python.org/3/library/functions.html\#int}{\sphinxstyleliteralemphasis{\sphinxupquote{int}}}) \textendash{} Total number of bins to divide the energy space into.

\item {} 
\sphinxAtStartPar
\sphinxstyleliteralstrong{\sphinxupquote{maxwellian}} (\sphinxhref{https://docs.python.org/3/library/functions.html\#bool}{\sphinxstyleliteralemphasis{\sphinxupquote{bool}}}) \textendash{} Option to draw a maxwellian distribution
with the same temperature for comparison.

\item {} 
\sphinxAtStartPar
\sphinxstyleliteralstrong{\sphinxupquote{save}} (\sphinxhref{https://docs.python.org/3/library/stdtypes.html\#str}{\sphinxstyleliteralemphasis{\sphinxupquote{str}}}) \textendash{} Optional location of directory to save the figure in.

\end{itemize}

\end{description}\end{quote}

\end{fulllineitems}


\sphinxstepscope


\subsubsection{pytb.ThunderBoltz.plot\_edfs}
\label{\detokenize{api/pytb.ThunderBoltz.plot_edfs:pytb-thunderboltz-plot-edfs}}\label{\detokenize{api/pytb.ThunderBoltz.plot_edfs::doc}}\index{plot\_edfs() (pytb.ThunderBoltz method)@\spxentry{plot\_edfs()}\spxextra{pytb.ThunderBoltz method}}

\begin{fulllineitems}
\phantomsection\label{\detokenize{api/pytb.ThunderBoltz.plot_edfs:pytb.ThunderBoltz.plot_edfs}}
\pysigstartsignatures
\pysiglinewithargsret{\sphinxcode{\sphinxupquote{ThunderBoltz.}}\sphinxbfcode{\sphinxupquote{plot\_edfs}}}{\sphinxparam{\DUrole{n,n}{steps}\DUrole{o,o}{=}\DUrole{default_value}{\textquotesingle{}last\textquotesingle{}}}, \sphinxparam{\DUrole{n,n}{sample\_cap}\DUrole{o,o}{=}\DUrole{default_value}{500000}}, \sphinxparam{\DUrole{n,n}{bins}\DUrole{o,o}{=}\DUrole{default_value}{100}}, \sphinxparam{\DUrole{n,n}{plot\_cs}\DUrole{o,o}{=}\DUrole{default_value}{False}}, \sphinxparam{\DUrole{n,n}{save}\DUrole{o,o}{=}\DUrole{default_value}{None}}}{}
\pysigstopsignatures
\sphinxAtStartPar
Plot the electron total energy distribution function, optionally
include the provided cross sections for comparison.
\begin{quote}\begin{description}
\sphinxlineitem{Parameters}\begin{itemize}
\item {} 
\sphinxAtStartPar
\sphinxstyleliteralstrong{\sphinxupquote{steps}} (\sphinxhref{https://docs.python.org/3/library/stdtypes.html\#str}{\sphinxstyleliteralemphasis{\sphinxupquote{str}}}\sphinxstyleliteralemphasis{\sphinxupquote{, }}\sphinxhref{https://docs.python.org/3/library/stdtypes.html\#list}{\sphinxstyleliteralemphasis{\sphinxupquote{list}}}\sphinxstyleliteralemphasis{\sphinxupquote{{[}}}\sphinxhref{https://docs.python.org/3/library/functions.html\#int}{\sphinxstyleliteralemphasis{\sphinxupquote{int}}}\sphinxstyleliteralemphasis{\sphinxupquote{{]}}}\sphinxstyleliteralemphasis{\sphinxupquote{, or }}\sphinxhref{https://docs.python.org/3/library/functions.html\#int}{\sphinxstyleliteralemphasis{\sphinxupquote{int}}}) \textendash{} 
\sphinxAtStartPar
Options for which time steps to
read:
\begin{itemize}
\item {} 
\sphinxAtStartPar
\sphinxcode{\sphinxupquote{"last"}}: Only read the VDF of the last time step

\item {} 
\sphinxAtStartPar
\sphinxcode{\sphinxupquote{"first"}}: Only read the VDF of the first time step

\item {} 
\sphinxAtStartPar
\sphinxcode{\sphinxupquote{"all"}}: Read a separate VDF for each time step.

\item {} 
\sphinxAtStartPar
\sphinxcode{\sphinxupquote{list{[}int{]}}}: Read VDF for each time step included in list.

\item {} 
\sphinxAtStartPar
\sphinxcode{\sphinxupquote{int}}: read VDF at one specific time step.

\end{itemize}


\item {} 
\sphinxAtStartPar
\sphinxstyleliteralstrong{\sphinxupquote{sample\_cap}} (\sphinxhref{https://docs.python.org/3/library/functions.html\#int}{\sphinxstyleliteralemphasis{\sphinxupquote{int}}}) \textendash{} Limit the number of samples read from the dump
file for very large files. Default is 500000. If bool(sample\_cap)
evaluates to \sphinxcode{\sphinxupquote{False}}, then no cap will be imposed.

\item {} 
\sphinxAtStartPar
\sphinxstyleliteralstrong{\sphinxupquote{bins}} (\sphinxhref{https://docs.python.org/3/library/functions.html\#int}{\sphinxstyleliteralemphasis{\sphinxupquote{int}}}) \textendash{} Total number of bins to divide the energy space into.

\item {} 
\sphinxAtStartPar
\sphinxstyleliteralstrong{\sphinxupquote{maxwellian}} (\sphinxhref{https://docs.python.org/3/library/functions.html\#bool}{\sphinxstyleliteralemphasis{\sphinxupquote{bool}}}) \textendash{} Option to draw a maxwellian distribution
with the same temperature for comparison.

\item {} 
\sphinxAtStartPar
\sphinxstyleliteralstrong{\sphinxupquote{save}} (\sphinxhref{https://docs.python.org/3/library/functions.html\#bool}{\sphinxstyleliteralemphasis{\sphinxupquote{bool}}}) \textendash{} Optional location of directory to save the figure in.

\end{itemize}

\sphinxlineitem{Returns}
\sphinxAtStartPar
The list of
figures and a list of their corresponding step indices.

\sphinxlineitem{Return type}
\sphinxAtStartPar
(Tuple{[}\sphinxhref{https://docs.python.org/3/library/stdtypes.html\#list}{list}{[}\sphinxhref{https://matplotlib.org/stable/api/figure\_api.html\#matplotlib.figure.Figure}{matplotlib.figure.Figure}{]}, \sphinxhref{https://docs.python.org/3/library/stdtypes.html\#list}{list}{[}\sphinxhref{https://docs.python.org/3/library/functions.html\#int}{int}{]})

\end{description}\end{quote}

\begin{sphinxadmonition}{note}{Note:}
\sphinxAtStartPar
It currently assumed that only data for one particle type is to be
plotted.
\end{sphinxadmonition}

\end{fulllineitems}


\sphinxstepscope


\subsubsection{pytb.ThunderBoltz.plot\_rates}
\label{\detokenize{api/pytb.ThunderBoltz.plot_rates:pytb-thunderboltz-plot-rates}}\label{\detokenize{api/pytb.ThunderBoltz.plot_rates::doc}}\index{plot\_rates() (pytb.ThunderBoltz method)@\spxentry{plot\_rates()}\spxextra{pytb.ThunderBoltz method}}

\begin{fulllineitems}
\phantomsection\label{\detokenize{api/pytb.ThunderBoltz.plot_rates:pytb.ThunderBoltz.plot_rates}}
\pysigstartsignatures
\pysiglinewithargsret{\sphinxcode{\sphinxupquote{ThunderBoltz.}}\sphinxbfcode{\sphinxupquote{plot\_rates}}}{\sphinxparam{\DUrole{n,n}{save}\DUrole{o,o}{=}\DUrole{default_value}{None}}, \sphinxparam{\DUrole{n,n}{stamp}\DUrole{o,o}{=}\DUrole{default_value}{None}}, \sphinxparam{\DUrole{n,n}{v}\DUrole{o,o}{=}\DUrole{default_value}{0}}, \sphinxparam{\DUrole{n,n}{update}\DUrole{o,o}{=}\DUrole{default_value}{True}}}{}
\pysigstopsignatures
\sphinxAtStartPar
Create a diagnostic plot of ThunderBoltz time series
data.
\begin{quote}\begin{description}
\sphinxlineitem{Parameters}\begin{itemize}
\item {} 
\sphinxAtStartPar
\sphinxstyleliteralstrong{\sphinxupquote{series}} (\sphinxhref{https://docs.python.org/3/library/stdtypes.html\#list}{\sphinxstyleliteralemphasis{\sphinxupquote{list}}}\sphinxstyleliteralemphasis{\sphinxupquote{{[}}}\sphinxhref{https://docs.python.org/3/library/stdtypes.html\#str}{\sphinxstyleliteralemphasis{\sphinxupquote{str}}}\sphinxstyleliteralemphasis{\sphinxupquote{{]}}}) \textendash{} The y\sphinxhyphen{}parameters to plot onto the time series figure.

\item {} 
\sphinxAtStartPar
\sphinxstyleliteralstrong{\sphinxupquote{save}} (\sphinxhref{https://docs.python.org/3/library/stdtypes.html\#str}{\sphinxstyleliteralemphasis{\sphinxupquote{str}}}) \textendash{} Option to save the plot to a file path.

\item {} 
\sphinxAtStartPar
\sphinxstyleliteralstrong{\sphinxupquote{stamp}} (\sphinxhref{https://docs.python.org/3/library/stdtypes.html\#list}{\sphinxstyleliteralemphasis{\sphinxupquote{list}}}\sphinxstyleliteralemphasis{\sphinxupquote{{[}}}\sphinxhref{https://docs.python.org/3/library/stdtypes.html\#str}{\sphinxstyleliteralemphasis{\sphinxupquote{str}}}\sphinxstyleliteralemphasis{\sphinxupquote{{]}}}) \textendash{} Option to stamp the figure with the value of
descriptive parameters, e.g. the field, or initial
number of particles. See {\hyperref[\detokenize{api/pytb.parameters.TBParameters:pytb.parameters.TBParameters}]{\sphinxcrossref{\sphinxcode{\sphinxupquote{TBParameters}}}}}
and {\hyperref[\detokenize{api/pytb.parameters.WrapParameters:pytb.parameters.WrapParameters}]{\sphinxcrossref{\sphinxcode{\sphinxupquote{WrapParameters}}}}}.

\item {} 
\sphinxAtStartPar
\sphinxstyleliteralstrong{\sphinxupquote{v}} (\sphinxhref{https://docs.python.org/3/library/functions.html\#int}{\sphinxstyleliteralemphasis{\sphinxupquote{int}}}) \textendash{} Verbosity \textendash{} 0: silent, 1: print file paths before
plotting.

\item {} 
\sphinxAtStartPar
\sphinxstyleliteralstrong{\sphinxupquote{update}} (\sphinxhref{https://docs.python.org/3/library/functions.html\#bool}{\sphinxstyleliteralemphasis{\sphinxupquote{bool}}}) \textendash{} If set to \sphinxcode{\sphinxupquote{False}}, assume required data has
already been parsed into ThunderBoltz frames.

\end{itemize}

\sphinxlineitem{Returns}
\sphinxAtStartPar
The plot\_rate figure object.

\sphinxlineitem{Return type}
\sphinxAtStartPar
\sphinxhref{https://matplotlib.org/stable/api/figure\_api.html\#matplotlib.figure.Figure}{\sphinxcode{\sphinxupquote{matplotlib.figure.Figure}}}

\end{description}\end{quote}

\end{fulllineitems}


\sphinxstepscope


\subsubsection{pytb.ThunderBoltz.plot\_timeseries}
\label{\detokenize{api/pytb.ThunderBoltz.plot_timeseries:pytb-thunderboltz-plot-timeseries}}\label{\detokenize{api/pytb.ThunderBoltz.plot_timeseries::doc}}\index{plot\_timeseries() (pytb.ThunderBoltz method)@\spxentry{plot\_timeseries()}\spxextra{pytb.ThunderBoltz method}}

\begin{fulllineitems}
\phantomsection\label{\detokenize{api/pytb.ThunderBoltz.plot_timeseries:pytb.ThunderBoltz.plot_timeseries}}
\pysigstartsignatures
\pysiglinewithargsret{\sphinxcode{\sphinxupquote{ThunderBoltz.}}\sphinxbfcode{\sphinxupquote{plot\_timeseries}}}{\sphinxparam{\DUrole{n,n}{series}\DUrole{o,o}{=}\DUrole{default_value}{None}}, \sphinxparam{\DUrole{n,n}{save}\DUrole{o,o}{=}\DUrole{default_value}{None}}, \sphinxparam{\DUrole{n,n}{stamp}\DUrole{o,o}{=}\DUrole{default_value}{{[}{]}}}, \sphinxparam{\DUrole{n,n}{v}\DUrole{o,o}{=}\DUrole{default_value}{0}}, \sphinxparam{\DUrole{n,n}{update}\DUrole{o,o}{=}\DUrole{default_value}{True}}}{}
\pysigstopsignatures
\sphinxAtStartPar
Create a diagnostic plot of ThunderBoltz time series
data.
\begin{quote}\begin{description}
\sphinxlineitem{Parameters}\begin{itemize}
\item {} 
\sphinxAtStartPar
\sphinxstyleliteralstrong{\sphinxupquote{series}} (\sphinxhref{https://docs.python.org/3/library/stdtypes.html\#list}{\sphinxstyleliteralemphasis{\sphinxupquote{list}}}\sphinxstyleliteralemphasis{\sphinxupquote{{[}}}\sphinxhref{https://docs.python.org/3/library/stdtypes.html\#str}{\sphinxstyleliteralemphasis{\sphinxupquote{str}}}\sphinxstyleliteralemphasis{\sphinxupquote{{]}}}) \textendash{} The y\sphinxhyphen{}parameters to plot onto the time series figure.

\item {} 
\sphinxAtStartPar
\sphinxstyleliteralstrong{\sphinxupquote{save}} (\sphinxhref{https://docs.python.org/3/library/stdtypes.html\#str}{\sphinxstyleliteralemphasis{\sphinxupquote{str}}}) \textendash{} Option to save the plot to a file path.

\item {} 
\sphinxAtStartPar
\sphinxstyleliteralstrong{\sphinxupquote{stamp}} (\sphinxhref{https://docs.python.org/3/library/stdtypes.html\#list}{\sphinxstyleliteralemphasis{\sphinxupquote{list}}}\sphinxstyleliteralemphasis{\sphinxupquote{{[}}}\sphinxhref{https://docs.python.org/3/library/stdtypes.html\#str}{\sphinxstyleliteralemphasis{\sphinxupquote{str}}}\sphinxstyleliteralemphasis{\sphinxupquote{{]}}}) \textendash{} Option to stamp the figure with the value of
descriptive parameters, e.g. the field, or initial
number of particles. See {\hyperref[\detokenize{api/pytb.parameters.TBParameters:pytb.parameters.TBParameters}]{\sphinxcrossref{\sphinxcode{\sphinxupquote{TBParameters}}}}}
and {\hyperref[\detokenize{api/pytb.parameters.WrapParameters:pytb.parameters.WrapParameters}]{\sphinxcrossref{\sphinxcode{\sphinxupquote{WrapParameters}}}}}.

\item {} 
\sphinxAtStartPar
\sphinxstyleliteralstrong{\sphinxupquote{v}} (\sphinxhref{https://docs.python.org/3/library/functions.html\#int}{\sphinxstyleliteralemphasis{\sphinxupquote{int}}}) \textendash{} Verbosity \textendash{} 0: silent, 1: print file paths before
plotting.

\item {} 
\sphinxAtStartPar
\sphinxstyleliteralstrong{\sphinxupquote{update}} (\sphinxhref{https://docs.python.org/3/library/functions.html\#bool}{\sphinxstyleliteralemphasis{\sphinxupquote{bool}}}) \textendash{} If set to \sphinxcode{\sphinxupquote{False}}, assume required data has
already been parsed into ThunderBoltz frames.

\end{itemize}

\sphinxlineitem{Returns}
\sphinxAtStartPar
The timeseries figure object.

\sphinxlineitem{Return type}
\sphinxAtStartPar
\sphinxhref{https://matplotlib.org/stable/api/figure\_api.html\#matplotlib.figure.Figure}{\sphinxcode{\sphinxupquote{matplotlib.figure.Figure}}}

\end{description}\end{quote}

\end{fulllineitems}


\sphinxstepscope


\subsubsection{pytb.ThunderBoltz.plot\_vdfs}
\label{\detokenize{api/pytb.ThunderBoltz.plot_vdfs:pytb-thunderboltz-plot-vdfs}}\label{\detokenize{api/pytb.ThunderBoltz.plot_vdfs::doc}}\index{plot\_vdfs() (pytb.ThunderBoltz method)@\spxentry{plot\_vdfs()}\spxextra{pytb.ThunderBoltz method}}

\begin{fulllineitems}
\phantomsection\label{\detokenize{api/pytb.ThunderBoltz.plot_vdfs:pytb.ThunderBoltz.plot_vdfs}}
\pysigstartsignatures
\pysiglinewithargsret{\sphinxcode{\sphinxupquote{ThunderBoltz.}}\sphinxbfcode{\sphinxupquote{plot\_vdfs}}}{\sphinxparam{\DUrole{n,n}{steps}\DUrole{o,o}{=}\DUrole{default_value}{\textquotesingle{}last\textquotesingle{}}}, \sphinxparam{\DUrole{n,n}{save}\DUrole{o,o}{=}\DUrole{default_value}{None}}, \sphinxparam{\DUrole{n,n}{bins}\DUrole{o,o}{=}\DUrole{default_value}{100}}, \sphinxparam{\DUrole{n,n}{sample\_cap}\DUrole{o,o}{=}\DUrole{default_value}{500000}}}{}
\pysigstopsignatures
\sphinxAtStartPar
Plot the joint distribution heat map between the x\sphinxhyphen{}y and x\sphinxhyphen{}z
velocities.
\begin{quote}\begin{description}
\sphinxlineitem{Parameters}\begin{itemize}
\item {} 
\sphinxAtStartPar
\sphinxstyleliteralstrong{\sphinxupquote{steps}} (\sphinxhref{https://docs.python.org/3/library/stdtypes.html\#str}{\sphinxstyleliteralemphasis{\sphinxupquote{str}}}\sphinxstyleliteralemphasis{\sphinxupquote{, }}\sphinxhref{https://docs.python.org/3/library/stdtypes.html\#list}{\sphinxstyleliteralemphasis{\sphinxupquote{list}}}\sphinxstyleliteralemphasis{\sphinxupquote{{[}}}\sphinxhref{https://docs.python.org/3/library/functions.html\#int}{\sphinxstyleliteralemphasis{\sphinxupquote{int}}}\sphinxstyleliteralemphasis{\sphinxupquote{{]}}}\sphinxstyleliteralemphasis{\sphinxupquote{, or }}\sphinxhref{https://docs.python.org/3/library/functions.html\#int}{\sphinxstyleliteralemphasis{\sphinxupquote{int}}}) \textendash{} 
\sphinxAtStartPar
Options for which time steps to
read:
\begin{itemize}
\item {} 
\sphinxAtStartPar
\sphinxcode{\sphinxupquote{"last"}}: Only read the VDF of the last time step

\item {} 
\sphinxAtStartPar
\sphinxcode{\sphinxupquote{"first"}}: Only read the VDF of the first time step

\item {} 
\sphinxAtStartPar
\sphinxcode{\sphinxupquote{"all"}}: Read a separate VDF for each time step.

\item {} 
\sphinxAtStartPar
\sphinxcode{\sphinxupquote{list{[}int{]}}}: Read VDF for each time step included in list.

\item {} 
\sphinxAtStartPar
\sphinxcode{\sphinxupquote{int}}: read VDF at one specific time step.

\end{itemize}


\item {} 
\sphinxAtStartPar
\sphinxstyleliteralstrong{\sphinxupquote{sample\_cap}} (\sphinxhref{https://docs.python.org/3/library/functions.html\#int}{\sphinxstyleliteralemphasis{\sphinxupquote{int}}}) \textendash{} Limit the number of samples read from the dump
file for very large files. Default is 500000. If bool(sample\_cap)
evaluates to \sphinxcode{\sphinxupquote{False}}, then no cap will be imposed.

\item {} 
\sphinxAtStartPar
\sphinxstyleliteralstrong{\sphinxupquote{bins}} (\sphinxhref{https://docs.python.org/3/library/functions.html\#int}{\sphinxstyleliteralemphasis{\sphinxupquote{int}}}) \textendash{} Total number of bins to divide the energy space into.

\item {} 
\sphinxAtStartPar
\sphinxstyleliteralstrong{\sphinxupquote{save}} (\sphinxhref{https://docs.python.org/3/library/stdtypes.html\#str}{\sphinxstyleliteralemphasis{\sphinxupquote{str}}}) \textendash{} Optional location of directory to save the figure in.

\end{itemize}

\sphinxlineitem{Returns}
\sphinxAtStartPar
The list of
figures and a list of their corresponding step indices.

\sphinxlineitem{Return type}
\sphinxAtStartPar
(Tuple{[}\sphinxhref{https://docs.python.org/3/library/stdtypes.html\#list}{list}{[}\sphinxhref{https://matplotlib.org/stable/api/figure\_api.html\#matplotlib.figure.Figure}{matplotlib.figure.Figure}{]}, \sphinxhref{https://docs.python.org/3/library/stdtypes.html\#list}{list}{[}\sphinxhref{https://docs.python.org/3/library/functions.html\#int}{int}{]})

\end{description}\end{quote}

\end{fulllineitems}


\sphinxstepscope


\subsubsection{pytb.ThunderBoltz.read}
\label{\detokenize{api/pytb.ThunderBoltz.read:pytb-thunderboltz-read}}\label{\detokenize{api/pytb.ThunderBoltz.read::doc}}\index{read() (pytb.ThunderBoltz method)@\spxentry{read()}\spxextra{pytb.ThunderBoltz method}}

\begin{fulllineitems}
\phantomsection\label{\detokenize{api/pytb.ThunderBoltz.read:pytb.ThunderBoltz.read}}
\pysigstartsignatures
\pysiglinewithargsret{\sphinxcode{\sphinxupquote{ThunderBoltz.}}\sphinxbfcode{\sphinxupquote{read}}}{\sphinxparam{\DUrole{n,n}{directory}\DUrole{o,o}{=}\DUrole{default_value}{None}}, \sphinxparam{\DUrole{n,n}{read\_input}\DUrole{o,o}{=}\DUrole{default_value}{True}}, \sphinxparam{\DUrole{n,n}{read\_cs\_data}\DUrole{o,o}{=}\DUrole{default_value}{False}}, \sphinxparam{\DUrole{n,n}{only}\DUrole{o,o}{=}\DUrole{default_value}{None}}}{}
\pysigstopsignatures
\sphinxAtStartPar
Read the simulation directory of a ThunderBoltz run, possibly
all of its input and output files.
\begin{quote}\begin{description}
\sphinxlineitem{Parameters}\begin{itemize}
\item {} 
\sphinxAtStartPar
\sphinxstyleliteralstrong{\sphinxupquote{directory}} (\sphinxhref{https://docs.python.org/3/library/stdtypes.html\#str}{\sphinxstyleliteralemphasis{\sphinxupquote{str}}}) \textendash{} The location of the simulation directory
from which to read.

\item {} 
\sphinxAtStartPar
\sphinxstyleliteralstrong{\sphinxupquote{read\_input}} (\sphinxhref{https://docs.python.org/3/library/functions.html\#bool}{\sphinxstyleliteralemphasis{\sphinxupquote{bool}}}) \textendash{} Whether or not to read any input data.

\item {} 
\sphinxAtStartPar
\sphinxstyleliteralstrong{\sphinxupquote{read\_cs\_data}} (\sphinxhref{https://docs.python.org/3/library/functions.html\#bool}{\sphinxstyleliteralemphasis{\sphinxupquote{bool}}}) \textendash{} Whether or not to read cross section data.
This can be expensive, and often isn’t necessary.

\item {} 
\sphinxAtStartPar
\sphinxstyleliteralstrong{\sphinxupquote{only}} (\sphinxhref{https://docs.python.org/3/library/stdtypes.html\#list}{\sphinxstyleliteralemphasis{\sphinxupquote{list}}}\sphinxstyleliteralemphasis{\sphinxupquote{ or }}\sphinxstyleliteralemphasis{\sphinxupquote{None}}) \textendash{} Only read certain types of files. Default is
\sphinxcode{\sphinxupquote{{[}"thunderboltz.out", "Particle\_Type", "Counts.dat", ""thunderboltz.log"{]}}}.

\end{itemize}

\end{description}\end{quote}

\end{fulllineitems}


\sphinxstepscope


\subsubsection{pytb.ThunderBoltz.read\_log}
\label{\detokenize{api/pytb.ThunderBoltz.read_log:pytb-thunderboltz-read-log}}\label{\detokenize{api/pytb.ThunderBoltz.read_log::doc}}\index{read\_log() (pytb.ThunderBoltz method)@\spxentry{read\_log()}\spxextra{pytb.ThunderBoltz method}}

\begin{fulllineitems}
\phantomsection\label{\detokenize{api/pytb.ThunderBoltz.read_log:pytb.ThunderBoltz.read_log}}
\pysigstartsignatures
\pysiglinewithargsret{\sphinxcode{\sphinxupquote{ThunderBoltz.}}\sphinxbfcode{\sphinxupquote{read\_log}}}{\sphinxparam{\DUrole{n,n}{logfile}}}{}
\pysigstopsignatures
\sphinxAtStartPar
Read json file from simulation directory. Update
the corresponding settings in the ThunderBoltz object.
\begin{quote}\begin{description}
\sphinxlineitem{Parameters}
\sphinxAtStartPar
\sphinxstyleliteralstrong{\sphinxupquote{logfile}} (\sphinxhref{https://docs.python.org/3/library/stdtypes.html\#str}{\sphinxstyleliteralemphasis{\sphinxupquote{str}}}) \textendash{} The path name of the logfile to read.

\end{description}\end{quote}

\end{fulllineitems}


\sphinxstepscope


\subsubsection{pytb.ThunderBoltz.read\_particle\_table}
\label{\detokenize{api/pytb.ThunderBoltz.read_particle_table:pytb-thunderboltz-read-particle-table}}\label{\detokenize{api/pytb.ThunderBoltz.read_particle_table::doc}}\index{read\_particle\_table() (pytb.ThunderBoltz method)@\spxentry{read\_particle\_table()}\spxextra{pytb.ThunderBoltz method}}

\begin{fulllineitems}
\phantomsection\label{\detokenize{api/pytb.ThunderBoltz.read_particle_table:pytb.ThunderBoltz.read_particle_table}}
\pysigstartsignatures
\pysiglinewithargsret{\sphinxcode{\sphinxupquote{ThunderBoltz.}}\sphinxbfcode{\sphinxupquote{read\_particle\_table}}}{\sphinxparam{\DUrole{n,n}{i}}}{}
\pysigstopsignatures
\sphinxAtStartPar
Read species specific output data, including
density, velocity, displacement, energy, and temperature.
\begin{quote}\begin{description}
\sphinxlineitem{Parameters}
\sphinxAtStartPar
\sphinxstyleliteralstrong{\sphinxupquote{i}} (\sphinxhref{https://docs.python.org/3/library/functions.html\#int}{\sphinxstyleliteralemphasis{\sphinxupquote{int}}}) \textendash{} The species index.

\sphinxlineitem{Returns}
\sphinxAtStartPar
The particle data for
species \sphinxcode{\sphinxupquote{i}}.

\sphinxlineitem{Return type}
\sphinxAtStartPar
\sphinxhref{http://pandas.pydata.org/pandas-docs/dev/reference/api/pandas.DataFrame.html\#pandas.DataFrame}{\sphinxcode{\sphinxupquote{pandas.DataFrame}}}

\end{description}\end{quote}

\end{fulllineitems}


\sphinxstepscope


\subsubsection{pytb.ThunderBoltz.read\_stdout}
\label{\detokenize{api/pytb.ThunderBoltz.read_stdout:pytb-thunderboltz-read-stdout}}\label{\detokenize{api/pytb.ThunderBoltz.read_stdout::doc}}\index{read\_stdout() (pytb.ThunderBoltz method)@\spxentry{read\_stdout()}\spxextra{pytb.ThunderBoltz method}}

\begin{fulllineitems}
\phantomsection\label{\detokenize{api/pytb.ThunderBoltz.read_stdout:pytb.ThunderBoltz.read_stdout}}
\pysigstartsignatures
\pysiglinewithargsret{\sphinxcode{\sphinxupquote{ThunderBoltz.}}\sphinxbfcode{\sphinxupquote{read\_stdout}}}{\sphinxparam{\DUrole{n,n}{fname}}}{}
\pysigstopsignatures
\sphinxAtStartPar
Read the banner output data.
\begin{quote}\begin{description}
\sphinxlineitem{Parameters}
\sphinxAtStartPar
\sphinxstyleliteralstrong{\sphinxupquote{fname}} (\sphinxhref{https://docs.python.org/3/library/stdtypes.html\#str}{\sphinxstyleliteralemphasis{\sphinxupquote{str}}}) \textendash{} The name of the \sphinxcode{\sphinxupquote{.out}} file to read.

\sphinxlineitem{Returns}
\sphinxAtStartPar
The banner data.

\sphinxlineitem{Return type}
\sphinxAtStartPar
\sphinxhref{http://pandas.pydata.org/pandas-docs/dev/reference/api/pandas.DataFrame.html\#pandas.DataFrame}{\sphinxcode{\sphinxupquote{pandas.DataFrame}}}

\end{description}\end{quote}

\begin{sphinxadmonition}{note}{Note:}
\sphinxAtStartPar
If ThunderBoltz warnings are found (e.g. particle overload),
a message will be appended to \sphinxcode{\sphinxupquote{err\_stack}}.
\end{sphinxadmonition}

\end{fulllineitems}


\sphinxstepscope


\subsubsection{pytb.ThunderBoltz.read\_tb\_params}
\label{\detokenize{api/pytb.ThunderBoltz.read_tb_params:pytb-thunderboltz-read-tb-params}}\label{\detokenize{api/pytb.ThunderBoltz.read_tb_params::doc}}\index{read\_tb\_params() (pytb.ThunderBoltz method)@\spxentry{read\_tb\_params()}\spxextra{pytb.ThunderBoltz method}}

\begin{fulllineitems}
\phantomsection\label{\detokenize{api/pytb.ThunderBoltz.read_tb_params:pytb.ThunderBoltz.read_tb_params}}
\pysigstartsignatures
\pysiglinewithargsret{\sphinxcode{\sphinxupquote{ThunderBoltz.}}\sphinxbfcode{\sphinxupquote{read\_tb\_params}}}{\sphinxparam{\DUrole{n,n}{fname}}, \sphinxparam{\DUrole{n,n}{ignore}\DUrole{o,o}{=}\DUrole{default_value}{{[}{]}}}}{}
\pysigstopsignatures
\sphinxAtStartPar
Takes file name of an input deck, updates the simulation
parameters and returns the simulation parameters which were
read from the file \sphinxcode{\sphinxupquote{fname}}.
\begin{quote}\begin{description}
\sphinxlineitem{Parameters}\begin{itemize}
\item {} 
\sphinxAtStartPar
\sphinxstyleliteralstrong{\sphinxupquote{fname}} (\sphinxhref{https://docs.python.org/3/library/stdtypes.html\#str}{\sphinxstyleliteralemphasis{\sphinxupquote{str}}}) \textendash{} The name of the indeck file to read.

\item {} 
\sphinxAtStartPar
\sphinxstyleliteralstrong{\sphinxupquote{ignore}} (\sphinxhref{https://docs.python.org/3/library/stdtypes.html\#list}{\sphinxstyleliteralemphasis{\sphinxupquote{list}}}\sphinxstyleliteralemphasis{\sphinxupquote{{[}}}\sphinxhref{https://docs.python.org/3/library/stdtypes.html\#str}{\sphinxstyleliteralemphasis{\sphinxupquote{str}}}\sphinxstyleliteralemphasis{\sphinxupquote{{]}}}) \textendash{} Don’t read certain ThunderBoltz
params, e.g. \sphinxcode{\sphinxupquote{{[}"MP", "QP"{]}}} would ignore the
mass and charge parameters in an indeck file.

\end{itemize}

\sphinxlineitem{Returns}
\sphinxAtStartPar
\sphinxcode{\sphinxupquote{tb\_params}}.

\sphinxlineitem{Return type}
\sphinxAtStartPar
\sphinxhref{https://docs.python.org/3/library/stdtypes.html\#dict}{dict}

\end{description}\end{quote}

\end{fulllineitems}


\sphinxstepscope


\subsubsection{pytb.ThunderBoltz.reset}
\label{\detokenize{api/pytb.ThunderBoltz.reset:pytb-thunderboltz-reset}}\label{\detokenize{api/pytb.ThunderBoltz.reset::doc}}\index{reset() (pytb.ThunderBoltz method)@\spxentry{reset()}\spxextra{pytb.ThunderBoltz method}}

\begin{fulllineitems}
\phantomsection\label{\detokenize{api/pytb.ThunderBoltz.reset:pytb.ThunderBoltz.reset}}
\pysigstartsignatures
\pysiglinewithargsret{\sphinxcode{\sphinxupquote{ThunderBoltz.}}\sphinxbfcode{\sphinxupquote{reset}}}{}{}
\pysigstopsignatures
\sphinxAtStartPar
Reset output data for a new run.

\end{fulllineitems}


\sphinxstepscope


\subsubsection{pytb.ThunderBoltz.run}
\label{\detokenize{api/pytb.ThunderBoltz.run:pytb-thunderboltz-run}}\label{\detokenize{api/pytb.ThunderBoltz.run::doc}}\index{run() (pytb.ThunderBoltz method)@\spxentry{run()}\spxextra{pytb.ThunderBoltz method}}

\begin{fulllineitems}
\phantomsection\label{\detokenize{api/pytb.ThunderBoltz.run:pytb.ThunderBoltz.run}}
\pysigstartsignatures
\pysiglinewithargsret{\sphinxcode{\sphinxupquote{ThunderBoltz.}}\sphinxbfcode{\sphinxupquote{run}}}{\sphinxparam{\DUrole{n,n}{src\_path}\DUrole{o,o}{=}\DUrole{default_value}{None}}, \sphinxparam{\DUrole{n,n}{bin\_path}\DUrole{o,o}{=}\DUrole{default_value}{None}}, \sphinxparam{\DUrole{n,n}{out\_file}\DUrole{o,o}{=}\DUrole{default_value}{\textquotesingle{}thunderboltz\textquotesingle{}}}, \sphinxparam{\DUrole{n,n}{monitor}\DUrole{o,o}{=}\DUrole{default_value}{False}}, \sphinxparam{\DUrole{n,n}{dryrun}\DUrole{o,o}{=}\DUrole{default_value}{False}}, \sphinxparam{\DUrole{n,n}{debug}\DUrole{o,o}{=}\DUrole{default_value}{False}}, \sphinxparam{\DUrole{n,n}{std\_banner}\DUrole{o,o}{=}\DUrole{default_value}{False}}, \sphinxparam{\DUrole{n,n}{live}\DUrole{o,o}{=}\DUrole{default_value}{False}}, \sphinxparam{\DUrole{n,n}{live\_rate}\DUrole{o,o}{=}\DUrole{default_value}{False}}}{}
\pysigstopsignatures\begin{description}
\sphinxlineitem{Execute with the current parameters in the simulation directory.}
\sphinxAtStartPar
The internal API ThunderBoltz version will be used in lieu of
user\sphinxhyphen{}provided binary/source files.

\end{description}
\begin{quote}\begin{description}
\sphinxlineitem{Parameters}\begin{itemize}
\item {} 
\sphinxAtStartPar
\sphinxstyleliteralstrong{\sphinxupquote{src\_path}} (\sphinxhref{https://docs.python.org/3/library/stdtypes.html\#str}{\sphinxstyleliteralemphasis{\sphinxupquote{str}}}) \textendash{} Optional path to source files to copy into the
simulation directory. The source is then compiled there.

\item {} 
\sphinxAtStartPar
\sphinxstyleliteralstrong{\sphinxupquote{bin\_path}} (\sphinxhref{https://docs.python.org/3/library/stdtypes.html\#str}{\sphinxstyleliteralemphasis{\sphinxupquote{str}}}) \textendash{} Optional path to binary executable to copy
into the simulation directory.

\item {} 
\sphinxAtStartPar
\sphinxstyleliteralstrong{\sphinxupquote{out\_file}} (\sphinxhref{https://docs.python.org/3/library/stdtypes.html\#str}{\sphinxstyleliteralemphasis{\sphinxupquote{str}}}) \textendash{} The file name for stdout buffer of the
ThunderBoltz process.

\item {} 
\sphinxAtStartPar
\sphinxstyleliteralstrong{\sphinxupquote{monitor}} (\sphinxhref{https://docs.python.org/3/library/functions.html\#bool}{\sphinxstyleliteralemphasis{\sphinxupquote{bool}}}) \textendash{} Runtime flag, when set to \sphinxtitleref{True} will generate
an empty \sphinxtitleref{monitor} file in the simulation directory. Deleting this
file will cause the ThunderBoltz process to exit, but allow
the wrapper to continue execution. This is useful performing
several simulation calculations sequentially, but a manual exit
is required for each one.

\item {} 
\sphinxAtStartPar
\sphinxstyleliteralstrong{\sphinxupquote{dryrun}} (\sphinxhref{https://docs.python.org/3/library/functions.html\#bool}{\sphinxstyleliteralemphasis{\sphinxupquote{bool}}}) \textendash{} Setup all the files for the calculation, but do not
run the calculation.

\item {} 
\sphinxAtStartPar
\sphinxstyleliteralstrong{\sphinxupquote{debug}} \textendash{} (bool): Compile with C++ \sphinxcode{\sphinxupquote{\sphinxhyphen{}g}} debug flag.

\item {} 
\sphinxAtStartPar
\sphinxstyleliteralstrong{\sphinxupquote{std\_banner}} (\sphinxhref{https://docs.python.org/3/library/functions.html\#bool}{\sphinxstyleliteralemphasis{\sphinxupquote{bool}}}) \textendash{} Toggle banner output streaming to stdout in
addition to being written to the \sphinxcode{\sphinxupquote{out\_file}} buffer.

\item {} 
\sphinxAtStartPar
\sphinxstyleliteralstrong{\sphinxupquote{live}} (\sphinxhref{https://docs.python.org/3/library/functions.html\#bool}{\sphinxstyleliteralemphasis{\sphinxupquote{bool}}}) \textendash{} Run and update time series plotting GUI during simulation.

\item {} 
\sphinxAtStartPar
\sphinxstyleliteralstrong{\sphinxupquote{live\_rate}} (\sphinxhref{https://docs.python.org/3/library/functions.html\#bool}{\sphinxstyleliteralemphasis{\sphinxupquote{bool}}}) \textendash{} Run and update rate plotting GUI during simulation.

\end{itemize}

\sphinxlineitem{Raises}
\sphinxAtStartPar
\sphinxhref{https://docs.python.org/3/library/exceptions.html\#RuntimeError}{\sphinxstyleliteralstrong{\sphinxupquote{RuntimeError}}} \textendash{} if there is no simulation directory set or if
    the one provided does not exist.

\end{description}\end{quote}

\end{fulllineitems}


\sphinxstepscope


\subsubsection{pytb.ThunderBoltz.set\_}
\label{\detokenize{api/pytb.ThunderBoltz.set_:pytb-thunderboltz-set}}\label{\detokenize{api/pytb.ThunderBoltz.set_::doc}}\index{set\_() (pytb.ThunderBoltz method)@\spxentry{set\_()}\spxextra{pytb.ThunderBoltz method}}

\begin{fulllineitems}
\phantomsection\label{\detokenize{api/pytb.ThunderBoltz.set_:pytb.ThunderBoltz.set_}}
\pysigstartsignatures
\pysiglinewithargsret{\sphinxcode{\sphinxupquote{ThunderBoltz.}}\sphinxbfcode{\sphinxupquote{set\_}}}{\sphinxparam{\DUrole{o,o}{**}\DUrole{n,n}{p}}}{}
\pysigstopsignatures
\sphinxAtStartPar
Update parameters, call appropriate functions ensuring
input parameters are self\sphinxhyphen{}consistent.
\begin{quote}\begin{description}
\sphinxlineitem{Parameters}
\sphinxAtStartPar
\sphinxstyleliteralstrong{\sphinxupquote{**p}} \textendash{} Optional keyword parameters to update the calculator.
can be any of {\hyperref[\detokenize{api/pytb.parameters.TBParameters:pytb.parameters.TBParameters}]{\sphinxcrossref{\sphinxcode{\sphinxupquote{TBParameters}}}}} or
{\hyperref[\detokenize{api/pytb.parameters.WrapParameters:pytb.parameters.WrapParameters}]{\sphinxcrossref{\sphinxcode{\sphinxupquote{WrapParameters}}}}}.

\end{description}\end{quote}

\end{fulllineitems}


\sphinxstepscope


\subsubsection{pytb.ThunderBoltz.set\_fixed\_tracking}
\label{\detokenize{api/pytb.ThunderBoltz.set_fixed_tracking:pytb-thunderboltz-set-fixed-tracking}}\label{\detokenize{api/pytb.ThunderBoltz.set_fixed_tracking::doc}}\index{set\_fixed\_tracking() (pytb.ThunderBoltz method)@\spxentry{set\_fixed\_tracking()}\spxextra{pytb.ThunderBoltz method}}

\begin{fulllineitems}
\phantomsection\label{\detokenize{api/pytb.ThunderBoltz.set_fixed_tracking:pytb.ThunderBoltz.set_fixed_tracking}}
\pysigstartsignatures
\pysiglinewithargsret{\sphinxcode{\sphinxupquote{ThunderBoltz.}}\sphinxbfcode{\sphinxupquote{set\_fixed\_tracking}}}{}{}
\pysigstopsignatures
\sphinxAtStartPar
Change all reaction species indices of differing reactant values
to be between only particle 0 and 1 (e.g. 0+1\sphinxhyphen{}\textgreater{}0+2 is changed to 0+1\sphinxhyphen{}\textgreater{}0+1).

\end{fulllineitems}


\sphinxstepscope


\subsubsection{pytb.ThunderBoltz.set\_ts\_plot\_params}
\label{\detokenize{api/pytb.ThunderBoltz.set_ts_plot_params:pytb-thunderboltz-set-ts-plot-params}}\label{\detokenize{api/pytb.ThunderBoltz.set_ts_plot_params::doc}}\index{set\_ts\_plot\_params() (pytb.ThunderBoltz method)@\spxentry{set\_ts\_plot\_params()}\spxextra{pytb.ThunderBoltz method}}

\begin{fulllineitems}
\phantomsection\label{\detokenize{api/pytb.ThunderBoltz.set_ts_plot_params:pytb.ThunderBoltz.set_ts_plot_params}}
\pysigstartsignatures
\pysiglinewithargsret{\sphinxcode{\sphinxupquote{ThunderBoltz.}}\sphinxbfcode{\sphinxupquote{set\_ts\_plot\_params}}}{\sphinxparam{\DUrole{n,n}{params}}}{}
\pysigstopsignatures
\sphinxAtStartPar
Set the default series plotted by
{\hyperref[\detokenize{api/pytb.ThunderBoltz.plot_timeseries:pytb.ThunderBoltz.plot_timeseries}]{\sphinxcrossref{\sphinxcode{\sphinxupquote{plot\_timeseries()}}}}}

\end{fulllineitems}


\sphinxstepscope


\subsubsection{pytb.ThunderBoltz.to\_pickleable}
\label{\detokenize{api/pytb.ThunderBoltz.to_pickleable:pytb-thunderboltz-to-pickleable}}\label{\detokenize{api/pytb.ThunderBoltz.to_pickleable::doc}}\index{to\_pickleable() (pytb.ThunderBoltz method)@\spxentry{to\_pickleable()}\spxextra{pytb.ThunderBoltz method}}

\begin{fulllineitems}
\phantomsection\label{\detokenize{api/pytb.ThunderBoltz.to_pickleable:pytb.ThunderBoltz.to_pickleable}}
\pysigstartsignatures
\pysiglinewithargsret{\sphinxcode{\sphinxupquote{ThunderBoltz.}}\sphinxbfcode{\sphinxupquote{to\_pickleable}}}{}{}
\pysigstopsignatures
\sphinxAtStartPar
Return a picklable version of this object.

\end{fulllineitems}


\sphinxstepscope


\subsubsection{pytb.ThunderBoltz.write\_input}
\label{\detokenize{api/pytb.ThunderBoltz.write_input:pytb-thunderboltz-write-input}}\label{\detokenize{api/pytb.ThunderBoltz.write_input::doc}}\index{write\_input() (pytb.ThunderBoltz method)@\spxentry{write\_input()}\spxextra{pytb.ThunderBoltz method}}

\begin{fulllineitems}
\phantomsection\label{\detokenize{api/pytb.ThunderBoltz.write_input:pytb.ThunderBoltz.write_input}}
\pysigstartsignatures
\pysiglinewithargsret{\sphinxcode{\sphinxupquote{ThunderBoltz.}}\sphinxbfcode{\sphinxupquote{write\_input}}}{\sphinxparam{\DUrole{n,n}{directory}}}{}
\pysigstopsignatures
\sphinxAtStartPar
Write all the input files into a directory with
the current settings.
\begin{quote}\begin{description}
\sphinxlineitem{Parameters}
\sphinxAtStartPar
\sphinxstyleliteralstrong{\sphinxupquote{directory}} (\sphinxhref{https://docs.python.org/3/library/stdtypes.html\#str}{\sphinxstyleliteralemphasis{\sphinxupquote{str}}}) \textendash{} The path to the simulation directory.

\end{description}\end{quote}

\end{fulllineitems}


\end{fulllineitems}



\subsection{Build from Files}
\label{\detokenize{ref:build-from-files}}

\begin{savenotes}\sphinxattablestart
\sphinxthistablewithglobalstyle
\sphinxthistablewithnovlinesstyle
\centering
\begin{tabulary}{\linewidth}[t]{\X{1}{2}\X{1}{2}}
\sphinxtoprule
\sphinxtableatstartofbodyhook
\sphinxAtStartPar
{\hyperref[\detokenize{api/pytb.tb.read:pytb.tb.read}]{\sphinxcrossref{\sphinxcode{\sphinxupquote{tb.read}}}}}(directory{[}, read\_cs\_data{]})
&
\sphinxAtStartPar
Create a ThunderBoltz object by reading from a ThunderBoltz simulation directory.
\\
\sphinxhline
\sphinxAtStartPar
{\hyperref[\detokenize{api/pytb.tb.query_tree:pytb.tb.query_tree}]{\sphinxcrossref{\sphinxcode{\sphinxupquote{tb.query\_tree}}}}}(directory{[}, name\_req, ...{]})
&
\sphinxAtStartPar
Walk a directory tree and search for ThunderBoltz simulation directories to read.
\\
\sphinxhline
\sphinxAtStartPar
{\hyperref[\detokenize{api/pytb.tb.CrossSections.from_LXCat:pytb.tb.CrossSections.from_LXCat}]{\sphinxcrossref{\sphinxcode{\sphinxupquote{tb.CrossSections.from\_LXCat}}}}}(fname)
&
\sphinxAtStartPar
Load cross section data from an LXCat .txt file
\\
\sphinxbottomrule
\end{tabulary}
\sphinxtableafterendhook\par
\sphinxattableend\end{savenotes}

\sphinxstepscope


\subsubsection{pytb.tb.read}
\label{\detokenize{api/pytb.tb.read:pytb-tb-read}}\label{\detokenize{api/pytb.tb.read::doc}}\index{read() (in module pytb.tb)@\spxentry{read()}\spxextra{in module pytb.tb}}

\begin{fulllineitems}
\phantomsection\label{\detokenize{api/pytb.tb.read:pytb.tb.read}}
\pysigstartsignatures
\pysiglinewithargsret{\sphinxcode{\sphinxupquote{pytb.tb.}}\sphinxbfcode{\sphinxupquote{read}}}{\sphinxparam{\DUrole{n,n}{directory}}, \sphinxparam{\DUrole{n,n}{read\_cs\_data}\DUrole{o,o}{=}\DUrole{default_value}{False}}}{}
\pysigstopsignatures
\sphinxAtStartPar
Create a ThunderBoltz object by reading from a
ThunderBoltz simulation directory.
\begin{quote}\begin{description}
\sphinxlineitem{Parameters}\begin{itemize}
\item {} 
\sphinxAtStartPar
\sphinxstyleliteralstrong{\sphinxupquote{directory}} (\sphinxhref{https://docs.python.org/3/library/stdtypes.html\#str}{\sphinxstyleliteralemphasis{\sphinxupquote{str}}}) \textendash{} The directory from which to
initialize the ThunderBoltz object.

\item {} 
\sphinxAtStartPar
\sphinxstyleliteralstrong{\sphinxupquote{read\_cs\_data}} (\sphinxhref{https://docs.python.org/3/library/functions.html\#bool}{\sphinxstyleliteralemphasis{\sphinxupquote{bool}}}) \textendash{} When set to true, the reader will look for cs\_data,
default is \sphinxcode{\sphinxupquote{False}}

\end{itemize}

\sphinxlineitem{Returns}
\sphinxAtStartPar
The ThunderBoltz object with
tabulated data if available.

\sphinxlineitem{Return type}
\sphinxAtStartPar
{\hyperref[\detokenize{api/pytb.ThunderBoltz:pytb.ThunderBoltz}]{\sphinxcrossref{\sphinxcode{\sphinxupquote{ThunderBoltz}}}}}

\end{description}\end{quote}

\end{fulllineitems}


\sphinxstepscope


\subsubsection{pytb.tb.query\_tree}
\label{\detokenize{api/pytb.tb.query_tree:pytb-tb-query-tree}}\label{\detokenize{api/pytb.tb.query_tree::doc}}\index{query\_tree() (in module pytb.tb)@\spxentry{query\_tree()}\spxextra{in module pytb.tb}}

\begin{fulllineitems}
\phantomsection\label{\detokenize{api/pytb.tb.query_tree:pytb.tb.query_tree}}
\pysigstartsignatures
\pysiglinewithargsret{\sphinxcode{\sphinxupquote{pytb.tb.}}\sphinxbfcode{\sphinxupquote{query\_tree}}}{\sphinxparam{\DUrole{n,n}{directory}}, \sphinxparam{\DUrole{n,n}{name\_req}\DUrole{o,o}{=}\DUrole{default_value}{None}}, \sphinxparam{\DUrole{n,n}{param\_req}\DUrole{o,o}{=}\DUrole{default_value}{None}}, \sphinxparam{\DUrole{n,n}{read\_cs\_data}\DUrole{o,o}{=}\DUrole{default_value}{False}}, \sphinxparam{\DUrole{n,n}{callback}\DUrole{o,o}{=}\DUrole{default_value}{None}}, \sphinxparam{\DUrole{n,n}{agg}\DUrole{o,o}{=}\DUrole{default_value}{True}}}{}
\pysigstopsignatures
\sphinxAtStartPar
Walk a directory tree and search for ThunderBoltz
simulation directories to read. Either return a list of
{\hyperref[\detokenize{api/pytb.ThunderBoltz:pytb.ThunderBoltz}]{\sphinxcrossref{\sphinxcode{\sphinxupquote{ThunderBoltz}}}}} objects, or a custom aggregation of
the output data.
\begin{quote}\begin{description}
\sphinxlineitem{Parameters}\begin{itemize}
\item {} 
\sphinxAtStartPar
\sphinxstyleliteralstrong{\sphinxupquote{directory}} (\sphinxhref{https://docs.python.org/3/library/stdtypes.html\#str}{\sphinxstyleliteralemphasis{\sphinxupquote{str}}}) \textendash{} The root path to search for ThunderBoltz
data in.

\item {} 
\sphinxAtStartPar
\sphinxstyleliteralstrong{\sphinxupquote{name\_req}} (\sphinxstyleliteralemphasis{\sphinxupquote{callable}}\sphinxstyleliteralemphasis{\sphinxupquote{{[}}}\sphinxhref{https://docs.python.org/3/library/stdtypes.html\#str}{\sphinxstyleliteralemphasis{\sphinxupquote{str}}}\sphinxstyleliteralemphasis{\sphinxupquote{,}}\sphinxhref{https://docs.python.org/3/library/functions.html\#bool}{\sphinxstyleliteralemphasis{\sphinxupquote{bool}}}\sphinxstyleliteralemphasis{\sphinxupquote{{]}}}) \textendash{} 
\sphinxAtStartPar
A requirement on the file path
names to be included in the query. The callable accepts
the file path of a thunderboltz simulation directory and
should return \sphinxcode{\sphinxupquote{True}} if that directory is to be included
in the query.

\sphinxAtStartPar
e.g. \sphinxcode{\sphinxupquote{name\_req=lambda s: "test\_type\_1" in s}}
would return only data in a subfolder \sphinxcode{\sphinxupquote{test\_type\_1}}.


\item {} 
\sphinxAtStartPar
\sphinxstyleliteralstrong{\sphinxupquote{param\_req}} (\sphinxhref{https://docs.python.org/3/library/stdtypes.html\#dict}{\sphinxstyleliteralemphasis{\sphinxupquote{dict}}}) \textendash{} 
\sphinxAtStartPar
A requirement on the
parameter settings of the ThunderBoltz calculations.
The dictionary corresponding to simulation parameters
that must be set by the read ThunderBoltz object.

\sphinxAtStartPar
e.g. \sphinxcode{\sphinxupquote{param\_req=\{"Ered": 100, "L": 1e\sphinxhyphen{}6\}}} would only
return data from calculations with a reduced field of
100 Td and a cell length of 1 \(\mu{\rm m}\).


\item {} 
\sphinxAtStartPar
\sphinxstyleliteralstrong{\sphinxupquote{callback}} \textendash{} (callable{[}{\hyperref[\detokenize{api/pytb.ThunderBoltz:pytb.ThunderBoltz}]{\sphinxcrossref{\sphinxcode{\sphinxupquote{ThunderBoltz}}}}}, Any{]}):
A function that accepts a ThunderBoltz object and
returns the desired data.

\item {} 
\sphinxAtStartPar
\sphinxstyleliteralstrong{\sphinxupquote{agg}} \textendash{} 
\sphinxAtStartPar
If \sphinxcode{\sphinxupquote{callback}} is set, attempt to aggregate the data
based on the data type:


\begin{savenotes}\sphinxattablestart
\sphinxthistablewithglobalstyle
\centering
\begin{tabulary}{\linewidth}[t]{TT}
\sphinxtoprule
\sphinxstyletheadfamily 
\sphinxAtStartPar
callback Return Type
&\sphinxstyletheadfamily 
\sphinxAtStartPar
Behavior
\\
\sphinxmidrule
\sphinxtableatstartofbodyhook
\sphinxAtStartPar
\sphinxhref{http://pandas.pydata.org/pandas-docs/dev/reference/api/pandas.DataFrame.html\#pandas.DataFrame}{\sphinxcode{\sphinxupquote{pandas.DataFrame}}}
&
\sphinxAtStartPar
Frames will be
concatenated row\sphinxhyphen{}wise and
one larger DataFrame will be
returned.
\\
\sphinxhline
\sphinxAtStartPar
list{[}\sphinxhref{http://pandas.pydata.org/pandas-docs/dev/reference/api/pandas.DataFrame.html\#pandas.DataFrame}{\sphinxcode{\sphinxupquote{pandas.DataFrame}}}{]}
&
\sphinxAtStartPar
A list of frames the same
length of the return value
will be returned. The frame
at index \sphinxcode{\sphinxupquote{i}} will contain
the concatenated data
from each simulation returned
by \sphinxcode{\sphinxupquote{callable(tb){[}i{]}}}.
\\
\sphinxhline
\sphinxAtStartPar
list{[}Any{]}
&
\sphinxAtStartPar
A list of lists will be
returned. The list at index
\sphinxcode{\sphinxupquote{i}} will contain a list
of items returned from each
call to \sphinxcode{\sphinxupquote{callable(tb){[}i{]}}}.
\\
\sphinxhline
\sphinxAtStartPar
Any
&
\sphinxAtStartPar
Return values will be
returned in a list.
\\
\sphinxbottomrule
\end{tabulary}
\sphinxtableafterendhook\par
\sphinxattableend\end{savenotes}

\sphinxAtStartPar
If \sphinxcode{\sphinxupquote{agg}} is set to False, always return a list of callback data
without any concatenation.


\end{itemize}

\sphinxlineitem{Returns}
\sphinxAtStartPar
See \sphinxcode{\sphinxupquote{agg}} option for behavior. Default return type
is list{[}{\hyperref[\detokenize{api/pytb.ThunderBoltz:pytb.ThunderBoltz}]{\sphinxcrossref{\sphinxcode{\sphinxupquote{ThunderBoltz}}}}}{]}.

\sphinxlineitem{Return type}
\sphinxAtStartPar
list{[}{\hyperref[\detokenize{api/pytb.ThunderBoltz:pytb.ThunderBoltz}]{\sphinxcrossref{\sphinxcode{\sphinxupquote{ThunderBoltz}}}}}{]}, or \sphinxhref{http://pandas.pydata.org/pandas-docs/dev/reference/api/pandas.DataFrame.html\#pandas.DataFrame}{\sphinxcode{\sphinxupquote{pandas.DataFrame}}}, or list{[}\sphinxhref{http://pandas.pydata.org/pandas-docs/dev/reference/api/pandas.DataFrame.html\#pandas.DataFrame}{\sphinxcode{\sphinxupquote{pandas.DataFrame}}}{]}, or list{[}list{[}Any{]}{]}

\end{description}\end{quote}

\end{fulllineitems}


\sphinxstepscope


\subsubsection{pytb.tb.CrossSections.from\_LXCat}
\label{\detokenize{api/pytb.tb.CrossSections.from_LXCat:pytb-tb-crosssections-from-lxcat}}\label{\detokenize{api/pytb.tb.CrossSections.from_LXCat::doc}}\index{from\_LXCat() (pytb.tb.CrossSections method)@\spxentry{from\_LXCat()}\spxextra{pytb.tb.CrossSections method}}

\begin{fulllineitems}
\phantomsection\label{\detokenize{api/pytb.tb.CrossSections.from_LXCat:pytb.tb.CrossSections.from_LXCat}}
\pysigstartsignatures
\pysiglinewithargsret{\sphinxcode{\sphinxupquote{CrossSections.}}\sphinxbfcode{\sphinxupquote{from\_LXCat}}}{\sphinxparam{\DUrole{n,n}{fname}}}{}
\pysigstopsignatures
\sphinxAtStartPar
Load cross section data from an LXCat .txt file

\end{fulllineitems}



\subsection{Cross Sections}
\label{\detokenize{ref:cross-sections}}

\begin{savenotes}\sphinxattablestart
\sphinxthistablewithglobalstyle
\sphinxthistablewithnovlinesstyle
\centering
\begin{tabulary}{\linewidth}[t]{\X{1}{2}\X{1}{2}}
\sphinxtoprule
\sphinxtableatstartofbodyhook
\sphinxAtStartPar
{\hyperref[\detokenize{api/pytb.CrossSections:pytb.CrossSections}]{\sphinxcrossref{\sphinxcode{\sphinxupquote{CrossSections}}}}}({[}directory, input\_path, ...{]})
&
\sphinxAtStartPar
ThunderBoltz cross section set data type.
\\
\sphinxhline
\sphinxAtStartPar
{\hyperref[\detokenize{api/pytb.Process:pytb.Process}]{\sphinxcrossref{\sphinxcode{\sphinxupquote{Process}}}}}(process\_type{[}, r1, r2, p1, p2, ...{]})
&
\sphinxAtStartPar
A reaction process determined by reaction and product indices, a process type, a potential threshold, and a corresponding cross section specification.
\\
\sphinxhline
\sphinxAtStartPar
{\hyperref[\detokenize{api/pytb.input.He_TB:pytb.input.He_TB}]{\sphinxcrossref{\sphinxcode{\sphinxupquote{input.He\_TB}}}}}({[}n, egen, analytic\_cs, eadf, ...{]})
&
\sphinxAtStartPar
Generate parameterized He cross section sets in the ThunderBoltz format.
\\
\sphinxhline
\sphinxAtStartPar
{\hyperref[\detokenize{api/pytb.input.convert:pytb.input.convert}]{\sphinxcrossref{\sphinxcode{\sphinxupquote{input.convert}}}}}(df, u1, u2{[}, inv, drop, add{]})
&
\sphinxAtStartPar
Convert easily between units with a labeled DataFrame.
\\
\sphinxbottomrule
\end{tabulary}
\sphinxtableafterendhook\par
\sphinxattableend\end{savenotes}

\sphinxstepscope


\subsubsection{pytb.CrossSections}
\label{\detokenize{api/pytb.CrossSections:pytb-crosssections}}\label{\detokenize{api/pytb.CrossSections::doc}}\index{CrossSections (class in pytb)@\spxentry{CrossSections}\spxextra{class in pytb}}

\begin{fulllineitems}
\phantomsection\label{\detokenize{api/pytb.CrossSections:pytb.CrossSections}}
\pysigstartsignatures
\pysiglinewithargsret{\sphinxbfcode{\sphinxupquote{class\DUrole{w,w}{  }}}\sphinxcode{\sphinxupquote{pytb.}}\sphinxbfcode{\sphinxupquote{CrossSections}}}{\sphinxparam{\DUrole{n,n}{directory}\DUrole{o,o}{=}\DUrole{default_value}{None}}, \sphinxparam{\DUrole{n,n}{input\_path}\DUrole{o,o}{=}\DUrole{default_value}{None}}, \sphinxparam{\DUrole{n,n}{read\_cs\_data}\DUrole{o,o}{=}\DUrole{default_value}{True}}, \sphinxparam{\DUrole{n,n}{cs\_dir\_name}\DUrole{o,o}{=}\DUrole{default_value}{\textquotesingle{}cross\_sections\textquotesingle{}}}, \sphinxparam{\DUrole{n,n}{input\_fname}\DUrole{o,o}{=}\DUrole{default_value}{None}}}{}
\pysigstopsignatures
\sphinxAtStartPar
ThunderBoltz cross section set data type. Consists of
a set of cross sections each with a file reference and a
reaction table.
\begin{quote}\begin{description}
\sphinxlineitem{Parameters}\begin{itemize}
\item {} 
\sphinxAtStartPar
\sphinxstyleliteralstrong{\sphinxupquote{directory}} (\sphinxhref{https://docs.python.org/3/library/stdtypes.html\#str}{\sphinxstyleliteralemphasis{\sphinxupquote{str}}}) \textendash{} The path to a ThunderBoltz simulation
directory in which input files are to be written.
Default is \sphinxcode{\sphinxupquote{None}}.

\item {} 
\sphinxAtStartPar
\sphinxstyleliteralstrong{\sphinxupquote{input\_path}} (\sphinxhref{https://docs.python.org/3/library/stdtypes.html\#str}{\sphinxstyleliteralemphasis{\sphinxupquote{str}}}) \textendash{} 
\sphinxAtStartPar
The path to a set of ThunderBoltz
input files from which input data can be read.
The file structure should be something like:

\begin{sphinxVerbatim}[commandchars=\\\{\}]
path/to/input\PYGZus{}path
 |———indeck\PYGZus{}file.in  \PYGZlt{}— The main ThunderBoltz indeck file.
 |———cross\PYGZus{}sections  \PYGZlt{}— Cross section directory
 |   |———cs1.dat     \PYGZlt{}— ThunderBoltz\PYGZhy{}formatted cross section file.
 |   |———cs2.dat     \PYGZlt{}— ThunderBoltz\PYGZhy{}formatted cross section file.
 |   ...
 ...
\end{sphinxVerbatim}


\item {} 
\sphinxAtStartPar
\sphinxstyleliteralstrong{\sphinxupquote{cs\_dir\_name}} (\sphinxhref{https://docs.python.org/3/library/stdtypes.html\#str}{\sphinxstyleliteralemphasis{\sphinxupquote{str}}}) \textendash{} The name of the cross section directory.

\item {} 
\sphinxAtStartPar
\sphinxstyleliteralstrong{\sphinxupquote{input\_fname}} (\sphinxhref{https://docs.python.org/3/library/stdtypes.html\#str}{\sphinxstyleliteralemphasis{\sphinxupquote{str}}}) \textendash{} The name of the main ThunderBoltz indeck
file. Default is None, in which case the indeck will be
searched for in \sphinxtitleref{input\_path} and must end with \sphinxtitleref{.in}.

\end{itemize}

\end{description}\end{quote}
\subsubsection*{Attributes}


\begin{savenotes}\sphinxattablestart
\sphinxthistablewithglobalstyle
\sphinxthistablewithnovlinesstyle
\centering
\begin{tabulary}{\linewidth}[t]{\X{1}{2}\X{1}{2}}
\sphinxtoprule
\sphinxtableatstartofbodyhook
\sphinxAtStartPar
\sphinxcode{\sphinxupquote{cs\_dir\_name}}
&
\sphinxAtStartPar
Place for cs files in simulation dir
\\
\sphinxhline
\sphinxAtStartPar
\sphinxcode{\sphinxupquote{input\_fname}}
&
\sphinxAtStartPar
Input deck filename default
\\
\sphinxhline
\sphinxAtStartPar
\sphinxcode{\sphinxupquote{table}}
&
\sphinxAtStartPar
The reaction table, with columns
\\
\sphinxhline
\sphinxAtStartPar
\sphinxcode{\sphinxupquote{data}}
&
\sphinxAtStartPar
Data tables for each cross section.
\\
\sphinxhline
\sphinxAtStartPar
\sphinxcode{\sphinxupquote{input\_path}}
&
\sphinxAtStartPar
Input path to default input data
\\
\sphinxbottomrule
\end{tabulary}
\sphinxtableafterendhook\par
\sphinxattableend\end{savenotes}
\subsubsection*{Methods}


\begin{savenotes}\sphinxattablestart
\sphinxthistablewithglobalstyle
\sphinxthistablewithnovlinesstyle
\centering
\begin{tabulary}{\linewidth}[t]{\X{1}{2}\X{1}{2}}
\sphinxtoprule
\sphinxtableatstartofbodyhook
\sphinxAtStartPar
{\hyperref[\detokenize{api/pytb.CrossSections.add_differential_model:pytb.CrossSections.add_differential_model}]{\sphinxcrossref{\sphinxcode{\sphinxupquote{add\_differential\_model}}}}}(rtype, name{[}, params{]})
&
\sphinxAtStartPar
Add a differential model to a certain type of process.
\\
\sphinxhline
\sphinxAtStartPar
{\hyperref[\detokenize{api/pytb.CrossSections.add_process:pytb.CrossSections.add_process}]{\sphinxcrossref{\sphinxcode{\sphinxupquote{add\_process}}}}}(p)
&
\sphinxAtStartPar
Take a Process object and update the cross section data and cross section reaction table.
\\
\sphinxhline
\sphinxAtStartPar
{\hyperref[\detokenize{api/pytb.CrossSections.add_processes:pytb.CrossSections.add_processes}]{\sphinxcrossref{\sphinxcode{\sphinxupquote{add\_processes}}}}}(ps)
&
\sphinxAtStartPar
Add multiple cross sections to the reaction table.
\\
\sphinxhline
\sphinxAtStartPar
{\hyperref[\detokenize{api/pytb.CrossSections.find_infile:pytb.CrossSections.find_infile}]{\sphinxcrossref{\sphinxcode{\sphinxupquote{find\_infile}}}}}()
&
\sphinxAtStartPar
Look for indeck file in input\_path.
\\
\sphinxhline
\sphinxAtStartPar
{\hyperref[\detokenize{api/pytb.CrossSections.from_LXCat:pytb.CrossSections.from_LXCat}]{\sphinxcrossref{\sphinxcode{\sphinxupquote{from\_LXCat}}}}}(fname)
&
\sphinxAtStartPar
Load cross section data from an LXCat .txt file
\\
\sphinxhline
\sphinxAtStartPar
{\hyperref[\detokenize{api/pytb.CrossSections.get_deck:pytb.CrossSections.get_deck}]{\sphinxcrossref{\sphinxcode{\sphinxupquote{get\_deck}}}}}()
&
\sphinxAtStartPar
Return the string formatted cross section table portion of the ThunderBoltz indeck.
\\
\sphinxhline
\sphinxAtStartPar
{\hyperref[\detokenize{api/pytb.CrossSections.plot_cs:pytb.CrossSections.plot_cs}]{\sphinxcrossref{\sphinxcode{\sphinxupquote{plot\_cs}}}}}({[}ax, legend, vsig, thresholds{]})
&
\sphinxAtStartPar
Plot the cross sections models.
\\
\sphinxhline
\sphinxAtStartPar
{\hyperref[\detokenize{api/pytb.CrossSections.read:pytb.CrossSections.read}]{\sphinxcrossref{\sphinxcode{\sphinxupquote{read}}}}}(input\_path{[}, read\_cs\_data{]})
&
\sphinxAtStartPar
Read ThunderBoltz cross section data from a directory with a single input file and a set of cross section files.
\\
\sphinxhline
\sphinxAtStartPar
{\hyperref[\detokenize{api/pytb.CrossSections.set_fixed_background:pytb.CrossSections.set_fixed_background}]{\sphinxcrossref{\sphinxcode{\sphinxupquote{set\_fixed\_background}}}}}({[}fixed{]})
&
\sphinxAtStartPar
Set all particle conserving processes to have the \sphinxtitleref{FixedParticle2} tag or not.
\\
\sphinxhline
\sphinxAtStartPar
{\hyperref[\detokenize{api/pytb.CrossSections.write:pytb.CrossSections.write}]{\sphinxcrossref{\sphinxcode{\sphinxupquote{write}}}}}({[}directory{]})
&
\sphinxAtStartPar
Write cross section files into the simulation cross section directory.
\\
\sphinxbottomrule
\end{tabulary}
\sphinxtableafterendhook\par
\sphinxattableend\end{savenotes}

\sphinxstepscope


\paragraph{pytb.CrossSections.add\_differential\_model}
\label{\detokenize{api/pytb.CrossSections.add_differential_model:pytb-crosssections-add-differential-model}}\label{\detokenize{api/pytb.CrossSections.add_differential_model::doc}}\index{add\_differential\_model() (pytb.CrossSections method)@\spxentry{add\_differential\_model()}\spxextra{pytb.CrossSections method}}

\begin{fulllineitems}
\phantomsection\label{\detokenize{api/pytb.CrossSections.add_differential_model:pytb.CrossSections.add_differential_model}}
\pysigstartsignatures
\pysiglinewithargsret{\sphinxcode{\sphinxupquote{CrossSections.}}\sphinxbfcode{\sphinxupquote{add\_differential\_model}}}{\sphinxparam{\DUrole{n,n}{rtype}}, \sphinxparam{\DUrole{n,n}{name}}, \sphinxparam{\DUrole{n,n}{params}\DUrole{o,o}{=}\DUrole{default_value}{None}}}{}
\pysigstopsignatures
\sphinxAtStartPar
Add a differential model to a certain type of process.
\begin{quote}\begin{description}
\sphinxlineitem{Parameters}\begin{itemize}
\item {} 
\sphinxAtStartPar
\sphinxstyleliteralstrong{\sphinxupquote{rtype}} (\sphinxhref{https://docs.python.org/3/library/stdtypes.html\#str}{\sphinxstyleliteralemphasis{\sphinxupquote{str}}}) \textendash{} “Elastic”, “Inelastic”, or “Ionization”,
the broad collision process type.

\item {} 
\sphinxAtStartPar
\sphinxstyleliteralstrong{\sphinxupquote{name}} (\sphinxhref{https://docs.python.org/3/library/stdtypes.html\#str}{\sphinxstyleliteralemphasis{\sphinxupquote{str}}}) \textendash{} 
\sphinxAtStartPar
The name of the differential process model.
Available built\sphinxhyphen{}in options for each \sphinxcode{\sphinxupquote{rtype}} are:


\begin{savenotes}\sphinxattablestart
\sphinxthistablewithglobalstyle
\centering
\begin{tabulary}{\linewidth}[t]{TT}
\sphinxtoprule
\sphinxstyletheadfamily 
\sphinxAtStartPar
\sphinxcode{\sphinxupquote{rtype}}
&\sphinxstyletheadfamily 
\sphinxAtStartPar
\sphinxcode{\sphinxupquote{name}}
\\
\sphinxmidrule
\sphinxtableatstartofbodyhook
\sphinxAtStartPar
Elastic
&
\sphinxAtStartPar
Park, Murphy
\\
\sphinxhline
\sphinxAtStartPar
Ionization
&
\sphinxAtStartPar
Equal, Uniform
\\
\sphinxbottomrule
\end{tabulary}
\sphinxtableafterendhook\par
\sphinxattableend\end{savenotes}


\item {} 
\sphinxAtStartPar
\sphinxstyleliteralstrong{\sphinxupquote{params}} (\sphinxhref{https://docs.python.org/3/library/stdtypes.html\#list}{\sphinxstyleliteralemphasis{\sphinxupquote{list}}}\sphinxstyleliteralemphasis{\sphinxupquote{{[}}}\sphinxhref{https://docs.python.org/3/library/functions.html\#float}{\sphinxstyleliteralemphasis{\sphinxupquote{float}}}\sphinxstyleliteralemphasis{\sphinxupquote{{]}}}) \textendash{} Optional list of parameters required
by the differential model.

\end{itemize}

\end{description}\end{quote}

\end{fulllineitems}


\sphinxstepscope


\paragraph{pytb.CrossSections.add\_process}
\label{\detokenize{api/pytb.CrossSections.add_process:pytb-crosssections-add-process}}\label{\detokenize{api/pytb.CrossSections.add_process::doc}}\index{add\_process() (pytb.CrossSections method)@\spxentry{add\_process()}\spxextra{pytb.CrossSections method}}

\begin{fulllineitems}
\phantomsection\label{\detokenize{api/pytb.CrossSections.add_process:pytb.CrossSections.add_process}}
\pysigstartsignatures
\pysiglinewithargsret{\sphinxcode{\sphinxupquote{CrossSections.}}\sphinxbfcode{\sphinxupquote{add\_process}}}{\sphinxparam{\DUrole{n,n}{p}}}{}
\pysigstopsignatures
\sphinxAtStartPar
Take a Process object and update the cross
section data and cross section reaction table.
\begin{quote}\begin{description}
\sphinxlineitem{Parameters}
\sphinxAtStartPar
\sphinxstyleliteralstrong{\sphinxupquote{p}} ({\hyperref[\detokenize{api/pytb.Process:pytb.Process}]{\sphinxcrossref{\sphinxstyleliteralemphasis{\sphinxupquote{Process}}}}}) \textendash{} The process object for a single
type of interaction.

\end{description}\end{quote}

\end{fulllineitems}


\sphinxstepscope


\paragraph{pytb.CrossSections.add\_processes}
\label{\detokenize{api/pytb.CrossSections.add_processes:pytb-crosssections-add-processes}}\label{\detokenize{api/pytb.CrossSections.add_processes::doc}}\index{add\_processes() (pytb.CrossSections method)@\spxentry{add\_processes()}\spxextra{pytb.CrossSections method}}

\begin{fulllineitems}
\phantomsection\label{\detokenize{api/pytb.CrossSections.add_processes:pytb.CrossSections.add_processes}}
\pysigstartsignatures
\pysiglinewithargsret{\sphinxcode{\sphinxupquote{CrossSections.}}\sphinxbfcode{\sphinxupquote{add\_processes}}}{\sphinxparam{\DUrole{n,n}{ps}}}{}
\pysigstopsignatures
\sphinxAtStartPar
Add multiple cross sections to the reaction table.
\begin{quote}\begin{description}
\sphinxlineitem{Parameters}
\sphinxAtStartPar
\sphinxstyleliteralstrong{\sphinxupquote{ps}} (\sphinxhref{https://docs.python.org/3/library/stdtypes.html\#list}{\sphinxstyleliteralemphasis{\sphinxupquote{list}}}\sphinxstyleliteralemphasis{\sphinxupquote{{[}}}{\hyperref[\detokenize{api/pytb.Process:pytb.Process}]{\sphinxcrossref{\sphinxstyleliteralemphasis{\sphinxupquote{Process}}}}}\sphinxstyleliteralemphasis{\sphinxupquote{{]}}}) \textendash{} A list of process objects to add.

\end{description}\end{quote}

\end{fulllineitems}


\sphinxstepscope


\paragraph{pytb.CrossSections.find\_infile}
\label{\detokenize{api/pytb.CrossSections.find_infile:pytb-crosssections-find-infile}}\label{\detokenize{api/pytb.CrossSections.find_infile::doc}}\index{find\_infile() (pytb.CrossSections method)@\spxentry{find\_infile()}\spxextra{pytb.CrossSections method}}

\begin{fulllineitems}
\phantomsection\label{\detokenize{api/pytb.CrossSections.find_infile:pytb.CrossSections.find_infile}}
\pysigstartsignatures
\pysiglinewithargsret{\sphinxcode{\sphinxupquote{CrossSections.}}\sphinxbfcode{\sphinxupquote{find\_infile}}}{}{}
\pysigstopsignatures
\sphinxAtStartPar
Look for indeck file in input\_path.
\begin{quote}\begin{description}
\sphinxlineitem{Raises}\begin{itemize}
\item {} 
\sphinxAtStartPar
\sphinxhref{https://docs.python.org/3/library/exceptions.html\#RuntimeError}{\sphinxstyleliteralstrong{\sphinxupquote{RuntimeError}}} \textendash{} If input\_path is not set, if multiple indecks

\item {} 
\sphinxAtStartPar
\sphinxstyleliteralstrong{\sphinxupquote{are found}}\sphinxstyleliteralstrong{\sphinxupquote{, or }}\sphinxstyleliteralstrong{\sphinxupquote{if no indecks are found.}} \textendash{} 

\end{itemize}

\end{description}\end{quote}

\end{fulllineitems}


\sphinxstepscope


\paragraph{pytb.CrossSections.from\_LXCat}
\label{\detokenize{api/pytb.CrossSections.from_LXCat:pytb-crosssections-from-lxcat}}\label{\detokenize{api/pytb.CrossSections.from_LXCat::doc}}\index{from\_LXCat() (pytb.CrossSections method)@\spxentry{from\_LXCat()}\spxextra{pytb.CrossSections method}}

\begin{fulllineitems}
\phantomsection\label{\detokenize{api/pytb.CrossSections.from_LXCat:pytb.CrossSections.from_LXCat}}
\pysigstartsignatures
\pysiglinewithargsret{\sphinxcode{\sphinxupquote{CrossSections.}}\sphinxbfcode{\sphinxupquote{from\_LXCat}}}{\sphinxparam{\DUrole{n,n}{fname}}}{}
\pysigstopsignatures
\sphinxAtStartPar
Load cross section data from an LXCat .txt file

\end{fulllineitems}


\sphinxstepscope


\paragraph{pytb.CrossSections.get\_deck}
\label{\detokenize{api/pytb.CrossSections.get_deck:pytb-crosssections-get-deck}}\label{\detokenize{api/pytb.CrossSections.get_deck::doc}}\index{get\_deck() (pytb.CrossSections method)@\spxentry{get\_deck()}\spxextra{pytb.CrossSections method}}

\begin{fulllineitems}
\phantomsection\label{\detokenize{api/pytb.CrossSections.get_deck:pytb.CrossSections.get_deck}}
\pysigstartsignatures
\pysiglinewithargsret{\sphinxcode{\sphinxupquote{CrossSections.}}\sphinxbfcode{\sphinxupquote{get\_deck}}}{}{}
\pysigstopsignatures
\sphinxAtStartPar
Return the string formatted cross section table portion of the
ThunderBoltz indeck.

\end{fulllineitems}


\sphinxstepscope


\paragraph{pytb.CrossSections.plot\_cs}
\label{\detokenize{api/pytb.CrossSections.plot_cs:pytb-crosssections-plot-cs}}\label{\detokenize{api/pytb.CrossSections.plot_cs::doc}}\index{plot\_cs() (pytb.CrossSections method)@\spxentry{plot\_cs()}\spxextra{pytb.CrossSections method}}

\begin{fulllineitems}
\phantomsection\label{\detokenize{api/pytb.CrossSections.plot_cs:pytb.CrossSections.plot_cs}}
\pysigstartsignatures
\pysiglinewithargsret{\sphinxcode{\sphinxupquote{CrossSections.}}\sphinxbfcode{\sphinxupquote{plot\_cs}}}{\sphinxparam{\DUrole{n,n}{ax}\DUrole{o,o}{=}\DUrole{default_value}{None}}, \sphinxparam{\DUrole{n,n}{legend}\DUrole{o,o}{=}\DUrole{default_value}{True}}, \sphinxparam{\DUrole{n,n}{vsig}\DUrole{o,o}{=}\DUrole{default_value}{False}}, \sphinxparam{\DUrole{n,n}{thresholds}\DUrole{o,o}{=}\DUrole{default_value}{False}}, \sphinxparam{\DUrole{o,o}{**}\DUrole{n,n}{plot\_args}}}{}
\pysigstopsignatures
\sphinxAtStartPar
Plot the cross sections models.
\begin{quote}\begin{description}
\sphinxlineitem{Parameters}\begin{itemize}
\item {} 
\sphinxAtStartPar
\sphinxstyleliteralstrong{\sphinxupquote{ax}} (\sphinxhref{https://matplotlib.org/stable/api/\_as\_gen/matplotlib.axes.Axes.html\#matplotlib.axes.Axes}{\sphinxcode{\sphinxupquote{Axes}}} or None) \textendash{} Optional
axes object to plot on top of, default is \sphinxcode{\sphinxupquote{None}}.
If ax is \sphinxcode{\sphinxupquote{None}}, then a new figure and Axes object
will be created.

\item {} 
\sphinxAtStartPar
\sphinxstyleliteralstrong{\sphinxupquote{legend}} (\sphinxhref{https://docs.python.org/3/library/functions.html\#bool}{\sphinxstyleliteralemphasis{\sphinxupquote{bool}}}\sphinxstyleliteralemphasis{\sphinxupquote{ or }}\sphinxhref{https://docs.python.org/3/library/stdtypes.html\#dict}{\sphinxstyleliteralemphasis{\sphinxupquote{dict}}}) \textendash{} Activate axes legend if true,
default is True. If a dictionary is passed, it is
interpreted as arguments to \sphinxhref{https://matplotlib.org/stable/api/\_as\_gen/matplotlib.axes.Axes.legend.html\#matplotlib.axes.Axes.legend}{\sphinxcode{\sphinxupquote{legend()}}}.

\item {} 
\sphinxAtStartPar
\sphinxstyleliteralstrong{\sphinxupquote{vsig}} (\sphinxhref{https://docs.python.org/3/library/functions.html\#bool}{\sphinxstyleliteralemphasis{\sphinxupquote{bool}}}) \textendash{} Plot \(\sqrt{\frac{2\epsilon}{m_{\rm e}}}\sigma(\epsilon)\)
rather than \(\sigma(\epsilon)\) on the y\sphinxhyphen{}axis.

\item {} 
\sphinxAtStartPar
\sphinxstyleliteralstrong{\sphinxupquote{thresholds}} (\sphinxhref{https://docs.python.org/3/library/functions.html\#bool}{\sphinxstyleliteralemphasis{\sphinxupquote{bool}}}) \textendash{} if \sphinxcode{\sphinxupquote{True}}, plot
\(\frac{\epsilon}{\epsilon_{\rm ion}} - 1\)
rather than \(\epsilon\) on the x\sphinxhyphen{}axis.

\item {} 
\sphinxAtStartPar
\sphinxstyleliteralstrong{\sphinxupquote{**plot\_args}} \textendash{} Optional arguments passed to Axes.plot().

\end{itemize}

\sphinxlineitem{Returns}
\sphinxAtStartPar
The axes object.

\sphinxlineitem{Return type}
\sphinxAtStartPar
ax (\sphinxhref{https://matplotlib.org/stable/api/\_as\_gen/matplotlib.axes.Axes.html\#matplotlib.axes.Axes}{matplotlib.axes.Axes})

\end{description}\end{quote}

\end{fulllineitems}


\sphinxstepscope


\paragraph{pytb.CrossSections.read}
\label{\detokenize{api/pytb.CrossSections.read:pytb-crosssections-read}}\label{\detokenize{api/pytb.CrossSections.read::doc}}\index{read() (pytb.CrossSections method)@\spxentry{read()}\spxextra{pytb.CrossSections method}}

\begin{fulllineitems}
\phantomsection\label{\detokenize{api/pytb.CrossSections.read:pytb.CrossSections.read}}
\pysigstartsignatures
\pysiglinewithargsret{\sphinxcode{\sphinxupquote{CrossSections.}}\sphinxbfcode{\sphinxupquote{read}}}{\sphinxparam{\DUrole{n,n}{input\_path}}, \sphinxparam{\DUrole{n,n}{read\_cs\_data}\DUrole{o,o}{=}\DUrole{default_value}{True}}}{}
\pysigstopsignatures
\sphinxAtStartPar
Read ThunderBoltz cross section data from a directory with a
single input file and a set of cross section files.
\begin{quote}\begin{description}
\sphinxlineitem{Parameters}\begin{itemize}
\item {} 
\sphinxAtStartPar
\sphinxstyleliteralstrong{\sphinxupquote{input\_path}} (\sphinxhref{https://docs.python.org/3/library/stdtypes.html\#str}{\sphinxstyleliteralemphasis{\sphinxupquote{str}}}) \textendash{} The path to the directory
with cross section data. If specified, CrossSections.input\_path
will be updated as well.

\item {} 
\sphinxAtStartPar
\sphinxstyleliteralstrong{\sphinxupquote{read\_cs\_data}} (\sphinxhref{https://docs.python.org/3/library/functions.html\#bool}{\sphinxstyleliteralemphasis{\sphinxupquote{bool}}}) \textendash{} If \sphinxcode{\sphinxupquote{False}}, only read the process header
information, and not the actual cs data itself, default
is \sphinxcode{\sphinxupquote{True}}.

\end{itemize}

\end{description}\end{quote}

\end{fulllineitems}


\sphinxstepscope


\paragraph{pytb.CrossSections.set\_fixed\_background}
\label{\detokenize{api/pytb.CrossSections.set_fixed_background:pytb-crosssections-set-fixed-background}}\label{\detokenize{api/pytb.CrossSections.set_fixed_background::doc}}\index{set\_fixed\_background() (pytb.CrossSections method)@\spxentry{set\_fixed\_background()}\spxextra{pytb.CrossSections method}}

\begin{fulllineitems}
\phantomsection\label{\detokenize{api/pytb.CrossSections.set_fixed_background:pytb.CrossSections.set_fixed_background}}
\pysigstartsignatures
\pysiglinewithargsret{\sphinxcode{\sphinxupquote{CrossSections.}}\sphinxbfcode{\sphinxupquote{set\_fixed\_background}}}{\sphinxparam{\DUrole{n,n}{fixed}\DUrole{o,o}{=}\DUrole{default_value}{True}}}{}
\pysigstopsignatures
\sphinxAtStartPar
Set all particle conserving processes to have the
\sphinxtitleref{FixedParticle2} tag or not.

\end{fulllineitems}


\sphinxstepscope


\paragraph{pytb.CrossSections.write}
\label{\detokenize{api/pytb.CrossSections.write:pytb-crosssections-write}}\label{\detokenize{api/pytb.CrossSections.write::doc}}\index{write() (pytb.CrossSections method)@\spxentry{write()}\spxextra{pytb.CrossSections method}}

\begin{fulllineitems}
\phantomsection\label{\detokenize{api/pytb.CrossSections.write:pytb.CrossSections.write}}
\pysigstartsignatures
\pysiglinewithargsret{\sphinxcode{\sphinxupquote{CrossSections.}}\sphinxbfcode{\sphinxupquote{write}}}{\sphinxparam{\DUrole{n,n}{directory}\DUrole{o,o}{=}\DUrole{default_value}{None}}}{}
\pysigstopsignatures
\sphinxAtStartPar
Write cross section files into the simulation
cross section directory.
\begin{quote}\begin{description}
\sphinxlineitem{Parameters}
\sphinxAtStartPar
\sphinxstyleliteralstrong{\sphinxupquote{directory}} (\sphinxhref{https://docs.python.org/3/library/stdtypes.html\#str}{\sphinxstyleliteralemphasis{\sphinxupquote{str}}}) \textendash{} Option to write to a specific directory.
If provided, \sphinxcode{\sphinxupquote{CrossSections.cs\_dir}} will be updated.
Default is None.

\sphinxlineitem{Raises}
\sphinxAtStartPar
\sphinxhref{https://docs.python.org/3/library/exceptions.html\#RuntimeError}{\sphinxstyleliteralstrong{\sphinxupquote{RuntimeError}}} \textendash{} If no directory is set or provided.

\end{description}\end{quote}

\end{fulllineitems}


\end{fulllineitems}


\sphinxstepscope


\subsubsection{pytb.Process}
\label{\detokenize{api/pytb.Process:pytb-process}}\label{\detokenize{api/pytb.Process::doc}}\index{Process (class in pytb)@\spxentry{Process}\spxextra{class in pytb}}

\begin{fulllineitems}
\phantomsection\label{\detokenize{api/pytb.Process:pytb.Process}}
\pysigstartsignatures
\pysiglinewithargsret{\sphinxbfcode{\sphinxupquote{class\DUrole{w,w}{  }}}\sphinxcode{\sphinxupquote{pytb.}}\sphinxbfcode{\sphinxupquote{Process}}}{\sphinxparam{\DUrole{n,n}{process\_type}}, \sphinxparam{\DUrole{n,n}{r1}\DUrole{o,o}{=}\DUrole{default_value}{0}}, \sphinxparam{\DUrole{n,n}{r2}\DUrole{o,o}{=}\DUrole{default_value}{1}}, \sphinxparam{\DUrole{n,n}{p1}\DUrole{o,o}{=}\DUrole{default_value}{0}}, \sphinxparam{\DUrole{n,n}{p2}\DUrole{o,o}{=}\DUrole{default_value}{1}}, \sphinxparam{\DUrole{n,n}{threshold}\DUrole{o,o}{=}\DUrole{default_value}{0.0}}, \sphinxparam{\DUrole{n,n}{cs\_func}\DUrole{o,o}{=}\DUrole{default_value}{None}}, \sphinxparam{\DUrole{n,n}{cs\_data}\DUrole{o,o}{=}\DUrole{default_value}{None}}, \sphinxparam{\DUrole{n,n}{name}\DUrole{o,o}{=}\DUrole{default_value}{None}}, \sphinxparam{\DUrole{n,n}{differential\_process}\DUrole{o,o}{=}\DUrole{default_value}{None}}, \sphinxparam{\DUrole{n,n}{nsamples}\DUrole{o,o}{=}\DUrole{default_value}{250}}}{}
\pysigstopsignatures
\sphinxAtStartPar
A reaction process determined by reaction and product indices,
a process type, a potential threshold, and a corresponding cross
section specification.
\begin{description}
\sphinxlineitem{process\_type: str}
\sphinxAtStartPar
Elastic | Inelastic | Ionization

\sphinxlineitem{r1,r2,p1,p2: int (0,1,0,1)}
\sphinxAtStartPar
The indices of the reactants and products.

\sphinxlineitem{threshold: float}
\sphinxAtStartPar
The threshold value for the process (e.g. the binding
energy of an ionization process).

\sphinxlineitem{cs\_func: callable}
\sphinxAtStartPar
Function that returns the cross section for this process in
\(\mathrm{m}^2\) given an incident electron energy in
eV (center of mass frame).

\sphinxlineitem{cs\_data: 2\sphinxhyphen{}D Array\sphinxhyphen{}Like}
\sphinxAtStartPar
Tabular cross section data with columns of energy (eV) and cross
section (\(\mathrm{m}^2\)).

\end{description}
\subsubsection*{Attributes}


\begin{savenotes}\sphinxattablestart
\sphinxthistablewithglobalstyle
\sphinxthistablewithnovlinesstyle
\centering
\begin{tabulary}{\linewidth}[t]{\X{1}{2}\X{1}{2}}
\sphinxtoprule
\sphinxtableatstartofbodyhook
\sphinxAtStartPar
\sphinxcode{\sphinxupquote{SAMPLE\_MAX}}
&
\sphinxAtStartPar

\\
\sphinxhline
\sphinxAtStartPar
\sphinxcode{\sphinxupquote{SAMPLE\_MIN}}
&
\sphinxAtStartPar

\\
\sphinxhline
\sphinxAtStartPar
\sphinxcode{\sphinxupquote{cs\_func}}
&
\sphinxAtStartPar

\\
\sphinxhline
\sphinxAtStartPar
\sphinxcode{\sphinxupquote{nsamples}}
&
\sphinxAtStartPar

\\
\sphinxbottomrule
\end{tabulary}
\sphinxtableafterendhook\par
\sphinxattableend\end{savenotes}
\subsubsection*{Methods}


\begin{savenotes}\sphinxattablestart
\sphinxthistablewithglobalstyle
\sphinxthistablewithnovlinesstyle
\centering
\begin{tabulary}{\linewidth}[t]{\X{1}{2}\X{1}{2}}
\sphinxtoprule
\sphinxtableatstartofbodyhook
\sphinxAtStartPar
{\hyperref[\detokenize{api/pytb.Process.add_differential_parameters:pytb.Process.add_differential_parameters}]{\sphinxcrossref{\sphinxcode{\sphinxupquote{add\_differential\_parameters}}}}}(name, params)
&
\sphinxAtStartPar
Typically differential processes require analytic forms due to the difficulty of extrapolation in several dimensions.
\\
\sphinxhline
\sphinxAtStartPar
{\hyperref[\detokenize{api/pytb.Process.auto_sample:pytb.Process.auto_sample}]{\sphinxcrossref{\sphinxcode{\sphinxupquote{auto\_sample}}}}}()
&
\sphinxAtStartPar
Check if the cross section needs a dense grid or not by comparing simple and dense grids.
\\
\sphinxhline
\sphinxAtStartPar
{\hyperref[\detokenize{api/pytb.Process.require_cs:pytb.Process.require_cs}]{\sphinxcrossref{\sphinxcode{\sphinxupquote{require\_cs}}}}}()
&
\sphinxAtStartPar
Ensure there is some kind of cross section data associated with this process.
\\
\sphinxhline
\sphinxAtStartPar
{\hyperref[\detokenize{api/pytb.Process.sample_cs:pytb.Process.sample_cs}]{\sphinxcrossref{\sphinxcode{\sphinxupquote{sample\_cs}}}}}({[}e\_points, grid\_type, nsamples{]})
&
\sphinxAtStartPar
Sample self.cs\_func on a grid of energies.
\\
\sphinxhline
\sphinxAtStartPar
{\hyperref[\detokenize{api/pytb.Process.to_cs_frame:pytb.Process.to_cs_frame}]{\sphinxcrossref{\sphinxcode{\sphinxupquote{to\_cs\_frame}}}}}(a)
&
\sphinxAtStartPar
Convert any kind of two\sphinxhyphen{}dimensional data to a pandas DataFrame with columns \sphinxtitleref{Energy (eV)} and \sphinxtitleref{Cross Section (m\textasciicircum{}2)}
\\
\sphinxhline
\sphinxAtStartPar
{\hyperref[\detokenize{api/pytb.Process.to_df:pytb.Process.to_df}]{\sphinxcrossref{\sphinxcode{\sphinxupquote{to\_df}}}}}()
&
\sphinxAtStartPar
Convert to properly formatted pandas DataFrame.
\\
\sphinxhline
\sphinxAtStartPar
{\hyperref[\detokenize{api/pytb.Process.zero_below_thresh:pytb.Process.zero_below_thresh}]{\sphinxcrossref{\sphinxcode{\sphinxupquote{zero\_below\_thresh}}}}}()
&
\sphinxAtStartPar
Enforce the ThunderBoltz required cross section format.
\\
\sphinxbottomrule
\end{tabulary}
\sphinxtableafterendhook\par
\sphinxattableend\end{savenotes}

\sphinxstepscope


\paragraph{pytb.Process.add\_differential\_parameters}
\label{\detokenize{api/pytb.Process.add_differential_parameters:pytb-process-add-differential-parameters}}\label{\detokenize{api/pytb.Process.add_differential_parameters::doc}}\index{add\_differential\_parameters() (pytb.Process method)@\spxentry{add\_differential\_parameters()}\spxextra{pytb.Process method}}

\begin{fulllineitems}
\phantomsection\label{\detokenize{api/pytb.Process.add_differential_parameters:pytb.Process.add_differential_parameters}}
\pysigstartsignatures
\pysiglinewithargsret{\sphinxcode{\sphinxupquote{Process.}}\sphinxbfcode{\sphinxupquote{add\_differential\_parameters}}}{\sphinxparam{\DUrole{n,n}{name}}, \sphinxparam{\DUrole{n,n}{params}}}{}
\pysigstopsignatures
\sphinxAtStartPar
Typically differential processes require analytic forms
due to the difficulty of extrapolation in several dimensions.
Add free parameters into an analytic differential model here.
\begin{quote}\begin{description}
\sphinxlineitem{Parameters}\begin{itemize}
\item {} 
\sphinxAtStartPar
\sphinxstyleliteralstrong{\sphinxupquote{name}} \textendash{} The name of the model for this differential process.

\item {} 
\sphinxAtStartPar
\sphinxstyleliteralstrong{\sphinxupquote{params}} \textendash{} The free parameters required for this differential model.

\end{itemize}

\end{description}\end{quote}

\end{fulllineitems}


\sphinxstepscope


\paragraph{pytb.Process.auto\_sample}
\label{\detokenize{api/pytb.Process.auto_sample:pytb-process-auto-sample}}\label{\detokenize{api/pytb.Process.auto_sample::doc}}\index{auto\_sample() (pytb.Process method)@\spxentry{auto\_sample()}\spxextra{pytb.Process method}}

\begin{fulllineitems}
\phantomsection\label{\detokenize{api/pytb.Process.auto_sample:pytb.Process.auto_sample}}
\pysigstartsignatures
\pysiglinewithargsret{\sphinxcode{\sphinxupquote{Process.}}\sphinxbfcode{\sphinxupquote{auto\_sample}}}{}{}
\pysigstopsignatures
\sphinxAtStartPar
Check if the cross section needs a dense grid or not by comparing
simple and dense grids.

\end{fulllineitems}


\sphinxstepscope


\paragraph{pytb.Process.require\_cs}
\label{\detokenize{api/pytb.Process.require_cs:pytb-process-require-cs}}\label{\detokenize{api/pytb.Process.require_cs::doc}}\index{require\_cs() (pytb.Process method)@\spxentry{require\_cs()}\spxextra{pytb.Process method}}

\begin{fulllineitems}
\phantomsection\label{\detokenize{api/pytb.Process.require_cs:pytb.Process.require_cs}}
\pysigstartsignatures
\pysiglinewithargsret{\sphinxcode{\sphinxupquote{Process.}}\sphinxbfcode{\sphinxupquote{require\_cs}}}{}{}
\pysigstopsignatures
\sphinxAtStartPar
Ensure there is some kind of cross section data associated with this
process.
\begin{quote}\begin{description}
\sphinxlineitem{Raises}
\sphinxAtStartPar
\sphinxhref{https://docs.python.org/3/library/exceptions.html\#RuntimeError}{\sphinxstyleliteralstrong{\sphinxupquote{RuntimeError}}} \textendash{} if there is no cross section data available.

\end{description}\end{quote}

\end{fulllineitems}


\sphinxstepscope


\paragraph{pytb.Process.sample\_cs}
\label{\detokenize{api/pytb.Process.sample_cs:pytb-process-sample-cs}}\label{\detokenize{api/pytb.Process.sample_cs::doc}}\index{sample\_cs() (pytb.Process method)@\spxentry{sample\_cs()}\spxextra{pytb.Process method}}

\begin{fulllineitems}
\phantomsection\label{\detokenize{api/pytb.Process.sample_cs:pytb.Process.sample_cs}}
\pysigstartsignatures
\pysiglinewithargsret{\sphinxcode{\sphinxupquote{Process.}}\sphinxbfcode{\sphinxupquote{sample\_cs}}}{\sphinxparam{\DUrole{n,n}{e\_points}\DUrole{o,o}{=}\DUrole{default_value}{None}}, \sphinxparam{\DUrole{n,n}{grid\_type}\DUrole{o,o}{=}\DUrole{default_value}{\textquotesingle{}log dense\textquotesingle{}}}, \sphinxparam{\DUrole{n,n}{nsamples}\DUrole{o,o}{=}\DUrole{default_value}{None}}}{}
\pysigstopsignatures
\sphinxAtStartPar
Sample self.cs\_func on a grid of energies.
\begin{quote}\begin{description}
\sphinxlineitem{Parameters}\begin{itemize}
\item {} 
\sphinxAtStartPar
\sphinxstyleliteralstrong{\sphinxupquote{e\_points}} (\sphinxstyleliteralemphasis{\sphinxupquote{ArrayLike}}) \textendash{} Explicit energy (eV) grid points on which to sample.

\item {} 
\sphinxAtStartPar
\sphinxstyleliteralstrong{\sphinxupquote{grid\_type}} (\sphinxhref{https://docs.python.org/3/library/stdtypes.html\#str}{\sphinxstyleliteralemphasis{\sphinxupquote{str}}}) \textendash{} 

\begin{savenotes}\sphinxattablestart
\sphinxthistablewithglobalstyle
\centering
\begin{tabulary}{\linewidth}[t]{TT}
\sphinxtoprule
\sphinxtableatstartofbodyhook
\sphinxAtStartPar
\sphinxcode{\sphinxupquote{"log dense"}}
&
\sphinxAtStartPar
sample \sphinxcode{\sphinxupquote{nsamples}} near threshold up to 1MeV.
\\
\sphinxhline
\sphinxAtStartPar
\sphinxcode{\sphinxupquote{"simple"}}
&
\sphinxAtStartPar
sample 0 eV \sphinxhyphen{} threshold \sphinxhyphen{} 1 MeV
\\
\sphinxhline
\sphinxAtStartPar
\sphinxcode{\sphinxupquote{None}}
&
\sphinxAtStartPar
Behavior will automatically be determined
by \sphinxcode{\sphinxupquote{auto\_sample}}.
\\
\sphinxbottomrule
\end{tabulary}
\sphinxtableafterendhook\par
\sphinxattableend\end{savenotes}


\item {} 
\sphinxAtStartPar
\sphinxstyleliteralstrong{\sphinxupquote{nsamples}} (\sphinxhref{https://docs.python.org/3/library/functions.html\#int}{\sphinxstyleliteralemphasis{\sphinxupquote{int}}}) \textendash{} Override self.nsamples for this sampling call.

\end{itemize}

\end{description}\end{quote}

\end{fulllineitems}


\sphinxstepscope


\paragraph{pytb.Process.to\_cs\_frame}
\label{\detokenize{api/pytb.Process.to_cs_frame:pytb-process-to-cs-frame}}\label{\detokenize{api/pytb.Process.to_cs_frame::doc}}\index{to\_cs\_frame() (pytb.Process method)@\spxentry{to\_cs\_frame()}\spxextra{pytb.Process method}}

\begin{fulllineitems}
\phantomsection\label{\detokenize{api/pytb.Process.to_cs_frame:pytb.Process.to_cs_frame}}
\pysigstartsignatures
\pysiglinewithargsret{\sphinxcode{\sphinxupquote{Process.}}\sphinxbfcode{\sphinxupquote{to\_cs\_frame}}}{\sphinxparam{\DUrole{n,n}{a}}}{}
\pysigstopsignatures
\sphinxAtStartPar
Convert any kind of two\sphinxhyphen{}dimensional data to a pandas
DataFrame with columns \sphinxtitleref{Energy (eV)} and \sphinxtitleref{Cross Section (m\textasciicircum{}2)}

\end{fulllineitems}


\sphinxstepscope


\paragraph{pytb.Process.to\_df}
\label{\detokenize{api/pytb.Process.to_df:pytb-process-to-df}}\label{\detokenize{api/pytb.Process.to_df::doc}}\index{to\_df() (pytb.Process method)@\spxentry{to\_df()}\spxextra{pytb.Process method}}

\begin{fulllineitems}
\phantomsection\label{\detokenize{api/pytb.Process.to_df:pytb.Process.to_df}}
\pysigstartsignatures
\pysiglinewithargsret{\sphinxcode{\sphinxupquote{Process.}}\sphinxbfcode{\sphinxupquote{to\_df}}}{}{}
\pysigstopsignatures
\sphinxAtStartPar
Convert to properly formatted pandas DataFrame.
\begin{quote}\begin{description}
\sphinxlineitem{Returns}
\sphinxAtStartPar
The process information.

\sphinxlineitem{Return type}
\sphinxAtStartPar
(\sphinxhref{http://pandas.pydata.org/pandas-docs/dev/reference/api/pandas.DataFrame.html\#pandas.DataFrame}{\sphinxcode{\sphinxupquote{pandas.DataFrame}}})

\sphinxlineitem{Raises}
\sphinxAtStartPar
\sphinxhref{https://docs.python.org/3/library/exceptions.html\#RuntimeError}{\sphinxstyleliteralstrong{\sphinxupquote{RuntimeError}}} \textendash{} if no data is available to produce a DataFrame

\end{description}\end{quote}

\end{fulllineitems}


\sphinxstepscope


\paragraph{pytb.Process.zero\_below\_thresh}
\label{\detokenize{api/pytb.Process.zero_below_thresh:pytb-process-zero-below-thresh}}\label{\detokenize{api/pytb.Process.zero_below_thresh::doc}}\index{zero\_below\_thresh() (pytb.Process method)@\spxentry{zero\_below\_thresh()}\spxextra{pytb.Process method}}

\begin{fulllineitems}
\phantomsection\label{\detokenize{api/pytb.Process.zero_below_thresh:pytb.Process.zero_below_thresh}}
\pysigstartsignatures
\pysiglinewithargsret{\sphinxcode{\sphinxupquote{Process.}}\sphinxbfcode{\sphinxupquote{zero\_below\_thresh}}}{}{}
\pysigstopsignatures
\sphinxAtStartPar
Enforce the ThunderBoltz required cross section format.
For processes with a non\sphinxhyphen{}zero threshold, include zero valued
points at 0 eV and at threshold energy.
\begin{quote}\begin{description}
\sphinxlineitem{Raises}
\sphinxAtStartPar
\sphinxhref{https://docs.python.org/3/library/exceptions.html\#RuntimeError}{\sphinxstyleliteralstrong{\sphinxupquote{RuntimeError}}} \textendash{} if there is no cross section data to format.

\end{description}\end{quote}

\end{fulllineitems}


\end{fulllineitems}


\sphinxstepscope


\subsubsection{pytb.input.He\_TB}
\label{\detokenize{api/pytb.input.He_TB:pytb-input-he-tb}}\label{\detokenize{api/pytb.input.He_TB::doc}}\index{He\_TB() (in module pytb.input)@\spxentry{He\_TB()}\spxextra{in module pytb.input}}

\begin{fulllineitems}
\phantomsection\label{\detokenize{api/pytb.input.He_TB:pytb.input.He_TB}}
\pysigstartsignatures
\pysiglinewithargsret{\sphinxcode{\sphinxupquote{pytb.input.}}\sphinxbfcode{\sphinxupquote{He\_TB}}}{\sphinxparam{\DUrole{n,n}{n}\DUrole{o,o}{=}\DUrole{default_value}{4}}, \sphinxparam{\DUrole{n,n}{egen}\DUrole{o,o}{=}\DUrole{default_value}{True}}, \sphinxparam{\DUrole{n,n}{analytic\_cs}\DUrole{o,o}{=}\DUrole{default_value}{True}}, \sphinxparam{\DUrole{n,n}{eadf}\DUrole{o,o}{=}\DUrole{default_value}{\textquotesingle{}default\textquotesingle{}}}, \sphinxparam{\DUrole{n,n}{ECS}\DUrole{o,o}{=}\DUrole{default_value}{None}}, \sphinxparam{\DUrole{n,n}{nsamples}\DUrole{o,o}{=}\DUrole{default_value}{250}}, \sphinxparam{\DUrole{n,n}{mix\_thresh}\DUrole{o,o}{=}\DUrole{default_value}{300.0}}, \sphinxparam{\DUrole{n,n}{fixed\_background}\DUrole{o,o}{=}\DUrole{default_value}{True}}}{}
\pysigstopsignatures
\sphinxAtStartPar
Generate parameterized He cross section sets in the ThunderBoltz format.
The data is from Igor Bray and Dmitry V Fursa 2011
J. Phys. B: At. Mol. Opt. Phys. 44 061001.
\begin{quote}\begin{description}
\sphinxlineitem{Parameters}\begin{itemize}
\item {} 
\sphinxAtStartPar
\sphinxstyleliteralstrong{\sphinxupquote{n}} (\sphinxhref{https://docs.python.org/3/library/functions.html\#int}{\sphinxstyleliteralemphasis{\sphinxupquote{int}}}) \textendash{} Include CCC excitation processes from the ground state to
(up to and including) states with principle quantum number \sphinxcode{\sphinxupquote{n}}.

\item {} 
\sphinxAtStartPar
\sphinxstyleliteralstrong{\sphinxupquote{egen}} (\sphinxhref{https://docs.python.org/3/library/functions.html\#bool}{\sphinxstyleliteralemphasis{\sphinxupquote{bool}}}) \textendash{} Allow secondary electron generation for the ionization model.

\item {} 
\sphinxAtStartPar
\sphinxstyleliteralstrong{\sphinxupquote{analytic\_cs}} (\sphinxhref{https://docs.python.org/3/library/stdtypes.html\#str}{\sphinxstyleliteralemphasis{\sphinxupquote{str}}}\sphinxstyleliteralemphasis{\sphinxupquote{ or }}\sphinxhref{https://docs.python.org/3/library/functions.html\#bool}{\sphinxstyleliteralemphasis{\sphinxupquote{bool}}}) \textendash{} use either tabulated data, analytic fits,
or a mix of both. Options are \sphinxcode{\sphinxupquote{False}}, \sphinxcode{\sphinxupquote{True}}, or \sphinxcode{\sphinxupquote{"mixed"}}.

\item {} 
\sphinxAtStartPar
\sphinxstyleliteralstrong{\sphinxupquote{eadf}} (\sphinxhref{https://docs.python.org/3/library/stdtypes.html\#str}{\sphinxstyleliteralemphasis{\sphinxupquote{str}}}) \textendash{} \sphinxcode{\sphinxupquote{"default"}}, or \sphinxcode{\sphinxupquote{"He\_Park"}}.

\item {} 
\sphinxAtStartPar
\sphinxstyleliteralstrong{\sphinxupquote{ECS}} (\sphinxhref{https://docs.python.org/3/library/stdtypes.html\#str}{\sphinxstyleliteralemphasis{\sphinxupquote{str}}}\sphinxstyleliteralemphasis{\sphinxupquote{ or }}\sphinxstyleliteralemphasis{\sphinxupquote{None}}) \textendash{} The total elastic cross section model. Options are
\sphinxcode{\sphinxupquote{"ICS"}} or \sphinxcode{\sphinxupquote{"MTCS"}}, default is \sphinxcode{\sphinxupquote{"ICS"}} if an anisotropic
angular distribution function is used and \sphinxcode{\sphinxupquote{"MTCS"}} if an isotropic
angular distribution function is used.

\item {} 
\sphinxAtStartPar
\sphinxstyleliteralstrong{\sphinxupquote{nsamples}} (\sphinxhref{https://docs.python.org/3/library/functions.html\#int}{\sphinxstyleliteralemphasis{\sphinxupquote{int}}}) \textendash{} The number of tabulated cross section values for analytic
sampling.

\item {} 
\sphinxAtStartPar
\sphinxstyleliteralstrong{\sphinxupquote{mix\_thresh}} (\sphinxhref{https://docs.python.org/3/library/functions.html\#float}{\sphinxstyleliteralemphasis{\sphinxupquote{float}}}) \textendash{} If \sphinxcode{\sphinxupquote{analytic\_cs}} is \sphinxcode{\sphinxupquote{"mixed"}}, use numerical data
at energies lower than this threshold value (in eV), and use analytic
data at higher energies.

\item {} 
\sphinxAtStartPar
\sphinxstyleliteralstrong{\sphinxupquote{fixed\_background}} (\sphinxhref{https://docs.python.org/3/library/functions.html\#bool}{\sphinxstyleliteralemphasis{\sphinxupquote{bool}}}) \textendash{} Flag to append “FixedParticle2” to each of the
reaction types in the indeck.

\end{itemize}

\sphinxlineitem{Returns}
\sphinxAtStartPar
\begin{description}
\sphinxlineitem{The {\hyperref[\detokenize{api/pytb.CrossSections:pytb.CrossSections}]{\sphinxcrossref{\sphinxcode{\sphinxupquote{CrossSections}}}}} object for Helium}
\sphinxAtStartPar
and the dictionary of ThunderBoltz parameters suitable
for the cross section model.

\end{description}


\sphinxlineitem{Return type}
\sphinxAtStartPar
Tuple{[}\sphinxhref{https://docs.python.org/3/library/stdtypes.html\#dict}{dict},\sphinxhref{https://docs.python.org/3/library/stdtypes.html\#dict}{dict}{]}

\end{description}\end{quote}

\end{fulllineitems}


\sphinxstepscope


\subsubsection{pytb.input.convert}
\label{\detokenize{api/pytb.input.convert:pytb-input-convert}}\label{\detokenize{api/pytb.input.convert::doc}}\index{convert() (in module pytb.input)@\spxentry{convert()}\spxextra{in module pytb.input}}

\begin{fulllineitems}
\phantomsection\label{\detokenize{api/pytb.input.convert:pytb.input.convert}}
\pysigstartsignatures
\pysiglinewithargsret{\sphinxcode{\sphinxupquote{pytb.input.}}\sphinxbfcode{\sphinxupquote{convert}}}{\sphinxparam{\DUrole{n,n}{df}}, \sphinxparam{\DUrole{n,n}{u1}}, \sphinxparam{\DUrole{n,n}{u2}}, \sphinxparam{\DUrole{n,n}{inv}\DUrole{o,o}{=}\DUrole{default_value}{False}}, \sphinxparam{\DUrole{n,n}{drop}\DUrole{o,o}{=}\DUrole{default_value}{False}}, \sphinxparam{\DUrole{n,n}{add}\DUrole{o,o}{=}\DUrole{default_value}{False}}}{}
\pysigstopsignatures
\sphinxAtStartPar
Convert easily between units with a labeled DataFrame.
Columns in the format \sphinxtitleref{\textless{}name\textgreater{} (\textless{}unit\textgreater{})} will be converted.

\end{fulllineitems}



\subsection{Parallel Computing}
\label{\detokenize{ref:parallel-computing}}

\begin{savenotes}\sphinxattablestart
\sphinxthistablewithglobalstyle
\sphinxthistablewithnovlinesstyle
\centering
\begin{tabulary}{\linewidth}[t]{\X{1}{2}\X{1}{2}}
\sphinxtoprule
\sphinxtableatstartofbodyhook
\sphinxAtStartPar
{\hyperref[\detokenize{api/pytb.parallel.MPRunner:pytb.parallel.MPRunner}]{\sphinxcrossref{\sphinxcode{\sphinxupquote{parallel.MPRunner}}}}}()
&
\sphinxAtStartPar
Interface for any kind of calculation that is run\sphinxhyphen{}able and set\sphinxhyphen{}able can be compatible with multiprocessing utilities like SlurmManager and DistributedPool.
\\
\sphinxhline
\sphinxAtStartPar
{\hyperref[\detokenize{api/pytb.parallel.DistributedPool:pytb.parallel.DistributedPool}]{\sphinxcrossref{\sphinxcode{\sphinxupquote{parallel.DistributedPool}}}}}(runner{[}, processes{]})
&
\sphinxAtStartPar
A multiprocessing Pool context for running calculations among cores with different settings.
\\
\sphinxhline
\sphinxAtStartPar
{\hyperref[\detokenize{api/pytb.parallel.SlurmManager:pytb.parallel.SlurmManager}]{\sphinxcrossref{\sphinxcode{\sphinxupquote{parallel.SlurmManager}}}}}(runner{[}, directory, ...{]})
&
\sphinxAtStartPar
A python context interface for the common Slurm HPC job manager to run more several intensive calculations on large clusters.
\\
\sphinxbottomrule
\end{tabulary}
\sphinxtableafterendhook\par
\sphinxattableend\end{savenotes}

\sphinxstepscope


\subsubsection{pytb.parallel.MPRunner}
\label{\detokenize{api/pytb.parallel.MPRunner:pytb-parallel-mprunner}}\label{\detokenize{api/pytb.parallel.MPRunner::doc}}\index{MPRunner (class in pytb.parallel)@\spxentry{MPRunner}\spxextra{class in pytb.parallel}}

\begin{fulllineitems}
\phantomsection\label{\detokenize{api/pytb.parallel.MPRunner:pytb.parallel.MPRunner}}
\pysigstartsignatures
\pysigline{\sphinxbfcode{\sphinxupquote{class\DUrole{w,w}{  }}}\sphinxcode{\sphinxupquote{pytb.parallel.}}\sphinxbfcode{\sphinxupquote{MPRunner}}}
\pysigstopsignatures
\sphinxAtStartPar
Interface for any kind of calculation that is run\sphinxhyphen{}able
and set\sphinxhyphen{}able can be compatible with multiprocessing utilities like
SlurmManager and DistributedPool.
\subsubsection*{Methods}


\begin{savenotes}\sphinxattablestart
\sphinxthistablewithglobalstyle
\sphinxthistablewithnovlinesstyle
\centering
\begin{tabulary}{\linewidth}[t]{\X{1}{2}\X{1}{2}}
\sphinxtoprule
\sphinxtableatstartofbodyhook
\sphinxAtStartPar
{\hyperref[\detokenize{api/pytb.parallel.MPRunner.get_directory:pytb.parallel.MPRunner.get_directory}]{\sphinxcrossref{\sphinxcode{\sphinxupquote{get\_directory}}}}}()
&
\sphinxAtStartPar
Return the directory in which the calculation is occurring.
\\
\sphinxhline
\sphinxAtStartPar
{\hyperref[\detokenize{api/pytb.parallel.MPRunner.run:pytb.parallel.MPRunner.run}]{\sphinxcrossref{\sphinxcode{\sphinxupquote{run}}}}}(**run\_options)
&
\sphinxAtStartPar
Run the calculation with the current settings.
\\
\sphinxhline
\sphinxAtStartPar
{\hyperref[\detokenize{api/pytb.parallel.MPRunner.set_:pytb.parallel.MPRunner.set_}]{\sphinxcrossref{\sphinxcode{\sphinxupquote{set\_}}}}}(**state\_options)
&
\sphinxAtStartPar
Update the internal state of the object being run.
\\
\sphinxhline
\sphinxAtStartPar
{\hyperref[\detokenize{api/pytb.parallel.MPRunner.to_pickleable:pytb.parallel.MPRunner.to_pickleable}]{\sphinxcrossref{\sphinxcode{\sphinxupquote{to\_pickleable}}}}}()
&
\sphinxAtStartPar
Returns a pickle\sphinxhyphen{}able portion of the object sufficient to run the calculations.
\\
\sphinxbottomrule
\end{tabulary}
\sphinxtableafterendhook\par
\sphinxattableend\end{savenotes}

\sphinxstepscope


\paragraph{pytb.parallel.MPRunner.get\_directory}
\label{\detokenize{api/pytb.parallel.MPRunner.get_directory:pytb-parallel-mprunner-get-directory}}\label{\detokenize{api/pytb.parallel.MPRunner.get_directory::doc}}\index{get\_directory() (pytb.parallel.MPRunner method)@\spxentry{get\_directory()}\spxextra{pytb.parallel.MPRunner method}}

\begin{fulllineitems}
\phantomsection\label{\detokenize{api/pytb.parallel.MPRunner.get_directory:pytb.parallel.MPRunner.get_directory}}
\pysigstartsignatures
\pysiglinewithargsret{\sphinxcode{\sphinxupquote{MPRunner.}}\sphinxbfcode{\sphinxupquote{get\_directory}}}{}{}
\pysigstopsignatures
\sphinxAtStartPar
Return the directory in which the calculation is occurring.
\begin{description}
\sphinxlineitem{Returns}
\sphinxAtStartPar
(str): The path of the directory in which the program is being run.

\end{description}

\end{fulllineitems}


\sphinxstepscope


\paragraph{pytb.parallel.MPRunner.run}
\label{\detokenize{api/pytb.parallel.MPRunner.run:pytb-parallel-mprunner-run}}\label{\detokenize{api/pytb.parallel.MPRunner.run::doc}}\index{run() (pytb.parallel.MPRunner method)@\spxentry{run()}\spxextra{pytb.parallel.MPRunner method}}

\begin{fulllineitems}
\phantomsection\label{\detokenize{api/pytb.parallel.MPRunner.run:pytb.parallel.MPRunner.run}}
\pysigstartsignatures
\pysiglinewithargsret{\sphinxcode{\sphinxupquote{MPRunner.}}\sphinxbfcode{\sphinxupquote{run}}}{\sphinxparam{\DUrole{o,o}{**}\DUrole{n,n}{run\_options}}}{}
\pysigstopsignatures
\sphinxAtStartPar
Run the calculation with the current settings.
\begin{quote}\begin{description}
\sphinxlineitem{Parameters}
\sphinxAtStartPar
\sphinxstyleliteralstrong{\sphinxupquote{**run\_options}} \textendash{} Keywords arguments that modify the nature of the
way the program runs.

\end{description}\end{quote}

\end{fulllineitems}


\sphinxstepscope


\paragraph{pytb.parallel.MPRunner.set\_}
\label{\detokenize{api/pytb.parallel.MPRunner.set_:pytb-parallel-mprunner-set}}\label{\detokenize{api/pytb.parallel.MPRunner.set_::doc}}\index{set\_() (pytb.parallel.MPRunner method)@\spxentry{set\_()}\spxextra{pytb.parallel.MPRunner method}}

\begin{fulllineitems}
\phantomsection\label{\detokenize{api/pytb.parallel.MPRunner.set_:pytb.parallel.MPRunner.set_}}
\pysigstartsignatures
\pysiglinewithargsret{\sphinxcode{\sphinxupquote{MPRunner.}}\sphinxbfcode{\sphinxupquote{set\_}}}{\sphinxparam{\DUrole{o,o}{**}\DUrole{n,n}{state\_options}}}{}
\pysigstopsignatures
\sphinxAtStartPar
Update the internal state of the object being run.
\begin{quote}\begin{description}
\sphinxlineitem{Parameters}
\sphinxAtStartPar
\sphinxstyleliteralstrong{\sphinxupquote{**state\_options}} \textendash{} Keywords arguments corresponding to attributes of the object
being updated.

\end{description}\end{quote}

\end{fulllineitems}


\sphinxstepscope


\paragraph{pytb.parallel.MPRunner.to\_pickleable}
\label{\detokenize{api/pytb.parallel.MPRunner.to_pickleable:pytb-parallel-mprunner-to-pickleable}}\label{\detokenize{api/pytb.parallel.MPRunner.to_pickleable::doc}}\index{to\_pickleable() (pytb.parallel.MPRunner method)@\spxentry{to\_pickleable()}\spxextra{pytb.parallel.MPRunner method}}

\begin{fulllineitems}
\phantomsection\label{\detokenize{api/pytb.parallel.MPRunner.to_pickleable:pytb.parallel.MPRunner.to_pickleable}}
\pysigstartsignatures
\pysiglinewithargsret{\sphinxcode{\sphinxupquote{MPRunner.}}\sphinxbfcode{\sphinxupquote{to\_pickleable}}}{}{}
\pysigstopsignatures
\sphinxAtStartPar
Returns a pickle\sphinxhyphen{}able portion of the object sufficient to run
the calculations.

\end{fulllineitems}


\end{fulllineitems}


\sphinxstepscope


\subsubsection{pytb.parallel.DistributedPool}
\label{\detokenize{api/pytb.parallel.DistributedPool:pytb-parallel-distributedpool}}\label{\detokenize{api/pytb.parallel.DistributedPool::doc}}\index{DistributedPool (class in pytb.parallel)@\spxentry{DistributedPool}\spxextra{class in pytb.parallel}}

\begin{fulllineitems}
\phantomsection\label{\detokenize{api/pytb.parallel.DistributedPool:pytb.parallel.DistributedPool}}
\pysigstartsignatures
\pysiglinewithargsret{\sphinxbfcode{\sphinxupquote{class\DUrole{w,w}{  }}}\sphinxcode{\sphinxupquote{pytb.parallel.}}\sphinxbfcode{\sphinxupquote{DistributedPool}}}{\sphinxparam{\DUrole{n,n}{runner}\DUrole{p,p}{:}\DUrole{w,w}{  }\DUrole{n,n}{{\hyperref[\detokenize{api/pytb.parallel.MPRunner:pytb.parallel.MPRunner}]{\sphinxcrossref{MPRunner}}}}}, \sphinxparam{\DUrole{n,n}{processes}\DUrole{o,o}{=}\DUrole{default_value}{None}}}{}
\pysigstopsignatures
\sphinxAtStartPar
A multiprocessing Pool context for running
calculations among cores with different settings.
\begin{quote}\begin{description}
\sphinxlineitem{Parameters}\begin{itemize}
\item {} 
\sphinxAtStartPar
\sphinxstyleliteralstrong{\sphinxupquote{runner}} ({\hyperref[\detokenize{api/pytb.parallel.MPRunner:pytb.parallel.MPRunner}]{\sphinxcrossref{\sphinxstyleliteralemphasis{\sphinxupquote{MPRunner}}}}}) \textendash{} The calculation runner.

\item {} 
\sphinxAtStartPar
\sphinxstyleliteralstrong{\sphinxupquote{processes}} (\sphinxhref{https://docs.python.org/3/library/functions.html\#int}{\sphinxstyleliteralemphasis{\sphinxupquote{int}}}) \textendash{} The number of cores to divide up the work.

\end{itemize}

\end{description}\end{quote}
\subsubsection*{Methods}


\begin{savenotes}\sphinxattablestart
\sphinxthistablewithglobalstyle
\sphinxthistablewithnovlinesstyle
\centering
\begin{tabulary}{\linewidth}[t]{\X{1}{2}\X{1}{2}}
\sphinxtoprule
\sphinxtableatstartofbodyhook
\sphinxAtStartPar
{\hyperref[\detokenize{api/pytb.parallel.DistributedPool.err_callback:pytb.parallel.DistributedPool.err_callback}]{\sphinxcrossref{\sphinxcode{\sphinxupquote{err\_callback}}}}}(err)
&
\sphinxAtStartPar
Print out errors that subprocesses encounter.
\\
\sphinxhline
\sphinxAtStartPar
{\hyperref[\detokenize{api/pytb.parallel.DistributedPool.submit:pytb.parallel.DistributedPool.submit}]{\sphinxcrossref{\sphinxcode{\sphinxupquote{submit}}}}}({[}run\_args{]})
&
\sphinxAtStartPar
Submit a single job with updated key words to the pool.
\\
\sphinxbottomrule
\end{tabulary}
\sphinxtableafterendhook\par
\sphinxattableend\end{savenotes}

\sphinxstepscope


\paragraph{pytb.parallel.DistributedPool.err\_callback}
\label{\detokenize{api/pytb.parallel.DistributedPool.err_callback:pytb-parallel-distributedpool-err-callback}}\label{\detokenize{api/pytb.parallel.DistributedPool.err_callback::doc}}\index{err\_callback() (pytb.parallel.DistributedPool method)@\spxentry{err\_callback()}\spxextra{pytb.parallel.DistributedPool method}}

\begin{fulllineitems}
\phantomsection\label{\detokenize{api/pytb.parallel.DistributedPool.err_callback:pytb.parallel.DistributedPool.err_callback}}
\pysigstartsignatures
\pysiglinewithargsret{\sphinxcode{\sphinxupquote{DistributedPool.}}\sphinxbfcode{\sphinxupquote{err\_callback}}}{\sphinxparam{\DUrole{n,n}{err}}}{}
\pysigstopsignatures
\sphinxAtStartPar
Print out errors that subprocesses encounter.

\end{fulllineitems}


\sphinxstepscope


\paragraph{pytb.parallel.DistributedPool.submit}
\label{\detokenize{api/pytb.parallel.DistributedPool.submit:pytb-parallel-distributedpool-submit}}\label{\detokenize{api/pytb.parallel.DistributedPool.submit::doc}}\index{submit() (pytb.parallel.DistributedPool method)@\spxentry{submit()}\spxextra{pytb.parallel.DistributedPool method}}

\begin{fulllineitems}
\phantomsection\label{\detokenize{api/pytb.parallel.DistributedPool.submit:pytb.parallel.DistributedPool.submit}}
\pysigstartsignatures
\pysiglinewithargsret{\sphinxcode{\sphinxupquote{DistributedPool.}}\sphinxbfcode{\sphinxupquote{submit}}}{\sphinxparam{\DUrole{n,n}{run\_args}\DUrole{o,o}{=}\DUrole{default_value}{\{\}}}, \sphinxparam{\DUrole{o,o}{**}\DUrole{n,n}{set\_args}}}{}
\pysigstopsignatures
\sphinxAtStartPar
Submit a single job with updated key words to the pool.
\begin{quote}\begin{description}
\sphinxlineitem{Parameters}
\sphinxAtStartPar
\sphinxstyleliteralstrong{\sphinxupquote{run\_args}} (\sphinxhref{https://docs.python.org/3/library/stdtypes.html\#dict}{\sphinxstyleliteralemphasis{\sphinxupquote{dict}}}) \textendash{} Keyword arguments to be passed to
{\hyperref[\detokenize{api/pytb.parallel.MPRunner.run:pytb.parallel.MPRunner.run}]{\sphinxcrossref{\sphinxcode{\sphinxupquote{run()}}}}}.

\end{description}\end{quote}
\begin{description}
\sphinxlineitem{{\color{red}\bfseries{}**}set\_args: Keyword arguments passed to}
\sphinxAtStartPar
{\hyperref[\detokenize{api/pytb.parallel.MPRunner.set_:pytb.parallel.MPRunner.set_}]{\sphinxcrossref{\sphinxcode{\sphinxupquote{set\_()}}}}}
before calling {\hyperref[\detokenize{api/pytb.parallel.MPRunner.run:pytb.parallel.MPRunner.run}]{\sphinxcrossref{\sphinxcode{\sphinxupquote{run()}}}}}.

\end{description}

\end{fulllineitems}


\end{fulllineitems}


\sphinxstepscope


\subsubsection{pytb.parallel.SlurmManager}
\label{\detokenize{api/pytb.parallel.SlurmManager:pytb-parallel-slurmmanager}}\label{\detokenize{api/pytb.parallel.SlurmManager::doc}}\index{SlurmManager (class in pytb.parallel)@\spxentry{SlurmManager}\spxextra{class in pytb.parallel}}

\begin{fulllineitems}
\phantomsection\label{\detokenize{api/pytb.parallel.SlurmManager:pytb.parallel.SlurmManager}}
\pysigstartsignatures
\pysiglinewithargsret{\sphinxbfcode{\sphinxupquote{class\DUrole{w,w}{  }}}\sphinxcode{\sphinxupquote{pytb.parallel.}}\sphinxbfcode{\sphinxupquote{SlurmManager}}}{\sphinxparam{\DUrole{n,n}{runner}\DUrole{p,p}{:}\DUrole{w,w}{  }\DUrole{n,n}{{\hyperref[\detokenize{api/pytb.parallel.MPRunner:pytb.parallel.MPRunner}]{\sphinxcrossref{MPRunner}}}}}, \sphinxparam{\DUrole{n,n}{directory}\DUrole{o,o}{=}\DUrole{default_value}{None}}, \sphinxparam{\DUrole{n,n}{modules}\DUrole{o,o}{=}\DUrole{default_value}{{[}\textquotesingle{}python\textquotesingle{}, \textquotesingle{}gcc\textquotesingle{}{]}}}, \sphinxparam{\DUrole{n,n}{mock}\DUrole{o,o}{=}\DUrole{default_value}{False}}, \sphinxparam{\DUrole{o,o}{**}\DUrole{n,n}{options}}}{}
\pysigstopsignatures
\sphinxAtStartPar
A python context interface for the common Slurm HPC job manager
to run more several intensive calculations on large clusters.
See \sphinxurl{https://slurm.schedmd.com/sbatch.html}.
\begin{quote}\begin{description}
\sphinxlineitem{Parameters}\begin{itemize}
\item {} 
\sphinxAtStartPar
\sphinxstyleliteralstrong{\sphinxupquote{runner}} ({\hyperref[\detokenize{api/pytb.parallel.MPRunner:pytb.parallel.MPRunner}]{\sphinxcrossref{\sphinxstyleliteralemphasis{\sphinxupquote{MPRunner}}}}}) \textendash{} The calculation runner.

\item {} 
\sphinxAtStartPar
\sphinxstyleliteralstrong{\sphinxupquote{directory}} (\sphinxhref{https://docs.python.org/3/library/stdtypes.html\#str}{\sphinxstyleliteralemphasis{\sphinxupquote{str}}}\sphinxstyleliteralemphasis{\sphinxupquote{ or }}\sphinxstyleliteralemphasis{\sphinxupquote{None}}) \textendash{} the current working directory.

\item {} 
\sphinxAtStartPar
\sphinxstyleliteralstrong{\sphinxupquote{modules}} (\sphinxhref{https://docs.python.org/3/library/stdtypes.html\#list}{\sphinxstyleliteralemphasis{\sphinxupquote{list}}}\sphinxstyleliteralemphasis{\sphinxupquote{{[}}}\sphinxhref{https://docs.python.org/3/library/stdtypes.html\#str}{\sphinxstyleliteralemphasis{\sphinxupquote{str}}}\sphinxstyleliteralemphasis{\sphinxupquote{{]}}}) \textendash{} A list of modules to be loaded by the
HPC module system.

\item {} 
\sphinxAtStartPar
\sphinxstyleliteralstrong{\sphinxupquote{mock}} (\sphinxhref{https://docs.python.org/3/library/functions.html\#bool}{\sphinxstyleliteralemphasis{\sphinxupquote{bool}}}) \textendash{} Option to test scripts without calling a slurm
manager.

\item {} 
\sphinxAtStartPar
\sphinxstyleliteralstrong{\sphinxupquote{**options}} \textendash{} Additional keyword arguments will be interpreted as
SLURM parameters.

\end{itemize}

\end{description}\end{quote}

\begin{sphinxadmonition}{note}{Note:}
\sphinxAtStartPar
This job manager currently only works for clusters that either
already have the gcc and python requirements installed on each
compute node, or clusters that use the
\sphinxhref{https://hpc-wiki.info/hpc/Modules}{Module System} to load
functionality.

\sphinxAtStartPar
The default behavior is to accommodate the module system as it
is common on most HPC machines. If you wish to avoid writing
\sphinxcode{\sphinxupquote{module load}} commands in the SLURM script, simply specify
\sphinxcode{\sphinxupquote{modules={[}{]}}} in the constructor.
\end{sphinxadmonition}
\subsubsection*{Attributes}


\begin{savenotes}\sphinxattablestart
\sphinxthistablewithglobalstyle
\sphinxthistablewithnovlinesstyle
\centering
\begin{tabulary}{\linewidth}[t]{\X{1}{2}\X{1}{2}}
\sphinxtoprule
\sphinxtableatstartofbodyhook
\sphinxAtStartPar
\sphinxcode{\sphinxupquote{directory}}
&
\sphinxAtStartPar
The simulation directory
\\
\sphinxhline
\sphinxAtStartPar
\sphinxcode{\sphinxupquote{modules}}
&
\sphinxAtStartPar
The list of modules to be loaded by the HPC module system.
\\
\sphinxhline
\sphinxAtStartPar
\sphinxcode{\sphinxupquote{runner}}
&
\sphinxAtStartPar
The {\hyperref[\detokenize{api/pytb.parallel.MPRunner:pytb.parallel.MPRunner}]{\sphinxcrossref{\sphinxcode{\sphinxupquote{MPRunner}}}}} object.
\\
\sphinxhline
\sphinxAtStartPar
\sphinxcode{\sphinxupquote{job\_ids}}
&
\sphinxAtStartPar
Store references to the slurm job numbers after jobs are submitted
\\
\sphinxhline
\sphinxAtStartPar
\sphinxcode{\sphinxupquote{options}}
&
\sphinxAtStartPar
The SLURM sbatch options
\\
\sphinxbottomrule
\end{tabulary}
\sphinxtableafterendhook\par
\sphinxattableend\end{savenotes}
\subsubsection*{Methods}


\begin{savenotes}\sphinxattablestart
\sphinxthistablewithglobalstyle
\sphinxthistablewithnovlinesstyle
\centering
\begin{tabulary}{\linewidth}[t]{\X{1}{2}\X{1}{2}}
\sphinxtoprule
\sphinxtableatstartofbodyhook
\sphinxAtStartPar
{\hyperref[\detokenize{api/pytb.parallel.SlurmManager.batch_script:pytb.parallel.SlurmManager.batch_script}]{\sphinxcrossref{\sphinxcode{\sphinxupquote{batch\_script}}}}}()
&
\sphinxAtStartPar
The SLURM job script.
\\
\sphinxhline
\sphinxAtStartPar
{\hyperref[\detokenize{api/pytb.parallel.SlurmManager.has_active:pytb.parallel.SlurmManager.has_active}]{\sphinxcrossref{\sphinxcode{\sphinxupquote{has\_active}}}}}()
&
\sphinxAtStartPar
Check whether any submitted jobs are still pending or running.
\\
\sphinxhline
\sphinxAtStartPar
{\hyperref[\detokenize{api/pytb.parallel.SlurmManager.has_pending:pytb.parallel.SlurmManager.has_pending}]{\sphinxcrossref{\sphinxcode{\sphinxupquote{has\_pending}}}}}()
&
\sphinxAtStartPar
Check whether any submitted jobs are still pending.
\\
\sphinxhline
\sphinxAtStartPar
{\hyperref[\detokenize{api/pytb.parallel.SlurmManager.join:pytb.parallel.SlurmManager.join}]{\sphinxcrossref{\sphinxcode{\sphinxupquote{join}}}}}()
&
\sphinxAtStartPar
Wait for all slurm jobs to finish.
\\
\sphinxhline
\sphinxAtStartPar
{\hyperref[\detokenize{api/pytb.parallel.SlurmManager.mock_run:pytb.parallel.SlurmManager.mock_run}]{\sphinxcrossref{\sphinxcode{\sphinxupquote{mock\_run}}}}}()
&
\sphinxAtStartPar
Act as a compute node and test the job scripts sequentially.
\\
\sphinxhline
\sphinxAtStartPar
{\hyperref[\detokenize{api/pytb.parallel.SlurmManager.process_batch_script:pytb.parallel.SlurmManager.process_batch_script}]{\sphinxcrossref{\sphinxcode{\sphinxupquote{process\_batch\_script}}}}}()
&
\sphinxAtStartPar
Inspect the batch script below and process it for use in sbatch.
\\
\sphinxhline
\sphinxAtStartPar
{\hyperref[\detokenize{api/pytb.parallel.SlurmManager.sbatch:pytb.parallel.SlurmManager.sbatch}]{\sphinxcrossref{\sphinxcode{\sphinxupquote{sbatch}}}}}()
&
\sphinxAtStartPar
Call slurm with current settings.
\\
\sphinxhline
\sphinxAtStartPar
{\hyperref[\detokenize{api/pytb.parallel.SlurmManager.set_:pytb.parallel.SlurmManager.set_}]{\sphinxcrossref{\sphinxcode{\sphinxupquote{set\_}}}}}(**options)
&
\sphinxAtStartPar
Update slurm manager options.
\\
\sphinxhline
\sphinxAtStartPar
{\hyperref[\detokenize{api/pytb.parallel.SlurmManager.submit:pytb.parallel.SlurmManager.submit}]{\sphinxcrossref{\sphinxcode{\sphinxupquote{submit}}}}}({[}run\_args{]})
&
\sphinxAtStartPar
Add a set of parameter updates to the job queue.
\\
\sphinxhline
\sphinxAtStartPar
{\hyperref[\detokenize{api/pytb.parallel.SlurmManager.write_slurm_script:pytb.parallel.SlurmManager.write_slurm_script}]{\sphinxcrossref{\sphinxcode{\sphinxupquote{write\_slurm\_script}}}}}({[}path, script\_name{]})
&
\sphinxAtStartPar
Write the SLURM batch script.
\\
\sphinxbottomrule
\end{tabulary}
\sphinxtableafterendhook\par
\sphinxattableend\end{savenotes}

\sphinxstepscope


\paragraph{pytb.parallel.SlurmManager.batch\_script}
\label{\detokenize{api/pytb.parallel.SlurmManager.batch_script:pytb-parallel-slurmmanager-batch-script}}\label{\detokenize{api/pytb.parallel.SlurmManager.batch_script::doc}}\index{batch\_script() (pytb.parallel.SlurmManager method)@\spxentry{batch\_script()}\spxextra{pytb.parallel.SlurmManager method}}

\begin{fulllineitems}
\phantomsection\label{\detokenize{api/pytb.parallel.SlurmManager.batch_script:pytb.parallel.SlurmManager.batch_script}}
\pysigstartsignatures
\pysiglinewithargsret{\sphinxcode{\sphinxupquote{SlurmManager.}}\sphinxbfcode{\sphinxupquote{batch\_script}}}{}{}
\pysigstopsignatures
\sphinxAtStartPar
The SLURM job script. This does not get called in the
parent process, but instead the source code is invoked in the
sbatch script/command for subprocess startup.

\end{fulllineitems}


\sphinxstepscope


\paragraph{pytb.parallel.SlurmManager.has\_active}
\label{\detokenize{api/pytb.parallel.SlurmManager.has_active:pytb-parallel-slurmmanager-has-active}}\label{\detokenize{api/pytb.parallel.SlurmManager.has_active::doc}}\index{has\_active() (pytb.parallel.SlurmManager method)@\spxentry{has\_active()}\spxextra{pytb.parallel.SlurmManager method}}

\begin{fulllineitems}
\phantomsection\label{\detokenize{api/pytb.parallel.SlurmManager.has_active:pytb.parallel.SlurmManager.has_active}}
\pysigstartsignatures
\pysiglinewithargsret{\sphinxcode{\sphinxupquote{SlurmManager.}}\sphinxbfcode{\sphinxupquote{has\_active}}}{}{}
\pysigstopsignatures
\sphinxAtStartPar
Check whether any submitted jobs are still pending or running.
\begin{quote}\begin{description}
\sphinxlineitem{Returns}
\sphinxAtStartPar
\begin{description}
\sphinxlineitem{\sphinxcode{\sphinxupquote{True}} if there are still jobs that are pending or}
\sphinxAtStartPar
running. \sphinxcode{\sphinxupquote{False}} otherwise.

\end{description}


\sphinxlineitem{Return type}
\sphinxAtStartPar
(\sphinxhref{https://docs.python.org/3/library/functions.html\#bool}{bool})

\end{description}\end{quote}

\end{fulllineitems}


\sphinxstepscope


\paragraph{pytb.parallel.SlurmManager.has\_pending}
\label{\detokenize{api/pytb.parallel.SlurmManager.has_pending:pytb-parallel-slurmmanager-has-pending}}\label{\detokenize{api/pytb.parallel.SlurmManager.has_pending::doc}}\index{has\_pending() (pytb.parallel.SlurmManager method)@\spxentry{has\_pending()}\spxextra{pytb.parallel.SlurmManager method}}

\begin{fulllineitems}
\phantomsection\label{\detokenize{api/pytb.parallel.SlurmManager.has_pending:pytb.parallel.SlurmManager.has_pending}}
\pysigstartsignatures
\pysiglinewithargsret{\sphinxcode{\sphinxupquote{SlurmManager.}}\sphinxbfcode{\sphinxupquote{has\_pending}}}{}{}
\pysigstopsignatures
\sphinxAtStartPar
Check whether any submitted jobs are still pending.
\begin{quote}\begin{description}
\sphinxlineitem{Returns}
\sphinxAtStartPar
\begin{description}
\sphinxlineitem{\sphinxcode{\sphinxupquote{True}} if there are still jobs that are pending.}
\sphinxAtStartPar
\sphinxcode{\sphinxupquote{False}} otherwise.

\end{description}


\sphinxlineitem{Return type}
\sphinxAtStartPar
(\sphinxhref{https://docs.python.org/3/library/functions.html\#bool}{bool})

\end{description}\end{quote}

\end{fulllineitems}


\sphinxstepscope


\paragraph{pytb.parallel.SlurmManager.join}
\label{\detokenize{api/pytb.parallel.SlurmManager.join:pytb-parallel-slurmmanager-join}}\label{\detokenize{api/pytb.parallel.SlurmManager.join::doc}}\index{join() (pytb.parallel.SlurmManager method)@\spxentry{join()}\spxextra{pytb.parallel.SlurmManager method}}

\begin{fulllineitems}
\phantomsection\label{\detokenize{api/pytb.parallel.SlurmManager.join:pytb.parallel.SlurmManager.join}}
\pysigstartsignatures
\pysiglinewithargsret{\sphinxcode{\sphinxupquote{SlurmManager.}}\sphinxbfcode{\sphinxupquote{join}}}{}{}
\pysigstopsignatures
\sphinxAtStartPar
Wait for all slurm jobs to finish.

\end{fulllineitems}


\sphinxstepscope


\paragraph{pytb.parallel.SlurmManager.mock\_run}
\label{\detokenize{api/pytb.parallel.SlurmManager.mock_run:pytb-parallel-slurmmanager-mock-run}}\label{\detokenize{api/pytb.parallel.SlurmManager.mock_run::doc}}\index{mock\_run() (pytb.parallel.SlurmManager method)@\spxentry{mock\_run()}\spxextra{pytb.parallel.SlurmManager method}}

\begin{fulllineitems}
\phantomsection\label{\detokenize{api/pytb.parallel.SlurmManager.mock_run:pytb.parallel.SlurmManager.mock_run}}
\pysigstartsignatures
\pysiglinewithargsret{\sphinxcode{\sphinxupquote{SlurmManager.}}\sphinxbfcode{\sphinxupquote{mock\_run}}}{}{}
\pysigstopsignatures
\sphinxAtStartPar
Act as a compute node and test the job scripts sequentially.

\end{fulllineitems}


\sphinxstepscope


\paragraph{pytb.parallel.SlurmManager.process\_batch\_script}
\label{\detokenize{api/pytb.parallel.SlurmManager.process_batch_script:pytb-parallel-slurmmanager-process-batch-script}}\label{\detokenize{api/pytb.parallel.SlurmManager.process_batch_script::doc}}\index{process\_batch\_script() (pytb.parallel.SlurmManager method)@\spxentry{process\_batch\_script()}\spxextra{pytb.parallel.SlurmManager method}}

\begin{fulllineitems}
\phantomsection\label{\detokenize{api/pytb.parallel.SlurmManager.process_batch_script:pytb.parallel.SlurmManager.process_batch_script}}
\pysigstartsignatures
\pysiglinewithargsret{\sphinxcode{\sphinxupquote{SlurmManager.}}\sphinxbfcode{\sphinxupquote{process\_batch\_script}}}{}{}
\pysigstopsignatures
\sphinxAtStartPar
Inspect the batch script below and process it for use
in sbatch.

\end{fulllineitems}


\sphinxstepscope


\paragraph{pytb.parallel.SlurmManager.sbatch}
\label{\detokenize{api/pytb.parallel.SlurmManager.sbatch:pytb-parallel-slurmmanager-sbatch}}\label{\detokenize{api/pytb.parallel.SlurmManager.sbatch::doc}}\index{sbatch() (pytb.parallel.SlurmManager method)@\spxentry{sbatch()}\spxextra{pytb.parallel.SlurmManager method}}

\begin{fulllineitems}
\phantomsection\label{\detokenize{api/pytb.parallel.SlurmManager.sbatch:pytb.parallel.SlurmManager.sbatch}}
\pysigstartsignatures
\pysiglinewithargsret{\sphinxcode{\sphinxupquote{SlurmManager.}}\sphinxbfcode{\sphinxupquote{sbatch}}}{}{}
\pysigstopsignatures
\sphinxAtStartPar
Call slurm with current settings.

\end{fulllineitems}


\sphinxstepscope


\paragraph{pytb.parallel.SlurmManager.set\_}
\label{\detokenize{api/pytb.parallel.SlurmManager.set_:pytb-parallel-slurmmanager-set}}\label{\detokenize{api/pytb.parallel.SlurmManager.set_::doc}}\index{set\_() (pytb.parallel.SlurmManager method)@\spxentry{set\_()}\spxextra{pytb.parallel.SlurmManager method}}

\begin{fulllineitems}
\phantomsection\label{\detokenize{api/pytb.parallel.SlurmManager.set_:pytb.parallel.SlurmManager.set_}}
\pysigstartsignatures
\pysiglinewithargsret{\sphinxcode{\sphinxupquote{SlurmManager.}}\sphinxbfcode{\sphinxupquote{set\_}}}{\sphinxparam{\DUrole{o,o}{**}\DUrole{n,n}{options}}}{}
\pysigstopsignatures
\sphinxAtStartPar
Update slurm manager options.
\begin{quote}\begin{description}
\sphinxlineitem{Parameters}
\sphinxAtStartPar
\sphinxstyleliteralstrong{\sphinxupquote{**options}} \textendash{} SLURM settings.

\end{description}\end{quote}

\end{fulllineitems}


\sphinxstepscope


\paragraph{pytb.parallel.SlurmManager.submit}
\label{\detokenize{api/pytb.parallel.SlurmManager.submit:pytb-parallel-slurmmanager-submit}}\label{\detokenize{api/pytb.parallel.SlurmManager.submit::doc}}\index{submit() (pytb.parallel.SlurmManager method)@\spxentry{submit()}\spxextra{pytb.parallel.SlurmManager method}}

\begin{fulllineitems}
\phantomsection\label{\detokenize{api/pytb.parallel.SlurmManager.submit:pytb.parallel.SlurmManager.submit}}
\pysigstartsignatures
\pysiglinewithargsret{\sphinxcode{\sphinxupquote{SlurmManager.}}\sphinxbfcode{\sphinxupquote{submit}}}{\sphinxparam{\DUrole{n,n}{run\_args}\DUrole{o,o}{=}\DUrole{default_value}{\{\}}}, \sphinxparam{\DUrole{o,o}{**}\DUrole{n,n}{settings}}}{}
\pysigstopsignatures
\sphinxAtStartPar
Add a set of parameter updates to the job queue.
Slurm is not invoked until the context is exited.
\begin{quote}\begin{description}
\sphinxlineitem{Parameters}\begin{itemize}
\item {} 
\sphinxAtStartPar
\sphinxstyleliteralstrong{\sphinxupquote{run\_args}} (\sphinxhref{https://docs.python.org/3/library/stdtypes.html\#dict}{\sphinxstyleliteralemphasis{\sphinxupquote{dict}}}) \textendash{} Keyword arguments to be passed to
{\hyperref[\detokenize{api/pytb.parallel.MPRunner.run:pytb.parallel.MPRunner.run}]{\sphinxcrossref{\sphinxcode{\sphinxupquote{run()}}}}}.

\item {} 
\sphinxAtStartPar
\sphinxstyleliteralstrong{\sphinxupquote{**settings}} \textendash{} Keyword arguments passed to the
{\hyperref[\detokenize{api/pytb.parallel.MPRunner.set_:pytb.parallel.MPRunner.set_}]{\sphinxcrossref{\sphinxcode{\sphinxupquote{set\_()}}}}}.
before calling {\hyperref[\detokenize{api/pytb.parallel.MPRunner.run:pytb.parallel.MPRunner.run}]{\sphinxcrossref{\sphinxcode{\sphinxupquote{run()}}}}}.

\end{itemize}

\end{description}\end{quote}

\end{fulllineitems}


\sphinxstepscope


\paragraph{pytb.parallel.SlurmManager.write\_slurm\_script}
\label{\detokenize{api/pytb.parallel.SlurmManager.write_slurm_script:pytb-parallel-slurmmanager-write-slurm-script}}\label{\detokenize{api/pytb.parallel.SlurmManager.write_slurm_script::doc}}\index{write\_slurm\_script() (pytb.parallel.SlurmManager method)@\spxentry{write\_slurm\_script()}\spxextra{pytb.parallel.SlurmManager method}}

\begin{fulllineitems}
\phantomsection\label{\detokenize{api/pytb.parallel.SlurmManager.write_slurm_script:pytb.parallel.SlurmManager.write_slurm_script}}
\pysigstartsignatures
\pysiglinewithargsret{\sphinxcode{\sphinxupquote{SlurmManager.}}\sphinxbfcode{\sphinxupquote{write\_slurm\_script}}}{\sphinxparam{\DUrole{n,n}{path}\DUrole{o,o}{=}\DUrole{default_value}{None}}, \sphinxparam{\DUrole{n,n}{script\_name}\DUrole{o,o}{=}\DUrole{default_value}{None}}}{}
\pysigstopsignatures
\sphinxAtStartPar
Write the SLURM batch script.

\end{fulllineitems}


\end{fulllineitems}




\renewcommand{\indexname}{Index}
\printindex
\end{document}
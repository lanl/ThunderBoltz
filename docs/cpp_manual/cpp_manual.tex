%% Generated by Sphinx.
\def\sphinxdocclass{report}
\documentclass[letterpaper,10pt,english,openany,oneside]{sphinxmanual}
\ifdefined\pdfpxdimen
   \let\sphinxpxdimen\pdfpxdimen\else\newdimen\sphinxpxdimen
\fi \sphinxpxdimen=.75bp\relax
\ifdefined\pdfimageresolution
    \pdfimageresolution= \numexpr \dimexpr1in\relax/\sphinxpxdimen\relax
\fi
%% let collapsible pdf bookmarks panel have high depth per default
\PassOptionsToPackage{bookmarksdepth=5}{hyperref}

\PassOptionsToPackage{booktabs}{sphinx}
\PassOptionsToPackage{colorrows}{sphinx}

\PassOptionsToPackage{warn}{textcomp}
\usepackage[utf8]{inputenc}
\usepackage{authblk}
\usepackage{listings}
\ifdefined\DeclareUnicodeCharacter
% support both utf8 and utf8x syntaxes
  \ifdefined\DeclareUnicodeCharacterAsOptional
    \def\sphinxDUC#1{\DeclareUnicodeCharacter{"#1}}
  \else
    \let\sphinxDUC\DeclareUnicodeCharacter
  \fi
  \sphinxDUC{00A0}{\nobreakspace}
  \sphinxDUC{2500}{\sphinxunichar{2500}}
  \sphinxDUC{2502}{\sphinxunichar{2502}}
  \sphinxDUC{2514}{\sphinxunichar{2514}}
  \sphinxDUC{251C}{\sphinxunichar{251C}}
  \sphinxDUC{2572}{\textbackslash}
\fi
\usepackage{cmap}
\usepackage[T1]{fontenc}
\usepackage{amsmath,amssymb,amstext}
\usepackage{babel}



\usepackage{tgtermes}
\usepackage{tgheros}
\renewcommand{\ttdefault}{txtt}



\usepackage[Bjarne]{fncychap}
\usepackage{sphinx}

\fvset{fontsize=auto}
\usepackage{geometry}


% Include hyperref last.
\usepackage{hyperref}
% Fix anchor placement for figures with captions.
\usepackage{hypcap}% it must be loaded after hyperref.
% Set up styles of URL: it should be placed after hyperref.
\urlstyle{same}


\usepackage{sphinxmessages}




\title{ThunderBoltz C++ Manual}
\date{Oct 30, 2023\\LA-UR-23-31893}
\release{0.1}
\author{Ryan Park, Brett Scheiner, Mark Zammit}
\affil{Los Alamos National Laboratory, Los Alamos, NM, 87545}
\newcommand{\sphinxlogo}{\vbox{}}
\renewcommand{\releasename}{Release}
\makeindex
\begin{document}

\ifdefined\shorthandoff
  \ifnum\catcode`\=\string=\active\shorthandoff{=}\fi
  \ifnum\catcode`\"=\active\shorthandoff{"}\fi
\fi

\pagestyle{empty}
\sphinxmaketitle
\pagestyle{plain}
\sphinxtableofcontents
\pagestyle{normal}
\phantomsection\label{\detokenize{short_index::doc}}


\sphinxAtStartPar
This documentation includes instruction on installation
and compilation of ThunderBoltz source code and descriptions
of the input and output simulation parameters involved.


\chapter{Installation}
\label{\detokenize{short_index:installation}}
\sphinxAtStartPar
For now, the code must be downloaded from a repository.
Use the following command to clone the code into a local repository.

\begin{sphinxVerbatim}[commandchars=\\\{\}]
git\PYG{+w}{ }clone\PYG{+w}{ }git@github.com:lanl/ThunderBoltz.git
\end{sphinxVerbatim}

\sphinxAtStartPar
You may need to set up SSH keys in order to access Github. See the
\sphinxhref{https://docs.github.com/en/authentication/connecting-to-github-with-ssh}{Githab SSH Guide}
to set up access to Github repositories.

\sphinxAtStartPar
The basic ThunderBoltz functionality is available either
as an executable in \sphinxcode{\sphinxupquote{bin/thunderboltz.bin}} or can be compiled from the
source in \sphinxcode{\sphinxupquote{src/thunderboltz/cpp}}.

\chapter{Compilation}
\label{\detokenize{short_index:compilation}}
\sphinxAtStartPar
ThunderBoltz requires a g++ of clang compiler and should be compiled
from source directories as

\begin{sphinxVerbatim}[commandchars=\\\{\}]
g++\PYG{+w}{ }\PYGZhy{}std\PYG{o}{=}c++17\PYG{+w}{ }\PYGZhy{}o\PYG{+w}{ }thunderboltz.bin\PYG{+w}{ }DSMC0D.cpp
\end{sphinxVerbatim}

\sphinxAtStartPar
Then run with

\begin{sphinxVerbatim}[commandchars=\\\{\}]
./thunderboltz.bin\PYG{+w}{ }inputfile.in
\end{sphinxVerbatim}

\sphinxAtStartPar
to use a manually constructed indeck file. The code is
maintained with the standard \sphinxtitleref{\sphinxhyphen{}Wall} and \sphinxtitleref{\sphinxhyphen{}Werror}
compiler options.

\sphinxAtStartPar
Here is an example of how to run a simple ThunderBoltz calculation.

\begin{sphinxVerbatim}[commandchars=\\\{\}]
\PYG{c+c1}{\PYGZsh{} Make a directory for testing}
mkdir\PYG{+w}{ }example\PYGZus{}sim
\PYG{n+nb}{cd}\PYG{+w}{ }example\PYGZus{}sim
\PYG{c+c1}{\PYGZsh{} Copy the source over}
cp\PYG{+w}{ }../src/thunderboltz/*\PYG{+w}{ }.
\PYG{c+c1}{\PYGZsh{} Copy example input files over}
cp\PYG{+w}{ }\PYGZhy{}r\PYG{+w}{ }../indecks/N2/*\PYG{+w}{ }.
\PYG{c+c1}{\PYGZsh{} Compile}
g++\PYG{+w}{ }\PYGZhy{}std\PYG{o}{=}c++17\PYG{+w}{ }\PYGZhy{}o\PYG{+w}{ }thunderboltz.bin\PYG{+w}{ }DSMC0D.cpp
\PYG{c+c1}{\PYGZsh{} Run}
./thunderboltz.bin\PYG{+w}{ }N2vib.in
\end{sphinxVerbatim}


\chapter{Simulation Parameters}
\label{\detokenize{short_index:simulation-parameters}}

\section{Input Parameters}
The ThunderBoltz simulation settings and their
default values.

% \def \roottables {../build/latex} % To get version from python docs
\def \roottables {.} % To get separate version without defaults (needs maintenance)

\begin{savenotes}\sphinxattablestart
\sphinxthistablewithglobalstyle
\sphinxthistablewithnovlinesstyle
\centering
\begin{tabulary}{\linewidth}[t]{\X{1}{2}\X{1}{2}}
\sphinxtoprule
\sphinxtableatstartofbodyhook
\sphinxAtStartPar
\sphinxcode{\sphinxupquote{B}}
&
\sphinxAtStartPar
(list{[}int{]}) Magnetic field vector (Tesla), default is \sphinxcode{\sphinxupquote{{[}0.0, 0.0, 0.0{]}}}.
\\
\sphinxhline
\sphinxAtStartPar
\sphinxcode{\sphinxupquote{CO}}
&
\sphinxAtStartPar
(str) Collision ordering, options are \sphinxcode{\sphinxupquote{"default"}} | \sphinxcode{\sphinxupquote{"Random"}} | \sphinxcode{\sphinxupquote{"Reverse"}}.
\\
\sphinxhline
\sphinxAtStartPar
\sphinxcode{\sphinxupquote{CR}}
&
\sphinxAtStartPar
(int) If \sphinxcode{\sphinxupquote{1}}, then the remainder of \(N_{\rm pairs}\) is carried into the next \(N_{\rm pairs}\) evaluation of the same process, default is \sphinxcode{\sphinxupquote{1}}.
\\
\sphinxhline
\sphinxAtStartPar
\sphinxcode{\sphinxupquote{DT}}
&
\sphinxAtStartPar
(float) Time increment interval (s), default is \sphinxcode{\sphinxupquote{5e\sphinxhyphen{}12}}.
\\
\sphinxhline
\sphinxAtStartPar
\sphinxcode{\sphinxupquote{E}}
&
\sphinxAtStartPar
(float) Electric field in z\sphinxhyphen{}direction (V/m), default is \sphinxcode{\sphinxupquote{\sphinxhyphen{}24640.0}}.
\\
\sphinxhline
\sphinxAtStartPar
\sphinxcode{\sphinxupquote{ET}}
&
\sphinxAtStartPar
(float) E\sphinxhyphen{}field oscillation frequency (Hz), default is \sphinxcode{\sphinxupquote{0}}.
\\
\sphinxhline
\sphinxAtStartPar
\sphinxcode{\sphinxupquote{EX}}
&
\sphinxAtStartPar
(int) Cross section extrapolation — options are \sphinxcode{\sphinxupquote{0}} (extrapolated cross sections are set to \(0\) m \(^2\)), or \sphinxcode{\sphinxupquote{1}} (linearly extrapolated from last two points), default is \sphinxcode{\sphinxupquote{0}}.
\\
\sphinxhline
\sphinxAtStartPar
\sphinxcode{\sphinxupquote{FV}}
&
\sphinxAtStartPar
(list{[}int{]}) Output velocity dump settings {[}start, stride, species ID{]}, default is \sphinxcode{\sphinxupquote{{[}1000, 1000000, 0{]}}}.
\\
\sphinxhline
\sphinxAtStartPar
\sphinxcode{\sphinxupquote{L}}
&
\sphinxAtStartPar
(float) Cell length (m), default is \sphinxcode{\sphinxupquote{1e\sphinxhyphen{}6}}.
\\
\sphinxhline
\sphinxAtStartPar
\sphinxcode{\sphinxupquote{LV}}
&
\sphinxAtStartPar
(list{[}str,int{]}) Optionally load particle velocities from a comma separated text file, default is \sphinxcode{\sphinxupquote{None}}; specify the name of the file at index \sphinxcode{\sphinxupquote{0}} and the particle species index it applies to at index \sphinxcode{\sphinxupquote{1}}.
\\
\sphinxhline
\sphinxAtStartPar
\sphinxcode{\sphinxupquote{MEM}}
&
\sphinxAtStartPar
(float) Request memory (GB) for particle arrays.
\\
\sphinxhline
\sphinxAtStartPar
\sphinxcode{\sphinxupquote{MP}}
&
\sphinxAtStartPar
(list{[}float{]}) Mass of each particle species (amu), default is \sphinxcode{\sphinxupquote{{[}5.4857e\sphinxhyphen{}4, 28.0{]}}}.
\\
\sphinxhline
\sphinxAtStartPar
\sphinxcode{\sphinxupquote{NP}}
&
\sphinxAtStartPar
(list{[}int{]}) Number of particles for each species, default is \sphinxcode{\sphinxupquote{{[}10000, 1000{]}}}.
\\
\sphinxhline
\sphinxAtStartPar
\sphinxcode{\sphinxupquote{NS}}
&
\sphinxAtStartPar
(int) Number of time steps, default is \sphinxcode{\sphinxupquote{1000001}}.
\\
\sphinxhline
\sphinxAtStartPar
\sphinxcode{\sphinxupquote{OS}}
&
\sphinxAtStartPar
(int) Time step stride for output parameters, default is \sphinxcode{\sphinxupquote{100}}.
\\
\sphinxhline
\sphinxAtStartPar
\sphinxcode{\sphinxupquote{QP}}
&
\sphinxAtStartPar
(list{[}int{]}) Charge (elementary units) of each particle species, default is \sphinxcode{\sphinxupquote{{[}\sphinxhyphen{}1.0, 0.0{]}}}.
\\
\sphinxhline
\sphinxAtStartPar
\sphinxcode{\sphinxupquote{SE}}
&
\sphinxAtStartPar
(int) When using a SLURM manager on HPC, auto dump particle velocity data before job allocation runs out — options are \sphinxcode{\sphinxupquote{0}} (don\textquotesingle{}t auto dump) | \sphinxcode{\sphinxupquote{1}} (dump using SLURM setup).
\\
\sphinxhline
\sphinxAtStartPar
\sphinxcode{\sphinxupquote{SP}}
&
\sphinxAtStartPar
(int) Number of species, default is \sphinxcode{\sphinxupquote{2}}.
\\
\sphinxhline
\sphinxAtStartPar
\sphinxcode{\sphinxupquote{TP}}
&
\sphinxAtStartPar
(list{[}float{]}) Temperature (eV) of each particle species, default is \sphinxcode{\sphinxupquote{{[}0.0, 0.0259{]}}}.
\\
\sphinxhline
\sphinxAtStartPar
\sphinxcode{\sphinxupquote{VS}}
&
\sphinxAtStartPar
(int) Number of random samples used to find \(\max_\epsilon (v\sigma(\epsilon))\) for each process, default is \sphinxcode{\sphinxupquote{1000}}.
\\
\sphinxhline
\sphinxAtStartPar
\sphinxcode{\sphinxupquote{VV}}
&
\sphinxAtStartPar
(list{[}float{]}) Flow velocity for each particle, default is \sphinxcode{\sphinxupquote{{[}0.0, 0.0{]}}}.
\\
\sphinxbottomrule
\end{tabulary}
\sphinxtableafterendhook\par
\sphinxattableend\end{savenotes}


\section{Output Parameters}

\subsection{Banner Output}
A listing of the main output parameters of the simulation. By
default, these are printed to standard out every ``OS" time
steps.

\begin{savenotes}\sphinxattablestart
\sphinxthistablewithglobalstyle
\sphinxthistablewithnovlinesstyle
\centering
\begin{tabulary}{\linewidth}[t]{\X{1}{2}\X{1}{2}}
\sphinxtoprule
\sphinxtableatstartofbodyhook
\sphinxAtStartPar
\sphinxcode{\sphinxupquote{E}}
&
\sphinxAtStartPar
(float) The electric field component (V/m) in the \(z\) direction, which can change in AC scenarios.
\\
\sphinxhline
\sphinxAtStartPar
\sphinxcode{\sphinxupquote{MEe}}
&
\sphinxAtStartPar
(float) The mean energy (eV) of the species at index \(0\) (usually electrons), computed as \(\langle\epsilon\rangle = \frac{m_0}{2N_0}\sum_{i=1}^{N_0}v_{0i}^2\) where \(m_0\) and \(N_0\) are the mass and particle count of the \(0^{\rm th}\) species, and \(v_{0i}\) is the velocity vector of the \(i^{\rm th}\) particle of species \(0\).
\\
\sphinxhline
\sphinxAtStartPar
\sphinxcode{\sphinxupquote{step}}
&
\sphinxAtStartPar
(int) The number of time steps elapsed in the simulation, with \sphinxcode{\sphinxupquote{t}} \(=0\) corresponding to \sphinxcode{\sphinxupquote{step}} \(=0\), and with \sphinxcode{\sphinxupquote{t}} = \sphinxcode{\sphinxupquote{DT}} corresponding to \sphinxcode{\sphinxupquote{step}} \(=1\).
\\
\sphinxhline
\sphinxAtStartPar
\sphinxcode{\sphinxupquote{t}}
&
\sphinxAtStartPar
(float) The time (s) elapsed in the simulation.
\\
\sphinxbottomrule
\end{tabulary}
\sphinxtableafterendhook\par
\sphinxattableend\end{savenotes}


\subsection{Particle Parameters}
A listing of additional output written for each species. A text
file is generated named ``Particle\_Type\_i" for each species ``i"
with the following data in each column.

\begin{savenotes}\sphinxattablestart
\sphinxthistablewithglobalstyle
\sphinxthistablewithnovlinesstyle
\centering
\begin{tabulary}{\linewidth}[t]{\X{1}{2}\X{1}{2}}
\sphinxtoprule
\sphinxtableatstartofbodyhook
\sphinxAtStartPar
\sphinxcode{\sphinxupquote{Ki}}
&
\sphinxAtStartPar
(float) The total kinetic energy (eV).
\\
\sphinxhline
\sphinxAtStartPar
\sphinxcode{\sphinxupquote{Mi}}
&
\sphinxAtStartPar
(float) The mean kinetic energy (eV).
\\
\sphinxhline
\sphinxAtStartPar
\sphinxcode{\sphinxupquote{Ni}}
&
\sphinxAtStartPar
(float) The number density (m \(^{-3}\)).
\\
\sphinxhline
\sphinxAtStartPar
\sphinxcode{\sphinxupquote{Rxi}}
&
\sphinxAtStartPar
(float) The mean x component of all particle displacements (m).
\\
\sphinxhline
\sphinxAtStartPar
\sphinxcode{\sphinxupquote{Ryi}}
&
\sphinxAtStartPar
(float) The mean y component of all particle displacements (m).
\\
\sphinxhline
\sphinxAtStartPar
\sphinxcode{\sphinxupquote{Rzi}}
&
\sphinxAtStartPar
(float) The mean z component of all particle displacements (m).
\\
\sphinxhline
\sphinxAtStartPar
\sphinxcode{\sphinxupquote{Txi}}
&
\sphinxAtStartPar
(float) The mean x component temperature (eV).
\\
\sphinxhline
\sphinxAtStartPar
\sphinxcode{\sphinxupquote{Tyi}}
&
\sphinxAtStartPar
(float) The mean y component temperature (eV).
\\
\sphinxhline
\sphinxAtStartPar
\sphinxcode{\sphinxupquote{Tzi}}
&
\sphinxAtStartPar
(float) The mean z component temperature (eV).
\\
\sphinxhline
\sphinxAtStartPar
\sphinxcode{\sphinxupquote{Vxi}}
&
\sphinxAtStartPar
(float) The mean x component velocity (m/s).
\\
\sphinxhline
\sphinxAtStartPar
\sphinxcode{\sphinxupquote{Vyi}}
&
\sphinxAtStartPar
(float) The mean y component velocity (m/s).
\\
\sphinxhline
\sphinxAtStartPar
\sphinxcode{\sphinxupquote{Vzi}}
&
\sphinxAtStartPar
(float) The mean z component velocity (m/s).
\\
\sphinxhline
\sphinxAtStartPar
\sphinxcode{\sphinxupquote{step}}
&
\sphinxAtStartPar
(int) The number of time steps elapsed in the simulation, with \sphinxcode{\sphinxupquote{t}} \(=0\) corresponding to \sphinxcode{\sphinxupquote{step}} \(=0\).
\\
\sphinxhline
\sphinxAtStartPar
\sphinxcode{\sphinxupquote{t}}
&
\sphinxAtStartPar
(float) The time (s) elapsed in the simulation.
\\
\sphinxbottomrule
\end{tabulary}
\sphinxtableafterendhook\par
\sphinxattableend\end{savenotes}




\chapter{Electron Growth and Memory Management}
\label{\detokenize{short_index:electron-growth-and-memory-management}}\label{\detokenize{short_index:memory}}
\sphinxAtStartPar
Depending on the ionization model and field strength,
ThunderBoltz may generate a large number of electrons.
In these cases, the appropriate amount of memory must be
allocated. The correct amount will be allocated automatically
in scenarios where no ionization process is used,
or when the \sphinxcode{\sphinxupquote{IonizationNoEgen}} model is used. This amount
will be allocated based on the sum of all \sphinxcode{\sphinxupquote{NP}} elements
times 4.

\sphinxAtStartPar
However, in scenarios where there is significant electron generation,
i.e. at high \(E\) fields with the \sphinxcode{\sphinxupquote{Ionization}} model on,
the default memory settings are not sufficient and the simulation
will exit with the error “Too many particles!”. To prevent this
specify the \sphinxcode{\sphinxupquote{MEM}} flag in the indeck. MEM will accept any
float representing the number of gigabytes to be made available
to the particle arrays.

\begin{sphinxadmonition}{warning}{Warning:}
\sphinxAtStartPar
If the value of \sphinxcode{\sphinxupquote{MEM}} is more than the actual number of
available GB, then the simulation will still run, but will
exit with a segmentation fault once too many particles are
created.
\end{sphinxadmonition}

\begin{sphinxadmonition}{warning}{Warning:}
\sphinxAtStartPar
When using multiple cores on the same machine / node, ensure
that each process has enough memory requested and that
the sum of memory requests does not exceed the available
pool of RAM.
\end{sphinxadmonition}

\chapter{Input Cross Section Format}

\section{Indeck Format}

ThunderBoltz input is contained within an input deck that specifies the simulation
behavior. Each line of the input deck, including comments, starts with a one or two
character long specifier called an index. The deck order is unimportant with the
exception that the number of species needs to be specified before other particle
definitions such as number, charge, temperature, flow velocity, or mass.

Comment lines begin with the characters \texttt{CC} and have the following comment in
quotes. The only restriction to this usage is that another index should not appear in
isolation. For example, a comment \texttt{CC "The electric field index is E "} will
produce an error because the parser will recognize the index E. An appropriate
modification is to prevent the E from appearing in isolation such as \texttt{CC "The
electric field index is \{E\} "}.

Here is a full example indeck:

\begin{lstlisting}
CC "Comment lines start with CC and have comment in quotes"
CC "Length of Box"
L 1e-6
CC "Particle Species Information"
CC "------------------------------"
CC "number of species {SP}"
SP 4
CC "number of particles for each species {NP}"
CC "NP 101250 135000 "
NP 6000 10000 0 0 
CC "temperature of each particle species"
TP  1.0 0.0 0.0 0.0
CC "flow velocity"
VV 50.0 0.0 0.0 0.0
CC "Charge of each particle species"
QP -1.0 0.0 0.0 0.0
CC "Mass of each particle species"
MP 5.4857e-4 28.0 28.0 28.0
CC "-----------------------------"
NS 33001
VS 1000
DT 1.0e-11
CC "note -10000V is 100Td, -100V is 1Td"
E -5000.0
ET 10000000
CC "0.001 = 10Hx"
B 0.0 0.00 0.0
CC " Collision Model Specification "
CC "----------------------"
CC "List cross sections first start with {CS}"
CS Data/N2/1.dat 0 1 ElasticFixedParticle2 0.0 0 1
CS Data/N2/v01.dat 0 1 InelasticChangeParticle2 0.275 0 2
CS Data/N2/v02.dat 0 1 InelasticChangeParticle2 0.59 0 3
CS Data/N2/v10.dat 0 2 InelasticChangeParticle2 0.0  0 1 
CS Data/N2/v20.dat 0 3 InelasticChangeParticle2 0.0  0 1
CS Data/N2/v12.dat 0 2 InelasticChangeParticle2 0.315  0 3
CS Data/N2/v21.dat 0 3 InelasticChangeParticle2 0.0  0 2 
CS Data/N2/1.dat 0 2 ElasticFixedParticle2 0.0 0 2
CS Data/N2/1.dat 0 3 ElasticFixedParticle2 0.0 0 3
CS Data/N2/4.dat 0 1 InelasticFixedParticle2 0.88 0 1
CS Data/N2/5.dat 0 1 InelasticFixedParticle2 1.17 0 1
CS Data/N2/6.dat 0 1 InelasticFixedParticle2 1.47 0 1
CS Data/N2/7.dat 0 1 InelasticFixedParticle2 1.76 0 1
CS Data/N2/8.dat 0 1 InelasticFixedParticle2 2.06 0 1
CS Data/N2/9.dat 0 1 InelasticFixedParticle2 2.35 0 1
CS Data/N2/10.dat 0 1 InelasticFixedParticle2 6.17 0 1
CS Data/N2/11.dat 0 1 InelasticFixedParticle2 7.00 0 1
CS Data/N2/12.dat 0 1 InelasticFixedParticle2 7.35 0 1
CS Data/N2/13.dat 0 1 InelasticFixedParticle2 7.36 0 1
CS Data/N2/14.dat 0 1 InelasticFixedParticle2 7.80 0 1
CS Data/N2/15.dat 0 1 InelasticFixedParticle2 8.16 0 1
CS Data/N2/16.dat 0 1 InelasticFixedParticle2 8.40 0 1
CS Data/N2/17.dat 0 1 InelasticFixedParticle2 8.55 0 1
CS Data/N2/18.dat 0 1 InelasticFixedParticle2 8.89 0 1
CS Data/N2/19.dat 0 1 InelasticFixedParticle2 11.03 0 1
CS Data/N2/20.dat 0 1 InelasticFixedParticle2 11.88 0 1
CS Data/N2/21.dat 0 1 InelasticFixedParticle2 12.25 0 1
CS Data/N2/22.dat 0 1 InelasticFixedParticle2 13.0 0 1
CS Data/N2/23.dat 0 1 InelasticFixedParticle2 15.6 0 1
CC "-----Output Control-----"
CC "Dump particles of type i: {FV} {start time in steps} {Stride in steps} {int type}"
FV 1000 150000 0
\end{lstlisting}

Note that the orderings of all parameters with multiple arguments is
consistent with respect to each species. Typically the first species
is assumed to be electrons.

\section{Cross Section and Collision Model Format}
The index \texttt{CS} specifies the particle interaction pairs, cross section data file,
collision model, reaction products, and threshold energy (or exothermic energy release)
for a reaction. The general form of input for this input is 

\begin{lstlisting}[frame=single]
CS [data] [reactant_1] [reactant_2] [model] [energy] [product_1] [product_2]. 
\end{lstlisting}
The required input options are the following:

\subsection{[data] : The cross section data file}
This is a path to a cross section data file for the energy dependent cross section
$\sigma(E)$ and should be indicated by a string. The file is assumed to be in two column
format with the first column being the energy in eV and the second column the cross
section in $m^2$. For example, the first few lines of the cross section file in
Data/N2/1.dat are the following:

\begin{lstlisting}
 0.000000e+0	1.100000e-20
 1.000000e-3	1.360000e-20
 2.000000e-3	1.490000e-20
 3.000000e-3	1.620000e-20
 2.000000e-3	1.810000e-20
 7.000000e-3	2.000000e-20
 8.500000e-3	2.100000e-20
 1.000000e-2	2.190000e-20
 1.500000e-2	2.550000e-20
 2.000000e-2	2.850000e-20
 3.000000e-2	3.400000e-20
 4.000000e-2	3.850000e-20
 5.000000e-2	4.330000e-20
 7.000000e-2	5.100000e-20
 \end{lstlisting}
 The code linearly interpolates the data for the DSMC collision evaluation so the input
 should be sufficiently well resolved for the intended purpose. 
 
\subsection{[reactant 1]  and [reactant 2]  : Colliding Particle Pairs/Reactants }
This option specifies the colliding particle species taking part in the interaction. Each
species is identified as an integer starting at 0 and counting to $N_{\rm species}-1$.
The species integer identifier is ordered in the same order as  \texttt{NP}, \texttt{MP},
\texttt{QP}, and \texttt{TP}.

\subsection{[product 1]  and [product 2]  : Reaction Products }
This option specifies the reactant species from a successful collision. This data is not
utilized for all collisions, but must be specified for each instance of \texttt{CS}. See
the details for each collision model for more information.

\subsection{[energy]: Threshold/Exothermic Energy }
This option is the threshold energy for a collision to occur and is specified in electron
volts (eV). If the value is negative, it indicates that the reaction releases energy into
the products during the reaction. Here it is assumed that the cross section file is zero
below the threshold energy. It this is not the case some collision model evaluation will
on occasion produce collision pairs where the collision energy is below the threshold for
the inelastic reaction to proceed.  In this case, the code will produce a warning and
will set the reactant energy to zero. The user should ensure that the cross section file
meets the required specification.

\subsection{[model]: Collision Model  }
All collision models utilize the energy dependent cross section file specified by the
file path [data] to evaluate the probability of interaction. This subsection lists the
particle interaction models that specify how energy/momentum is transferred between
particles. The current collision model options are listed in the next section.

\chapter{Description of Available Collision Models}

\section{Isotropic Elastic Collisions} 
In this collision model the energy and momentum exchange between particles is calculated
assuming a uniformly random isotropic post-collision scattering angle. This model is
intended to be used with elastic momentum cross section data, where in many cross
section databases  momentum transfer cross sections are included instead of elastic cross
sections.  These can  be used in conjunction with the isotropic and elastic collision
model to produce transport data that captures the momentum transfer statistics of the
anisotropic scattering when models for the anisotropic scattering angle distribution are unavailable.

\section{Anisotropic Elastic Collisions} 
ThunderBoltz includes the framework needed for the implementation of anisotropic
scattering models. Generally, independent of the particular anisotropic model, the post
collision scattering angle is determined in the center of mass frame. The relative
velocity vector is rotated in the center of mass frame throughout the collision and needs
to be transformed back to the original laboratory (simulation) frame to determine the
post collision velocities. The mechanism for performing this transformation is available
in the code for the implementation of anisotropic models. See the discussion in
Ref.~\cite{donko2021} for details of the procedure.

At present, ThunderBoltz includes the anisotropic elastic scattering model from
Ref.~\cite{Park_2022}. The model provides an invertible angular distribution
function that allows the post collision scattering angle of an electron to be determined
by its relative collision energy and a uniform random number between 0 and 1 such that
the scattering angle distribution of electrons approximates that given by the elastic
differential scattering cross section when collisions are evaluated with an appropriate
elastic integrated scattering cross section. The model currently has known fit parameters
for the inverted angular distribution function and the elastic integrated scattering
cross section of H, He, H$_2$, and their isotopes.

\section{Non-Reactive Collisions} 
\label{Sect:Non-Reactive_Collision}
This collision option can be combined with inelastic collisions and neglects the products
produced in reactions, i.e the reactants are the same as the products.  The appropriate
energy transfer or threshold energy cost of the reaction is calculated and subtracted
from the reactants post collision velocities.  This collision type is useful for
emulating the behavior of Boltzmann solvers by maintaining an unchanging background, or
when the tracking of excited states of some subset of species in reactive kinetics
simulations is unimportant. As an example, if the final density of  
N$_2$($A^3\Sigma_u^+$) is unimportant, the reaction
e~+~N$_2$~$\to$~e~+~N$_2$($A^3\Sigma_u^+$) is modeled as e~+~N$_2$~$\to$~e~+~N$_2$ with
the excitation energy of the $A^3\Sigma_u^+$ state subtracted from the relative collision
energy, but no particles are removed and no new particle products are generated.

\section{Fixed Heavy Particle Collisions} 
\label{Sect:Fixed_Heavy_Particle_Collision}
To calculate transport of a charged species in a fixed background gas it is necessary to
maintain the statistical properties of the background (e.g. temperature, energy, flow and
other moments of the VDF). If energy exchange in elastic or inelastic collisions is
allowed the background will gain energy from the field accelerated particles. For this
reason, it is desirable to have a collision model that fixes the background particles. In
this case, the electron or ion energy loss is calculated as in an elastic collision, but
the neutral heavy particle energy and velocity is not updated, maintaining the
statistical properties of the background gas distribution.

\section{Isotropic Inelastic Collisions}
The energy and momentum exchange between particles is calculated using a uniformly random
isotropic post collision scattering angle and the threshold energy for the inelastic
collision process is subtracted in the center of mass frame. This model can be combined
with the Fixed Heavy Particle and Non-Reactive Collision options as described in
Secs.~\ref{Sect:Non-Reactive_Collision} and \ref{Sect:Fixed_Heavy_Particle_Collision},
or the species type of the reactants is switched to that of the products given in the
collision specification line. Other than ionization, currently all inelastic collisions
use this scattering model.

\section{Electron Impact Ionization}
For ionizing collisions, a model is needed to determine how to partition energy in excess
of the ionization energy between the products. The residual ion maintains the velocity
and direction of the target neutral/ion, while the scattering angle of the product
electrons is assumed to be isotropic. The excess energy available for sharing between the
electrons is calculated by assuming the post-collision residual ion maintains the same
velocity as the pre-collision target neutral/ion. The energy balance in the rest frame of
the target is $\epsilon_{\rm s} = \epsilon - \epsilon_{\rm ion}-\epsilon_{\rm ej}$, where
$\epsilon$, $\epsilon_{\rm s}$, and $\epsilon_{\rm ej}$ are the incident, scattered, and
ejected electron energies, and $\epsilon_{\rm ion}$ is the ionization energy of the bound
electron. Generally, the ejected electron can take on a distribution of values which can
be represented by an electron energy sharing distribution. Three models for this
distribution are implemented.  These are the \emph{one takes all} model: where one
electron is ejected with 0 eV and the other with $\epsilon - \epsilon_{\rm ion}$
\cite{Hagelaar_2005}, the \emph{equal energy sharing} model\cite{Hagelaar_2005}:
where each electron is ejected with energy $(\epsilon - \epsilon_{\rm ion})/2$, and the
\emph{uniform energy sharing} model\cite{CHUNG20053}: where the energy of one electron
has a uniform distribution in the range $[0,(\epsilon - \epsilon_{\rm ion})/2]$ and the
other is determined from conservation of energy.

\bibliographystyle{plain}
\bibliography{cpp_manual}

\end{document}

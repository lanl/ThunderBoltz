\begin{savenotes}\sphinxattablestart
\sphinxthistablewithglobalstyle
\sphinxthistablewithnovlinesstyle
\centering
\begin{tabulary}{\linewidth}[t]{\X{1}{2}\X{1}{2}}
\sphinxtoprule
\sphinxtableatstartofbodyhook
\sphinxAtStartPar
\sphinxcode{\sphinxupquote{B}}
&
\sphinxAtStartPar
(list{[}int{]}) Magnetic field vector (Tesla), default is \sphinxcode{\sphinxupquote{{[}0.0, 0.0, 0.0{]}}}.
\\
\sphinxhline
\sphinxAtStartPar
\sphinxcode{\sphinxupquote{CR}}
&
\sphinxAtStartPar
(int) If \sphinxcode{\sphinxupquote{1}}, then the remainder of \(N_{\rm pairs}\) is carried into the next \(N_{\rm pairs}\) evaluation of the same process, default is \sphinxcode{\sphinxupquote{0}}.
\\
\sphinxhline
\sphinxAtStartPar
\sphinxcode{\sphinxupquote{DT}}
&
\sphinxAtStartPar
(float) Time increment interval (s).
\\
\sphinxhline
\sphinxAtStartPar
\sphinxcode{\sphinxupquote{E}}
&
\sphinxAtStartPar
(float) Electric field in z\sphinxhyphen{}direction (V/m), default is \sphinxcode{\sphinxupquote{\sphinxhyphen{}0}}.
\\
\sphinxhline
\sphinxAtStartPar
\sphinxcode{\sphinxupquote{ET}}
&
\sphinxAtStartPar
(float) E\sphinxhyphen{}field oscillation frequency (Hz), default is \sphinxcode{\sphinxupquote{0}}.
\\
\sphinxhline
\sphinxAtStartPar
\sphinxcode{\sphinxupquote{EX}}
&
\sphinxAtStartPar
(int) Cross section extrapolation — options are \sphinxcode{\sphinxupquote{0}} (extrapolated cross sections are set to \(0\) m \(^2\)), or \sphinxcode{\sphinxupquote{1}} (linearly extrapolated from last two points), default is \sphinxcode{\sphinxupquote{0}}.
\\
\sphinxhline
\sphinxAtStartPar
\sphinxcode{\sphinxupquote{FV}}
&
\sphinxAtStartPar
(list{[}int{]}) Output velocity dump settings {[}start, stride, species ID{]}.
\\
\sphinxhline
\sphinxAtStartPar
\sphinxcode{\sphinxupquote{L}}
&
\sphinxAtStartPar
(float) Cell length (m).
\\
\sphinxhline
\sphinxAtStartPar
\sphinxcode{\sphinxupquote{LV}}
&
\sphinxAtStartPar
(list{[}str,int{]}) Optionally load particle velocities from a comma separated text file; specify the name of the file at index \sphinxcode{\sphinxupquote{0}} and the particle species index it applies to at index \sphinxcode{\sphinxupquote{1}}.
\\
\sphinxhline
\sphinxAtStartPar
\sphinxcode{\sphinxupquote{MEM}}
&
\sphinxAtStartPar
(float) Request memory (GB) for particle arrays.
\\
\sphinxhline
\sphinxAtStartPar
\sphinxcode{\sphinxupquote{MP}}
&
\sphinxAtStartPar
(list{[}float{]}) Mass of each particle species (amu).
\\
\sphinxhline
\sphinxAtStartPar
\sphinxcode{\sphinxupquote{NP}}
&
\sphinxAtStartPar
(list{[}int{]}) Number of particles for each species.
\\
\sphinxhline
\sphinxAtStartPar
\sphinxcode{\sphinxupquote{NS}}
&
\sphinxAtStartPar
(int) Number of time steps.
\\
\sphinxhline
\sphinxAtStartPar
\sphinxcode{\sphinxupquote{OS}}
&
\sphinxAtStartPar
(int) Time step stride for output parameters, default is \sphinxcode{\sphinxupquote{100}}.
\\
\sphinxhline
\sphinxAtStartPar
\sphinxcode{\sphinxupquote{QP}}
&
\sphinxAtStartPar
(list{[}int{]}) Charge (elementary units) of each particle species.
\\
\sphinxhline
\sphinxAtStartPar
\sphinxcode{\sphinxupquote{SE}}
&
\sphinxAtStartPar
(int) When using a SLURM manager on HPC, auto dump particle velocity data before job allocation runs out — options are \sphinxcode{\sphinxupquote{0}} (don\textquotesingle{}t auto dump) | \sphinxcode{\sphinxupquote{1}} (dump using SLURM setup).
\\
\sphinxhline
\sphinxAtStartPar
\sphinxcode{\sphinxupquote{SP}}
&
\sphinxAtStartPar
(int) Number of species.
\\
\sphinxhline
\sphinxAtStartPar
\sphinxcode{\sphinxupquote{TP}}
&
\sphinxAtStartPar
(list{[}float{]}) Temperature (eV) of each particle species.
\\
\sphinxhline
\sphinxAtStartPar
\sphinxcode{\sphinxupquote{VS}}
&
\sphinxAtStartPar
(int) Number of random samples used to find \(\max_\epsilon (v\sigma(\epsilon))\) for each process, default is \sphinxcode{\sphinxupquote{1000}}.
\\
\sphinxhline
\sphinxAtStartPar
\sphinxcode{\sphinxupquote{VV}}
&
\sphinxAtStartPar
(list{[}float{]}) Flow velocity for each particle.
\\
\sphinxbottomrule
\end{tabulary}
\sphinxtableafterendhook\par
\sphinxattableend\end{savenotes}
